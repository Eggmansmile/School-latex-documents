\documentclass[a4paper,11pt]{article}
\usepackage[utf8]{inputenc}
\usepackage{amsmath,amssymb,amsfonts}
\usepackage{physics}
\usepackage{siunitx}
\usepackage{graphicx}
\usepackage{hyperref}
\usepackage{enumitem}

\title{Oefeningen en oplossingen: Fourier, systemen en eigenwaarden}
\author{}
\date{}

\begin{document}
\maketitle
\tableofcontents
\newpage

% ==========================
% OEFENINGEN
% ==========================
\section{Oefeningen}

\subsection{Oefening 1: Fourierreeks van een blokgolf}
Beschouw het periodieke signaal $x(t)$ met periode $T = 2\pi$, gedefinieerd als
\[
x(t) =
\begin{cases}
1, & 0 < t < \pi,\\
-1, & -\pi < t < 0,
\end{cases}
\]
en periodiek voortgezet met periode $2\pi$.

\begin{enumerate}[label=(\alph*)]
  \item Toon aan dat $x(t)$ een oneven functie is.
  \item Bepaal de reële Fourierreeks
  \[
    x(t) = a_0 + \sum_{n=1}^{\infty} \big( a_n \cos(nt) + b_n \sin(nt) \big).
  \]
  \item Schrijf de reeks in compacte vorm enkel met sinustermen.
\end{enumerate}

\subsection{Oefening 2: Eenvoudige Fouriertransformatie}
Gegeven is het signaal
\[
x(t) =
\begin{cases}
1, & |t| \le 1,\\
0, & \text{anders}.
\end{cases}
\]

\begin{enumerate}[label=(\alph*)]
  \item Schrijf de definitie van de continue-tijd Fouriertransformatie $X(\omega)$.
  \item Bepaal $X(\omega)$ expliciet door integratie.
  \item Schets kwalitatief het amplitude\-spectrum $|X(\omega)|$.
\end{enumerate}

\subsection{Oefening 3: Eigenschappen van de Fouriertransformatie}
Laat $x(t)$ een signaal zijn met Fouriertransformatie $X(\omega)$.

\begin{enumerate}[label=(\alph*)]
  \item Geef de uitdrukking voor de Fouriertransformatie van $x(t-t_0)$ in functie van $X(\omega)$.
  \item Geef de uitdrukking voor de Fouriertransformatie van $e^{j\omega_0 t} x(t)$.
  \item Pas beide eigenschappen toe op het resultaat van Oefening 2 voor $x(t-2)$ en $e^{j3t}x(t)$.
\end{enumerate}

\subsection{Oefening 4: LTC-systeem en impulsrespons}
Beschouw een lineair tijdsinvariant continu-tijd systeem (LTC) beschreven door de differentiaalvergelijking
\[
\frac{d y(t)}{dt} + 3 y(t) = x(t).
\]

\begin{enumerate}[label=(\alph*)]
  \item Bepaal de impulsrespons $h(t)$ van het systeem.
  \item Schrijf de uitvoer $y(t)$ voor een willekeurige ingang $x(t)$ in termen van een convolutie met $h(t)$.
  \item Bepaal de systeemfunctie $H(j\omega)$ via Fouriertransformatie.
\end{enumerate}

\subsection{Oefening 5: Convolutie met een exponentiële impulsrespons}
Neem het systeem met impulsrespons
\[
h(t) = e^{-t} u(t),
\]
waar $u(t)$ de eenheidsstap is. De ingang is
\[
x(t) = u(t).
\]

\begin{enumerate}[label=(\alph*)]
  \item Schrijf de definitie van de convolutie $y(t) = (x * h)(t)$.
  \item Bereken $y(t)$ expliciet uit de definitie.
  \item Vergelijk het resultaat met de oplossing van de differentiaalvergelijking
  \[
    \frac{d y(t)}{dt} + y(t) = u(t), \quad y(0^-)=0.
  \]
\end{enumerate}

\subsection{Oefening 6: Frequentierespons}
Voor het systeem met impulsrespons $h(t) = e^{-t} u(t)$:

\begin{enumerate}[label=(\alph*)]
  \item Bepaal de frequentierespons $H(j\omega)$.
  \item Bepaal de amplitude- en fasekarakteristiek.
  \item Bepaal de sinus-in-steady-state respons voor ingang $x(t) = \cos(\omega_0 t)$.
\end{enumerate}

\subsection{Oefening 7: Eigenwaarden en stabiliteit}
Beschouw het toestandsruimtesysteem
\[
\dot{\mathbf{x}}(t) = A \mathbf{x}(t), \quad
A = \begin{bmatrix}
0 & 1\\
-2 & -3
\end{bmatrix}.
\]

\begin{enumerate}[label=(\alph*)]
  \item Bepaal de eigenwaarden van $A$.
  \item Bespreek de (asymptotische) stabiliteit van het systeem.
  \item Schets kwalitatief het gedrag van oplossingen in de tijd.
\end{enumerate}

\subsection{Oefening 8: Eigenwaarden en impulsrespons van een tweede-orde systeem}
Een LTC-systeem heeft overdrachtsfunctie
\[
H(s) = \frac{1}{s^2 + 2\zeta \omega_0 s + \omega_0^2},
\]
met $\omega_0 > 0$ en dempingsfactor $\zeta$.

\begin{enumerate}[label=(\alph*)]
  \item Bepaal de polen van het systeem in functie van $\zeta$ en $\omega_0$.
  \item Bespreek voor de gevallen $\zeta < 1$, $\zeta = 1$ en $\zeta > 1$ het type gedrag (ondergedempt, kritisch gedempt, overgedempt).
  \item Geef de bijhorende impulsresponsvorm (geen volledige afleiding nodig, maar het type functie).
\end{enumerate}

\newpage
% ==========================
% OPLOSSINGEN
% ==========================
\section{Oplossingen}

\subsection*{Oefening 1: Fourierreeks van een blokgolf}

\textbf{Opgave.}\\[2mm]
Beschouw het periodieke signaal $x(t)$ met periode $T = 2\pi$, gedefinieerd als
\[
x(t) =
\begin{cases}
1, & 0 < t < \pi,\\
-1, & -\pi < t < 0,
\end{cases}
\]
en periodiek voortgezet met periode $2\pi$.

\begin{enumerate}[label=(\alph*)]
  \item Toon aan dat $x(t)$ een oneven functie is.
  \item Bepaal de reële Fourierreeks
  \[
    x(t) = a_0 + \sum_{n=1}^{\infty} \big( a_n \cos(nt) + b_n \sin(nt) \big).
  \]
  \item Schrijf de reeks in compacte vorm enkel met sinustermen.
\end{enumerate}

\medskip
\textbf{Oplossing.}

\paragraph{(a) Oneven functie}
We testen de voorwaarde $x(-t) = -x(t)$.

Voor $0 < t < \pi$ geldt $x(t) = 1$. Dan is $-t \in (-\pi,0)$ en dus $x(-t) = -1 = -x(t)$.

Voor $-\pi < t < 0$ geldt $x(t) = -1$. Dan is $-t \in (0,\pi)$ en dus $x(-t) = 1 = -x(t)$.

Dus $x(-t) = -x(t)$ voor alle $t$ in één periode, en dus is $x(t)$ oneven.

\paragraph{(b) Coëfficiënten}
Omdat $x(t)$ oneven is, zijn alle cosinus\-coëfficiënten nul:
\[
a_0 = 0, \quad a_n = 0 \quad \forall n \ge 1.
\]

We bepalen $b_n$:
\[
b_n = \frac{1}{\pi} \int_{-\pi}^{\pi} x(t) \sin(nt)\, dt.
\]
Door de oneven/even-symmetrie kunnen we schrijven
\[
b_n = \frac{2}{\pi} \int_{0}^{\pi} x(t) \sin(nt)\, dt = \frac{2}{\pi} \int_{0}^{\pi} 1 \cdot \sin(nt)\, dt.
\]
We integreren:
\[
\int_{0}^{\pi} \sin(nt)\, dt
= \left[-\frac{\cos(nt)}{n}\right]_{0}^{\pi}
= -\frac{\cos(n\pi) - \cos(0)}{n}
= -\frac{(-1)^n - 1}{n}.
\]
Dus
\[
b_n = \frac{2}{\pi} \left( -\frac{(-1)^n - 1}{n} \right)
= \frac{2}{\pi} \cdot \frac{1 - (-1)^n}{n}.
\]
Merk op:
\[
1 - (-1)^n =
\begin{cases}
0, & n \text{ even},\\
2, & n \text{ oneven}.
\end{cases}
\]
Daarom:
\[
b_n =
\begin{cases}
0, & n \text{ even},\\[4pt]
\displaystyle \frac{4}{\pi n}, & n \text{ oneven}.
\end{cases}
\]

\paragraph{(c) Compacte vorm}
De Fourierreeks wordt
\[
x(t) = \sum_{\substack{n=1\\ n \text{ oneven}}}^{\infty} \frac{4}{\pi n} \sin(nt)
= \frac{4}{\pi} \left( \sin t + \frac{1}{3}\sin(3t) + \frac{1}{5}\sin(5t) + \cdots \right).
\]

\subsection*{Oefening 2: Eenvoudige Fouriertransformatie}

\textbf{Opgave.}\\[2mm]
Gegeven is het signaal
\[
x(t) =
\begin{cases}
1, & |t| \le 1,\\
0, & \text{anders}.
\end{cases}
\]

\begin{enumerate}[label=(\alph*)]
  \item Schrijf de definitie van de continue-tijd Fouriertransformatie $X(\omega)$.
  \item Bepaal $X(\omega)$ expliciet door integratie.
  \item Schets kwalitatief het amplitude\-spectrum $|X(\omega)|$.
\end{enumerate}

\medskip
\textbf{Oplossing.}

\paragraph{(a) Definitie}
We gebruiken de conventie
\[
X(\omega) = \int_{-\infty}^{\infty} x(t) e^{-j\omega t} \, dt.
\]

\paragraph{(b) Berekening}
Omdat $x(t) = 1$ voor $|t|\le 1$ en $0$ elders:
\[
X(\omega) = \int_{-1}^{1} 1 \cdot e^{-j\omega t} \, dt.
\]
We integreren:
\[
X(\omega) = \left[ \frac{e^{-j\omega t}}{-j\omega} \right]_{-1}^{1}
= \frac{1}{-j\omega}\left( e^{-j\omega} - e^{j\omega} \right).
\]
We herkennen:
\[
e^{-j\omega} - e^{j\omega} = -2j \sin(\omega).
\]
Dus
\[
X(\omega)
= \frac{1}{-j\omega}(-2j \sin\omega)
= \frac{2\sin\omega}{\omega}.
\]
Met de sinc\-notatie (zonder normalisatie) kunnen we schrijven:
\[
X(\omega) = 2\,\frac{\sin\omega}{\omega}.
\]

\paragraph{(c) Amplitudespectrum}
Het amplitudespectrum is
\[
|X(\omega)| = \left| 2\,\frac{\sin\omega}{\omega} \right|.
\]
Dit is een even functie van $\omega$, met een hoofdmaximum $|X(0)| = 2$ (limiet), en nulpunten bij $\omega = \pm n\pi$ met $n=1,2,\dots$.

\subsection*{Oefening 3: Eigenschappen van de Fouriertransformatie}

\textbf{Opgave.}\\[2mm]
Laat $x(t)$ een signaal zijn met Fouriertransformatie $X(\omega)$.

\begin{enumerate}[label=(\alph*)]
  \item Geef de uitdrukking voor de Fouriertransformatie van $x(t-t_0)$ in functie van $X(\omega)$.
  \item Geef de uitdrukking voor de Fouriertransformatie van $e^{j\omega_0 t} x(t)$.
  \item Pas beide eigenschappen toe op het resultaat van Oefening 2 voor $x(t-2)$ en $e^{j3t}x(t)$.
\end{enumerate}

\medskip
\textbf{Oplossing.}

\paragraph{(a) Tijdverschuiving}
Als $x(t) \longleftrightarrow X(\omega)$, dan geldt:
\[
x(t - t_0) \longleftrightarrow e^{-j\omega t_0} X(\omega).
\]

\paragraph{(b) Modulatie}
Voor modulatie met $e^{j\omega_0 t}$:
\[
e^{j\omega_0 t} x(t) \longleftrightarrow X(\omega - \omega_0).
\]

\paragraph{(c) Toepassing op Oefening 2}
Voor $x(t-2)$:
\[
x(t-2) \longleftrightarrow e^{-j\omega \cdot 2} X(\omega)
= e^{-j2\omega} \cdot 2 \frac{\sin\omega}{\omega}.
\]

Voor $e^{j3t} x(t)$:
\[
e^{j3t}x(t) \longleftrightarrow X(\omega - 3)
= 2\,\frac{\sin(\omega - 3)}{\omega - 3}.
\]

\subsection*{Oefening 4: LTC-systeem en impulsrespons}

\textbf{Opgave.}\\[2mm]
Beschouw een lineair tijdsinvariant continu-tijd systeem (LTC) beschreven door de differentiaalvergelijking
\[
\frac{d y(t)}{dt} + 3 y(t) = x(t).
\]

\begin{enumerate}[label=(\alph*)]
  \item Bepaal de impulsrespons $h(t)$ van het systeem.
  \item Schrijf de uitvoer $y(t)$ voor een willekeurige ingang $x(t)$ in termen van een convolutie met $h(t)$.
  \item Bepaal de systeemfunctie $H(j\omega)$ via Fouriertransformatie.
\end{enumerate}

\medskip
\textbf{Oplossing.}

\paragraph{(a) Impulsrespons}
We hebben
\[
\frac{dy(t)}{dt} + 3y(t) = x(t).
\]
Voor de impulsrespons nemen we $x(t) = \delta(t)$ en uitgang $h(t)$:
\[
\frac{dh(t)}{dt} + 3h(t) = \delta(t).
\]
Voor $t>0$ is de rechterkant $0$, dus
\[
\frac{dh(t)}{dt} + 3h(t) = 0.
\]
De oplossing heeft de vorm $h(t) = C e^{-3t}$ voor $t>0$. Uit de impulsvoorwaarde (of met Laplace) volgt dat $h(t)$ causaal is met
\[
H(s) = \frac{1}{s+3} \Rightarrow h(t) = e^{-3t}u(t).
\]

\paragraph{(b) Convolutie-uitdrukking}
Voor een willekeurige ingang $x(t)$:
\[
y(t) = (x*h)(t) = \int_{-\infty}^{\infty} x(\tau) h(t-\tau)\, d\tau.
\]

\paragraph{(c) Systeemfunctie}
Met Fouriertransformatie (of Laplace $s = j\omega$):
\[
H(j\omega) = \frac{1}{j\omega + 3}.
\]

\subsection*{Oefening 5: Convolutie met een exponentiële impulsrespons}

\textbf{Opgave.}\\[2mm]
Neem het systeem met impulsrespons
\[
h(t) = e^{-t} u(t),
\]
waar $u(t)$ de eenheidsstap is. De ingang is
\[
x(t) = u(t).
\]

\begin{enumerate}[label=(\alph*)]
  \item Schrijf de definitie van de convolutie $y(t) = (x * h)(t)$.
  \item Bereken $y(t)$ expliciet uit de definitie.
  \item Vergelijk het resultaat met de oplossing van de differentiaalvergelijking
  \[
    \frac{d y(t)}{dt} + y(t) = u(t), \quad y(0^-)=0.
  \]
\end{enumerate}

\medskip
\textbf{Oplossing.}

\paragraph{(a) Definitie}
\[
y(t) = (x*h)(t) = \int_{-\infty}^{\infty} x(\tau) h(t-\tau)\, d\tau.
\]

\paragraph{(b) Berekening}
We hebben $x(t)=u(t)$ en $h(t)=e^{-t}u(t)$.

Voor $t<0$ is $x(\tau)=0$ of $h(t-\tau)=0$, dus $y(t)=0$.

Voor $t\ge 0$:
\[
y(t) = \int_{0}^{t} u(\tau)\, e^{-(t-\tau)} u(t-\tau)\, d\tau
= \int_{0}^{t} e^{-(t-\tau)} d\tau.
\]
Laat $v = t-\tau$, dan loopt $\tau:0\to t$ overeen met $v:t\to 0$:
\[
y(t) = \int_{t}^{0} e^{-v} (-dv) = \int_{0}^{t} e^{-v} dv
= \left[ -e^{-v}\right]_{0}^{t} = 1 - e^{-t}.
\]
Dus
\[
y(t) = (1 - e^{-t}) u(t).
\]

\paragraph{(c) Vergelijking met differentiaalvergelijking}
De differentiaalvergelijking
\[
\frac{dy(t)}{dt} + y(t) = u(t), \quad y(0^-)=0
\]
heeft als klassieke oplossing
\[
y(t) = 1 - e^{-t} \quad \text{voor } t\ge 0,
\]
en $y(t)=0$ voor $t<0$. Dit komt overeen met het convolutieresultaat.

\subsection*{Oefening 6: Frequentierespons}

\textbf{Opgave.}\\[2mm]
Voor het systeem met impulsrespons $h(t) = e^{-t} u(t)$:

\begin{enumerate}[label=(\alph*)]
  \item Bepaal de frequentierespons $H(j\omega)$.
  \item Bepaal de amplitude- en fasekarakteristiek.
  \item Bepaal de sinus-in-steady-state respons voor ingang $x(t) = \cos(\omega_0 t)$.
\end{enumerate}

\medskip
\textbf{Oplossing.}

\paragraph{(a) $H(j\omega)$}
Fouriertransformatie van $h(t) = e^{-t}u(t)$:
\[
H(j\omega) = \int_{0}^{\infty} e^{-t} e^{-j\omega t}\, dt
= \int_{0}^{\infty} e^{-(1+j\omega)t}\, dt
= \frac{1}{1 + j\omega},
\]
als $\Re(1)>0$.

\paragraph{(b) Amplitude- en fasekarakteristiek}
We schrijven
\[
H(j\omega) = \frac{1}{1 + j\omega}.
\]
De amplitude is
\[
|H(j\omega)| = \frac{1}{\sqrt{1+\omega^2}}.
\]
De fase is
\[
\arg H(j\omega) = -\arctan(\omega).
\]

\paragraph{(c) Sinus-in-steady-state respons}
Voor ingang $x(t) = \cos(\omega_0 t)$ schrijven we
\[
x(t) = \Re\{ e^{j\omega_0 t}\}.
\]
In steady state geeft het systeem een uitgang
\[
y_{\text{ss}}(t) = |H(j\omega_0)| \cos(\omega_0 t + \arg H(j\omega_0)).
\]
Dus
\[
y_{\text{ss}}(t) = \frac{1}{\sqrt{1+\omega_0^2}} \cos\big(\omega_0 t - \arctan(\omega_0)\big).
\]

\subsection*{Oefening 7: Eigenwaarden en stabiliteit}

\textbf{Opgave.}\\[2mm]
Beschouw het toestandsruimtesysteem
\[
\dot{\mathbf{x}}(t) = A \mathbf{x}(t), \quad
A = \begin{bmatrix}
0 & 1\\
-2 & -3
\end{bmatrix}.
\]

\begin{enumerate}[label=(\alph*)]
  \item Bepaal de eigenwaarden van $A$.
  \item Bespreek de (asymptotische) stabiliteit van het systeem.
  \item Schets kwalitatief het gedrag van oplossingen in de tijd.
\end{enumerate}

\medskip
\textbf{Oplossing.}

\paragraph{(a) Eigenwaarden}
We lossen
\[
\det(\lambda I - A) = 0.
\]
Met
\[
A = \begin{bmatrix}
0 & 1\\
-2 & -3
\end{bmatrix},
\quad
\lambda I - A =
\begin{bmatrix}
\lambda & -1\\
2 & \lambda + 3
\end{bmatrix},
\]
krijgen we
\[
\det(\lambda I - A) = \lambda(\lambda+3) - (-1)\cdot 2
= \lambda^2 + 3\lambda + 2.
\]
We lossen
\[
\lambda^2 + 3\lambda + 2 = 0
\Rightarrow (\lambda+1)(\lambda+2) = 0.
\]
Dus
\[
\lambda_1 = -1, \quad \lambda_2 = -2.
\]

\paragraph{(b) Stabiliteit}
Beide eigenwaarden hebben negatieve reële delen, dus het systeem is asymptotisch stabiel: alle oplossingen gaan naar $0$ als $t\to\infty$.

\paragraph{(c) Kwalitatief gedrag}
Omdat de eigenwaarden reëel en negatief zijn, zijn de trajecten in de toestandsruimte exponentieel dalend zonder oscillaties. Componenten langs de eigenvector bij $\lambda=-2$ dalen sneller dan langs de eigenvector bij $\lambda=-1$.

\subsection*{Oefening 8: Eigenwaarden en impulsrespons van een tweede-orde systeem}

\textbf{Opgave.}\\[2mm]
Een LTC-systeem heeft overdrachtsfunctie
\[
H(s) = \frac{1}{s^2 + 2\zeta \omega_0 s + \omega_0^2},
\]
met $\omega_0 > 0$ en dempingsfactor $\zeta$.

\begin{enumerate}[label=(\alph*)]
  \item Bepaal de polen van het systeem in functie van $\zeta$ en $\omega_0$.
  \item Bespreek voor de gevallen $\zeta < 1$, $\zeta = 1$ en $\zeta > 1$ het type gedrag (ondergedempt, kritisch gedempt, overgedempt).
  \item Geef de bijhorende impulsresponsvorm (geen volledige afleiding nodig, maar het type functie).
\end{enumerate}

\medskip
\textbf{Oplossing.}

\paragraph{(a) Polen}
De polen zijn de oplossingen van
\[
s^2 + 2\zeta \omega_0 s + \omega_0^2 = 0.
\]
Met de bekende formule:
\[
s_{1,2} = -\zeta \omega_0 \pm \omega_0 \sqrt{\zeta^2 - 1}.
\]

\paragraph{(b) Gedrag}
\begin{itemize}
  \item $\zeta < 1$: de term onder de wortel is negatief, dus complexe geconjugeerde polen. Dit geeft een ondergedempt (oscillerend) gedrag met exponentieel afnemende sinusoïden.
  \item $\zeta = 1$: dubbele reële pool $s = -\omega_0$. Dit is kritisch gedempt, snel naar evenwicht zonder overshoot.
  \item $\zeta > 1$: twee verschillende reële negatieve polen. Overgedempt, geen oscillaties, maar trager dan kritisch gedempt.
\end{itemize}

\paragraph{(c) Impulsresponsvorm}
\begin{itemize}
  \item $\zeta < 1$: impulsrespons van de vorm
  \[
  h(t) = \frac{\omega_0}{\sqrt{1-\zeta^2}} e^{-\zeta \omega_0 t} \sin(\omega_d t)\, u(t),
  \]
  met $\omega_d = \omega_0\sqrt{1-\zeta^2}$.
  \item $\zeta = 1$: impulsrespons van de vorm
  \[
  h(t) = (C_1 + C_2 t) e^{-\omega_0 t} u(t).
  \]
  \item $\zeta > 1$: som van twee exponenten:
  \[
  h(t) = C_1 e^{s_1 t} u(t) + C_2 e^{s_2 t} u(t),
  \]
  met $s_1,s_2$ de reële polen.
\end{itemize}

\end{document}
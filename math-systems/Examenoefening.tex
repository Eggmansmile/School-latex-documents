\documentclass[12pt,a4paper,twoside]{book}


\newcommand{\oefening}{}
\newcommand{\oefeningheader}{\textbf{Oefening: \oefening }\vspace{0.3cm}\hrule\vspace{0.5cm}\newline}
\begin{document}

\section{LTC-systemenen}



\renewcommand{\oefening}{1}
\oefeningheader

Geef voor de volgende systeemfuncties aan of het een laagdoorlaat-, hoogdoorlaat-
, banddoorlaat- of bandstopsysteem is, of een fasedraaier (all-pass-systeem).
Geef ook de orde, de versterking op DC en de versterking op oneindig.


(a) H(s) =
8s/
s^2+8s +25

neem het limiet naar nul en naar oneindig en vervang s met jw

lim s->0 H(s) = 0 => hoogdoorlaat
lim s->∞ H(s) = 0 => laagdoorlaat




\renewcommand{\oefening}{2}
\oefeningheader
Om de systeemrespons $y(t)$ te bepalen voor de gegeven differentiaalvergelijking,
 volgen we een stappenplan waarbij we de algemene oplossing opsplitsen
 in een homogeen deel ($y_h$) en een particulier deel ($y_p$). De totale oplossing is dan $y(t) = y_h(t) + y_p(t)$.De gegeven vergelijking is:$$y''(t) + 2y'(t) + 10y = 0.52 \sin(2t)$$
Met beginvoorwaarden: $y(0) = 0.02$ en $y'(0) = 0$.

. De homogene oplossing ($y_h$)We lossen eerst 
de homogene vergelijking $y'' + 2y' + 10y = 0$ op.
 Hiervoor stellen we de karakteristieke vergelijking 
op door $y = e^{\lambda t}$ in te vullen:$$\lambda^2 + 2\lambda + 10 = 0$$

\renewcommand{\oefening}{3}
\oefeningheader
Geven de impulsrespons h(t) van een LTC-systeem. Bepaal de transferfunctie, de polen en nulpunten en de differentiaalvergelijking. Schets de
amplitude- en faserespons en het poolnulpuntendiagram.
a) h(t) = e^−tsin(2t)u(t)
b) h(t) = t e−t u(t)
c) h(t) = t sin(2t)u(t

a) 
laplacetransformatie van h(t):
H(s) = L{e^−tsin(2t)u(t)} = 2/((s+1)^2 + 4)


\oefening{4}
Laten we kijken hoe we de respons van een systeem bepalen in het Laplacedomein.
 In de kern draait dit om de formule:$$Y(s) = H(s) \cdot X(s)$$Hierbij is $Y(s)$ 
 de output, $H(s)$
 de transferfunctie van het systeem en $X(s)$ het getransformeerde inputsignaal.


 Opgave 6: Systeemrespons bepalenGegeven is de transferfunctie $H(s) = \frac{2}{s+1}$.
  Om de tijdrespons $y(t)$ te vinden,
   volgen we drie stappen:Transformeer 
   de input $x(t)$ naar $X(s)$.
   Vermenigvuldig $H(s)$ met $X(s)$ om $Y(s)$ te krijgen.
   Pas de inverse Laplace-transformatie toe om terug te gaan 
   naar $y(t)$.
   (a) Staprespons: $x(t) = u(t)$Input: 
   $X(s) = \frac{1}{s}$Output in s-domein: $Y(s)
    = \frac{2}{s+1} \cdot \frac{1}{s} = \frac{2}{s(s+1)}$Breuksplitsen: $\frac{2}{s(s+1)} = \frac{2}{s} - \frac{2}{s+1}$Tijdrespons: $y(t) = 2(1 - e^{-t})u(t)$µ
(b) Sinusrespons: $x(t) = \sin(2t)u(t)$Input: $X(s) = \frac{2}{s^2+4}$Output in s-domein: $Y(s) = \frac{2}{s+1} \cdot \frac{2}{s^2+4} = \frac{4}{(s+1)(s^2+4)}$Na breuksplitsen en inverteren krijg je een combinatie van een uitstervende exponent (transiënt) en een constante sinus (stationaire toestand).
(d) Exponentiële respons: $x(t) = e^{-t}u(t)$Input: $X(s) = \frac{1}{s+1}$Output in s-domein: $Y(s) = \frac{2}{s+1} \cdot \frac{1}{s+1} = \frac{2}{(s+1)^2}$Tijdrespons: Gebruik de eigenschap $\mathcal{L}\{t e^{-at}\} = \frac{1}{(s+a)^2}$.Dit geeft: $y(t) = 2t e^{-t} u(t)$.

\section{Examenoefening}


\end{document}

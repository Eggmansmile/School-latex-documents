\documentclass[a4paper,11pt]{article}

\usepackage[dutch]{babel}
\usepackage[utf8]{inputenc}
\usepackage{amsmath}
\usepackage{amssymb}
\usepackage{geometry}
\usepackage{enumitem}
\usepackage{graphicx}

\geometry{margin=2.5cm}

\title{Oefeningen Wiskunde voor Systemen}
\author{KU Leuven - ESAT}
\date{\today}

\begin{document}

\maketitle
\tableofcontents
\newpage

\section{Hoofdstuk 1: Signalen en Systemen - Eerste Kennismaking}

\subsection{Oefening 1.1: Lineaire Systemen}

Gegeven een systeem met operator $\mathcal{T}$ gedefinieerd als $\mathcal{T}\{x(t)\} = 3x(t) + 2$.

\textbf{Vraag:} Onderzoek of dit systeem lineair is.

\subsection{Oefening 1.2: RC-Circuit}

Een RC-circuit heeft $R = 1000$ $\Omega$ en $C = 10$ $\mu$F. De ingangsspanning is een stapfunctie $v_{\text{in}}(t) = 5u(t)$ V.

\textbf{Vraag:}
\begin{enumerate}[label=(\alph*)]
\item Schrijf de differentiaalvergelijking op voor de uitgangsspanning $v_{\text{uit}}(t)$.
\item Bereken de tijdsconstante $\tau$ van het circuit.
\item Bepaal de uitgangsspanning na $10$ ms als $v_{\text{uit}}(0) = 0$ V.
\end{enumerate}

\subsection{Oefening 1.3: Radioactief Verval}

Een radio-isotoop heeft een halveringstijd van $6$ uur. Om 08:00 uur wordt 20 mg geproduceerd.

\textbf{Vraag:} Hoeveel milligram blijft over om 14:00 uur?

\subsection{Oefening 1.4: Homogeniteit en Additiviteit}

Gegeven twee systemen:
\begin{itemize}
\item Systeem A: $\mathcal{T}\{x(t)\} = 2x(t)$
\item Systeem B: $\mathcal{T}\{x(t)\} = x(t) + 1$
\end{itemize}

\textbf{Vraag:}
\begin{enumerate}[label=(\alph*)]
\item Test beide systemen op homogeniteit (schaling): $\mathcal{T}\{ax(t)\} = a\mathcal{T}\{x(t)\}$.
\item Test beide systemen op additiviteit: $\mathcal{T}\{x_1(t) + x_2(t)\} = \mathcal{T}\{x_1(t)\} + \mathcal{T}\{x_2(t)\}$.
\item Bepaal voor elk systeem of het lineair is.
\end{enumerate}

\subsection{Oefening 1.5: LTI-Systeem Herkenning}

Welke van de volgende systemen zijn lineair en tijdinvariant (LTI)?

\begin{enumerate}[label=(\alph*)]
\item $y(t) = x(t-2)$
\item $y(t) = tx(t)$
\item $y(t) = |x(t)|$
\item $y(t) = \int_0^t x(\tau) d\tau$
\end{enumerate}

\textbf{Vraag:} Motiveer je antwoorden.

\newpage

\section{Hoofdstuk 2: Basissignalen en Bewerkingen}

\subsection{Oefening 2.1: Exponenti\"ele Functies}

Gegeven de signalen $x_1(t) = e^{0.2t}$ en $x_2(t) = e^{-0.5t}$.

\textbf{Vraag:}
\begin{enumerate}[label=(\alph*)]
\item Bepaal welk signaal exponenti\"ele groei en welk exponentieel verval vertoont.
\item Bereken de waarde van elk signaal op $t = 5$ s.
\end{enumerate}

\subsection{Oefening 2.2: Sinus en Cosinus}

Een sinusgolf is gegeven door $x(t) = 3\sin(4\pi t + \frac{\pi}{6})$.

\textbf{Vraag:}
\begin{enumerate}[label=(\alph*)]
\item Bepaal de amplitude, hoekfrequentie $\omega$, frequentie $f$, en fasehoek.
\item Schrijf dit signaal als een cosinusfunctie.
\end{enumerate}

\subsection{Oefening 2.3: Complexe Exponenti\"ele Functie}

Gegeven $z(t) = e^{j2\pi t}$.

\textbf{Vraag:}
\begin{enumerate}[label=(\alph*)]
\item Schrijf dit signaal in termen van sinus en cosinus gebruikmakend van de formule van Euler.
\item Bepaal de waarde op $t = 0.25$ s.
\end{enumerate}

\subsection{Oefening 2.4: Convolutie}

Bereken de convolutie van twee pulssignalen:
\[
f(t) = u(t) - u(t-1), \quad g(t) = u(t) - u(t-1)
\]

\textbf{Vraag:} Bepaal $(f * g)(t)$ en schets het resultaat.

\subsection{Oefening 2.5: Signaalverschuiving en Schaling}

Gegeven het signaal $x(t) = e^{-t}u(t)$.

\textbf{Vraag:}
\begin{enumerate}[label=(\alph*)]
\item Bepaal $y_1(t) = x(t-2)$ (tijdsverschuiving).
\item Bepaal $y_2(t) = x(2t)$ (tijdscompressie).
\item Bepaal $y_3(t) = 2x(t)$ (amplitude schaling).
\item Schets alle drie signalen.
\end{enumerate}

\subsection{Oefening 2.6: Signaalenergie}

Bepaal de energie van de volgende signalen:

\textbf{Vraag:}
\begin{enumerate}[label=(\alph*)]
\item $x(t) = e^{-t}u(t)$
\item $x(t) = 2\sin(t)u(t)$ over $0 \le t \le \pi$
\item $x(t) = \text{rect}(t) = u(t+0.5) - u(t-0.5)$
\end{enumerate}

\newpage

\section{Hoofdstuk 3: Laplacetransformatie}

\subsection{Oefening 3.1: Eenvoudige Laplacetransformaties}

Bepaal de Laplacetransformatie van de volgende functies:

\textbf{Vraag:}
\begin{enumerate}[label=(\alph*)]
\item $f(t) = 5u(t)$
\item $f(t) = e^{-3t}u(t)$
\item $f(t) = t \cdot u(t)$
\item $f(t) = \cos(5t) \cdot u(t)$
\end{enumerate}

\subsection{Oefening 3.2: Inverse Laplacetransformatie}

Bepaal de inverse Laplacetransformatie van:
\[
F(s) = \frac{3}{s+2} + \frac{5}{s^2 + 4}
\]

\textbf{Vraag:} Vind $f(t)$.

\subsection{Oefening 3.3: Eerste-Orde Systeem}

Los de volgende differentiaalvergelijking op met Laplacetransformatie:
\[
\frac{dy}{dt} + 4y = 8u(t), \quad y(0) = 2
\]

\textbf{Vraag:} Bepaal $y(t)$.

\subsection{Oefening 3.4: Tweede-Orde Systeem}

Een massa-veer-dempersysteem wordt beschreven door:
\[
\frac{d^2y}{dt^2} + 4\frac{dy}{dt} + 3y = 0, \quad y(0) = 1, \quad y'(0) = 0
\]

\textbf{Vraag:}
\begin{enumerate}[label=(\alph*)]
\item Bepaal de karakteristieke vergelijking.
\item Vind de wortels van de karakteristieke vergelijking.
\item Los de differentiaalvergelijking op voor $y(t)$.
\end{enumerate}

\subsection{Oefening 3.5: Laplace-transformatie met Verschuiving}

Gegeven $F(s) = \frac{2}{s^2 + 4}$.

\textbf{Vraag:}
\begin{enumerate}[label=(\alph*)]
\item Bepaal $f(t) = \mathcal{L}^{-1}\{F(s)\}$.
\item Bepaal $g(t) = \mathcal{L}^{-1}\{e^{-2s}F(s)\}$ gebruikmakend van de tijdsverschuivingsstelling.
\end{enumerate}

\subsection{Oefening 3.6: Partieelbreuken}

Bepaal de inverse Laplacetransformatie van:
\[
F(s) = \frac{10}{(s+1)(s+2)(s+3)}
\]

\textbf{Vraag:} Bepaal $f(t)$ via partieelbreukontwikkeling.

\newpage

\section{Hoofdstuk 4: Fouriertransformatie}

\subsection{Oefening 4.1: Fouriertransformatie van Rechthoekpuls}

Gegeven een rechthoekpuls:
\[
f(t) = \begin{cases}
A, & -T/2 < t < T/2 \\
0, & \text{elsewhere}
\end{cases}
\]

\textbf{Vraag:}
\begin{enumerate}[label=(\alph*)]
\item Bepaal de Fouriertransformatie $F(j\omega)$.
\item Schrijf het resultaat in de vorm van een sinc-functie.
\item Bepaal de eerste nulpunten van het spectrum.
\end{enumerate}

\subsection{Oefening 4.2: Verschuivingsstelling}

Gegeven $F\{f(t)\} = F(j\omega)$ bepaal de Fouriertransformatie van $f(t-t_0)$.

\textbf{Vraag:}
\begin{enumerate}[label=(\alph*)]
\item Geef het bewijs van de verschuivingsstelling.
\item Pas deze toe op de puls uit oefening 4.1 met $t_0 = 1$ s, $A = 2$, $T = 2$ s.
\item Bespreek het effect op amplitude- en fasespectrum.
\end{enumerate}

\subsection{Oefening 4.3: Modulation Property}

Bepaal de Fouriertransformatie van het gemoduleerde signaal:
\[
x(t) = \cos(10\pi t) \cdot \text{rect}(t)
\]

waarij $\text{rect}(t) = u(t+1) - u(t-1)$ is.

\textbf{Vraag:}
\begin{enumerate}[label=(\alph*)]
\item Pas de modulatiestelling toe.
\item Schets het amplitude- en fasespectrum.
\end{enumerate}

\subsection{Oefening 4.4: Parsevals Stelling}

De energiedichtheid van een signaal wordt gegeven door Parsevals stelling. Gegeven $f(t) = e^{-t}u(t)$.

\textbf{Vraag:}
\begin{enumerate}[label=(\alph*)]
\item Bepaal de totale energie in het tijdsdomein: $E = \int_{-\infty}^{\infty} |f(t)|^2 dt$.
\item Bepaal $F(j\omega)$ en controleer de energie in het frequentiedomein.
\item Verifieer Parsevals stelling.
\end{enumerate}

\subsection{Oefening 4.5: Exponentieel Signaal}

Gegeven $f(t) = e^{-a|t|}$ met $a > 0$.

\textbf{Vraag:}
\begin{enumerate}[label=(\alph*)]
\item Bepaal de Fouriertransformatie $F(j\omega)$.
\item Schets het amplitude- en fasespectrum.
\item Bepaal de bandbreedte (eerste nulpunt).
\end{enumerate}

\subsection{Oefening 4.6: Dirac Delta}

Bepaal de Fouriertransformatie van:

\textbf{Vraag:}
\begin{enumerate}[label=(\alph*)]
\item $f(t) = \delta(t)$ (impulsfunctie)
\item $f(t) = \delta(t-t_0)$ (verschoven impulsfunctie)
\item $f(t) = \cos(\omega_0 t)$ (hint: gebruik dat $\cos(\omega_0 t) = \frac{1}{2}(e^{j\omega_0 t} + e^{-j\omega_0 t})$)
\end{enumerate}

\newpage

\section{Hoofdstuk 5: Fourierreeksen}

\subsection{Oefening 5.1: Blokgolf}

Een periodieke blokgolf met periode $T = 2$ s is gedefinieerd als:
\[
f(t) = \begin{cases}
1, & 0 < t < 1 \\
-1, & 1 < t < 2
\end{cases}
\]

\textbf{Vraag:}
\begin{enumerate}[label=(\alph*)]
\item Bepaal of de functie even, oneven, of geen van beide is.
\item Bereken de Fourierco\"effici\"enten $a_0$, $a_n$, en $b_n$.
\item Schrijf de Fourierreeks tot de 3e harmonische.
\end{enumerate}

\subsection{Oefening 5.2: Zaagtandgolf}

Een zaagtandgolf met periode $T = 1$ s is gegeven door $f(t) = 2t$ voor $0 < t < 1$.

\textbf{Vraag:}
\begin{enumerate}[label=(\alph*)]
\item Bereken $a_0$.
\item Bepaal $b_1$ en $b_2$.
\item Schrijf de benaderende Fouriersom met 2 termen.
\end{enumerate}

\subsection{Oefening 5.3: Driehoekgolf}

Een driehoekgolf met periode $T = 2$ is gedefinieerd als:
\[
f(t) = \begin{cases}
t, & 0 \le t < 1 \\
2-t, & 1 \le t < 2
\end{cases}
\]

\textbf{Vraag:}
\begin{enumerate}[label=(\alph*)]
\item Is dit signaal even of oneven?
\item Bepaal de Fourierco\"effici\"enten.
\item Schrijf de eerste drie niet-nul termen van de Fourierreeks.
\end{enumerate}

\subsection{Oefening 5.4: Parseval's Stelling voor Fourierreeksen}

Gegeven de blokgolf uit oefening 5.1.

\textbf{Vraag:}
\begin{enumerate}[label=(\alph*)]
\item Bereken de gemiddelde macht van het signaal: $P = \frac{1}{T}\int_0^T f^2(t) dt$.
\item Bereken $P$ uit de Fourierco\"effici\"enten met Parsevals stelling: $P = a_0^2 + \frac{1}{2}\sum_{n=1}^{\infty} (a_n^2 + b_n^2)$.
\item Controlleer dat beide methoden dezelfde waarde geven.
\end{enumerate}

\newpage

\section{Hoofdstuk 6: LTC-Systemen}

\subsection{Oefening 6.1: Impulsrespons}

Een eerste-orde systeem heeft impulsrespons $h(t) = 2e^{-5t}u(t)$.

\textbf{Vraag:}
\begin{enumerate}[label=(\alph*)]
\item Bereken de systeemrespons op een stapingang $f(t) = u(t)$ door convolutie.
\item Verifieer je antwoord met de Laplacetransformatie.
\end{enumerate}

\subsection{Oefening 6.2: Massa-Veer-Demper}

Een massa-veer-dempersysteem heeft $m = 2$ kg, $k = 8$ N/m, en $c = 4$ Ns/m.

\textbf{Vraag:}
\begin{enumerate}[label=(\alph*)]
\item Schrijf de differentiaalvergelijking.
\item Bepaal de natuurlijke eigenfrequentie $\omega_0$.
\item Is het systeem ondergedempt, kritisch gedempt, of overgedempt?
\end{enumerate}

\subsection{Oefening 6.3: Frequentierespons}

Gegeven een LTC-systeem met overdracht $H(s) = \frac{10}{s+5}$.

\textbf{Vraag:}
\begin{enumerate}[label=(\alph*)]
\item Bepaal de frequentierespons $H(j\omega)$.
\item Bepaal de amplitude- en faserespons.
\item Bepaal de 3dB bandbreedte (waar $|H(j\omega)| = \frac{|H(0)|}{\sqrt{2}}$).
\end{enumerate}

\subsection{Oefening 6.4: Cascade Systemen}

Gegeven twee LTC-systemen in cascade:
\[
H_1(s) = \frac{5}{s+2}, \quad H_2(s) = \frac{3}{s+3}
\]

\textbf{Vraag:}
\begin{enumerate}[label=(\alph*)]
\item Bepaal de totale overdracht $H(s) = H_1(s) \cdot H_2(s)$.
\item Bepaal de impulsrespons $h(t) = \mathcal{L}^{-1}\{H(s)\}$.
\end{enumerate}

\newpage

\section{Hoofdstuk 7: Eigenwaarden en Eigenvectoren}

\subsection{Oefening 7.1: Eigenwaarden Berekenen}

Gegeven de matrix:
\[
A = \begin{pmatrix} 4 & 1 \\ 2 & 3 \end{pmatrix}
\]

\textbf{Vraag:}
\begin{enumerate}[label=(\alph*)]
\item Bepaal de karakteristieke veelterm $f(\lambda) = |A - \lambda I|$.
\item Vind de eigenwaarden van $A$.
\item Bereken voor elke eigenwaarde een bijbehorende eigenvector.
\end{enumerate}

\subsection{Oefening 7.2: Eigenwaarden van Tweede-Orde Systeem}

Voor het systeem uit oefening 3.4 ($\frac{d^2y}{dt^2} + 4\frac{dy}{dt} + 3y = 0$).

\textbf{Vraag:}
\begin{enumerate}[label=(\alph*)]
\item Schrijf dit differentiaalvergelijkingssysteem als een eerste-orde matrixvergelijking:
\[
\frac{d}{dt}\begin{pmatrix} y \\ \dot{y} \end{pmatrix} = \begin{pmatrix} 0 & 1 \\ -3 & -4 \end{pmatrix} \begin{pmatrix} y \\ \dot{y} \end{pmatrix}
\]
\item Bepaal de eigenwaarden van deze systeemmatrix.
\item Verifieer dat dit overeenkomt met de karakteristieke vergelijking uit oefening 3.4.
\item Bepaal de eigenvectoren.
\end{enumerate}

\subsection{Oefening 7.3: Diagonalisatie}

Gegeven de matrix:
\[
A = \begin{pmatrix} 5 & -2 \\ -2 & 2 \end{pmatrix}
\]

\textbf{Vraag:}
\begin{enumerate}[label=(\alph*)]
\item Bepaal alle eigenwaarden en bijbehorende eigenvectoren.
\item Controlleer dat de eigenvectoren orthogonaal zijn (omdat $A$ symmetrisch is).
\item Vorm de matrix $P$ met eigenvectoren als kolommen en bepaal $P^{-1}AP = D$ waarbij $D$ diagonaal is.
\end{enumerate}

\subsection{Oefening 7.4: Gerschgorin-Cirkelstelling}

Gegeven de matrix:
\[
A = \begin{pmatrix} 4 & 0.5 & 0.2 \\ 0.3 & -2 & 0.1 \\ 0.2 & 0.4 & 3 \end{pmatrix}
\]

\textbf{Vraag:}
\begin{enumerate}[label=(\alph*)]
\item Bepaal de Gerschgorin-cirkels voor deze matrix.
\item Geef grenzen voor de eigenwaarden op basis van de stelling.
\item Bepaal de eigenwaarden numeriek en controlleer of ze binnen de cirkels vallen.
\end{enumerate}

\subsection{Oefening 7.5: Eigenvectoren en Orthogonaliteit}

Gegeven de matrix:
\[
B = \begin{pmatrix} 3 & 1 \\ 1 & 3 \end{pmatrix}
\]

\textbf{Vraag:}
\begin{enumerate}[label=(\alph*)]
\item Bepaal de eigenwaarden.
\item Bepaal de eigenvectoren.
\item Toon aan dat de eigenvectoren orthogonaal zijn.
\item Normaliseer de eigenvectoren tot eenheid.
\end{enumerate}

\newpage

\section{Oplossingen}

\subsection{Oplossingen Hoofdstuk 1}

\subsubsection*{Oplossing 1.1}

\textbf{Gegeven:} Systeem met operator $\mathcal{T}\{x(t)\} = 3x(t) + 2$.

\textbf{Vraag:} Onderzoek of dit systeem lineair is.

\textbf{Oplossing:}

Het systeem is \textbf{niet lineair}.

\textbf{Bewijs:}
\begin{itemize}
\item Test eigenschap 1 (schaling): $\mathcal{T}\{ax(t)\} = 3ax(t) + 2 \neq a\mathcal{T}\{x(t)\} = a(3x(t) + 2) = 3ax(t) + 2a$
\item Voor $a \neq 1$ geldt: $2 \neq 2a$, dus schaling klopt niet.
\item Omdat de schalingsvoorwaarde niet voldaan is, is het systeem niet lineair.
\end{itemize}

\subsubsection*{Oplossing 1.2}

\textbf{Gegeven:} RC-circuit met $R = 1000$ $\Omega$, $C = 10$ $\mu$F, ingangsspanning $v_{\text{in}}(t) = 5u(t)$ V.

\textbf{Vraag:} 
\begin{enumerate}[label=(\alph*)]
\item Schrijf de differentiaalvergelijking op.
\item Bereken $\tau$.
\item Bepaal $v_{\text{uit}}(0.01)$.
\end{enumerate}

\textbf{Oplossing:}

\begin{enumerate}[label=(\alph*)]
\item De differentiaalvergelijking van een RC-circuit: $RC\frac{dv_{\text{uit}}}{dt} + v_{\text{uit}} = v_{\text{in}}$

\[
(1000)(10 \times 10^{-6})\frac{dv_{\text{uit}}}{dt} + v_{\text{uit}} = 5u(t)
\]

\[
0.01\frac{dv_{\text{uit}}}{dt} + v_{\text{uit}} = 5u(t)
\]

\item Tijdsconstante: $\tau = RC = (1000)(10 \times 10^{-6}) = 0.01$ s $= 10$ ms

\item Voor een staprespons met $v_{\text{uit}}(0) = 0$:

\[
v_{\text{uit}}(t) = 5(1 - e^{-t/\tau})u(t) = 5(1 - e^{-100t})u(t)
\]

Op $t = 10$ ms $= 0.01$ s:

\[
v_{\text{uit}}(0.01) = 5(1 - e^{-1}) = 5(1 - 0.368) = 5(0.632) = 3.16 \text{ V}
\]
\end{enumerate}

\subsubsection*{Oplossing 1.3}

\textbf{Gegeven:} Radio-isotoop met halveringstijd $t_{1/2} = 6$ uur. Initieelle hoeveelheid: 20 mg om 08:00.

\textbf{Vraag:} Hoeveel mg blijft over om 14:00?

\textbf{Oplossing:}

Model radioactief verval: $N(t) = N_0 e^{-kt}$

Halveringstijd $t_{1/2} = 6$ uur: $\frac{1}{2} = e^{-6k} \Rightarrow k = \frac{\ln(2)}{6} = 0.1155$ h$^{-1}$

Van 08:00 tot 14:00 is $\Delta t = 6$ uur:

\[
N(6) = 20 e^{-0.1155 \times 6} = 20 e^{-0.693} = 20 \times 0.5 = 10 \text{ mg}
\]

\textbf{Antwoord:} 10 mg blijft over.

\subsubsection*{Oplossing 1.4}

\textbf{Gegeven:} Twee systemen:
- Systeem A: $\mathcal{T}\{x(t)\} = 2x(t)$
- Systeem B: $\mathcal{T}\{x(t)\} = x(t) + 1$

\textbf{Vraag:} Test op homogeniteit en additiviteit; bepaal lineariteit.

\textbf{Oplossing:}

\textbf{Systeem A: } $\mathcal{T}\{x(t)\} = 2x(t)$

Homogeniteit: $\mathcal{T}\{ax(t)\} = 2ax(t) = a(2x(t)) = a\mathcal{T}\{x(t)\}$ \checkmark

Additiviteit: $\mathcal{T}\{x_1(t) + x_2(t)\} = 2(x_1(t) + x_2(t)) = 2x_1(t) + 2x_2(t) = \mathcal{T}\{x_1(t)\} + \mathcal{T}\{x_2(t)\}$ \checkmark

\textbf{Systeem A is LINEAIR.}

\textbf{Systeem B: } $\mathcal{T}\{x(t)\} = x(t) + 1$

Homogeniteit: $\mathcal{T}\{ax(t)\} = ax(t) + 1 \neq a(x(t) + 1) = ax(t) + a = a\mathcal{T}\{x(t)\}$ (voor $a \neq 1$) \text{\sffamily X}

\textbf{Systeem B is NIET LINEAIR.}

\subsubsection*{Oplossing 1.5}

\textbf{Gegeven:} Vier systemen. 
\textbf{Vraag:} Welke zijn LTI?

\textbf{Oplossing:}

\begin{enumerate}[label=(\alph*)]
\item $y(t) = x(t-2)$: LINEAIR en TIJDINVARIANT \checkmark (zuivere vertraging)

\item $y(t) = tx(t)$: LINEAIR maar NIET TIJDINVARIANT \text{\sffamily X} (co\"effici\"ent varieert in tijd)

\item $y(t) = |x(t)|$: NIET LINEAIR \text{\sffamily X} (niet additief/homogeen)

\item $y(t) = \int_0^t x(\tau) d\tau$: LINEAIR en TIJDINVARIANT \checkmark (integrator)

\end{enumerate}

\textbf{Antwoord:} (a) en (d) zijn LTI.

\subsection{Oplossingen Hoofdstuk 2}

\subsubsection*{Oplossing 2.1}

\textbf{Gegeven:} $x_1(t) = e^{0.2t}$ en $x_2(t) = e^{-0.5t}$.

\textbf{Vraag:} Bepaal welk signaal exponenti\"ele groei/verval vertoont; bereken waarden op $t = 5$ s.

\textbf{Oplossing:}

\begin{enumerate}[label=(\alph*)]
\item $x_1(t) = e^{0.2t}$: exponenti\"ele \textbf{groei} (positieve exponent)
$x_2(t) = e^{-0.5t}$: exponentieel \textbf{verval} (negatieve exponent)

\item Op $t = 5$ s:
\begin{align*}
x_1(5) &= e^{0.2 \times 5} = e^{1} \approx 2.718\\
x_2(5) &= e^{-0.5 \times 5} = e^{-2.5} \approx 0.082
\end{align*}
\end{enumerate}

\subsubsection*{Oplossing 2.2}

\textbf{Gegeven:} $x(t) = 3\sin(4\pi t + \frac{\pi}{6})$.

\textbf{Vraag:} Bepaal amplitude, hoekfrequentie, frequentie, fasehoek; schrijf als cosinus.

\textbf{Oplossing:}

\begin{enumerate}[label=(\alph*)]
\item Van $x(t) = 3\sin(4\pi t + \frac{\pi}{6})$:
\begin{itemize}
\item Amplitude: $A = 3$
\item Hoekfrequentie: $\omega = 4\pi$ rad/s
\item Frequentie: $f = \frac{\omega}{2\pi} = \frac{4\pi}{2\pi} = 2$ Hz
\item Fasehoek: $\phi = \frac{\pi}{6}$ rad $= 30\textdegree$
\end{itemize}

\item Als cosinusfunctie: $\sin(\theta) = \cos(\theta - \frac{\pi}{2})$

\[
x(t) = 3\cos\left(4\pi t + \frac{\pi}{6} - \frac{\pi}{2}\right) = 3\cos\left(4\pi t - \frac{\pi}{3}\right)
\]
\end{enumerate}

\subsubsection*{Oplossing 2.3}

\textbf{Gegeven:} $z(t) = e^{j2\pi t}$.

\textbf{Vraag:} Schrijf als sinus/cosinus; bepaal waarde op $t = 0.25$ s.

\textbf{Oplossing:}

\begin{enumerate}[label=(\alph*)]
\item Formule van Euler: $e^{j\theta} = \cos(\theta) + j\sin(\theta)$

\[
z(t) = e^{j2\pi t} = \cos(2\pi t) + j\sin(2\pi t)
\]

\item Op $t = 0.25$ s:

\[
z(0.25) = \cos(2\pi \times 0.25) + j\sin(2\pi \times 0.25) = \cos(\frac{\pi}{2}) + j\sin(\frac{\pi}{2}) = 0 + j = j
\]
\end{enumerate}

\subsubsection*{Oplossing 2.4}

\textbf{Gegeven:} $f(t) = u(t) - u(t-1)$ en $g(t) = u(t) - u(t-1)$.

\textbf{Vraag:} Bepaal $(f * g)(t)$ en schets.

\textbf{Oplossing:}

Voor twee identieke pulsen van breedte 1:

\[
(f * g)(t) = \begin{cases}
0, & t < 0\\
t, & 0 \le t < 1\\
2-t, & 1 \le t < 2\\
0, & t \ge 2
\end{cases}
\]

Dit is een \textbf{driehoeksfunctie} met maximum 1 op $t=1$ en breedte 2.

\subsubsection*{Oplossing 2.5}

\textbf{Gegeven:} $x(t) = e^{-t}u(t)$.

\textbf{Vraag:} Bepaal $y_1(t) = x(t-2)$, $y_2(t) = x(2t)$, $y_3(t) = 2x(t)$.

\textbf{Oplossing:}

\begin{enumerate}[label=(\alph*)]
\item Tijdsverschuiving: $y_1(t) = e^{-(t-2)}u(t-2) = e^{2}e^{-t}u(t-2)$ (start op $t=2$)

\item Tijdscompressie: $y_2(t) = e^{-2t}u(2t) = e^{-2t}u(t)$ (twee keer sneller)

\item Amplitude schaling: $y_3(t) = 2e^{-t}u(t)$ (twee keer hoger)

\item Schetsing: $y_1$ begint op $t=2$, $y_2$ vervalt sneller, $y_3$ heeft dubbele amplitude.
\end{enumerate}

\subsubsection*{Oplossing 2.6}

\textbf{Gegeven:} Drie signalen waarvan de energie bepaald moet worden.

\textbf{Vraag:} Bereken energies.

\textbf{Oplossing:}

\begin{enumerate}[label=(\alph*)]
\item $E_1 = \int_0^{\infty} |e^{-t}|^2 dt = \int_0^{\infty} e^{-2t} dt = \left[-\frac{1}{2}e^{-2t}\right]_0^{\infty} = \frac{1}{2}$

\item $E_2 = \int_0^{\pi} |2\sin(t)|^2 dt = 4\int_0^{\pi} \sin^2(t) dt = 4 \int_0^{\pi} \frac{1-\cos(2t)}{2} dt = 2[\pi - 0] = 2\pi$

\item $E_3 = \int_{-0.5}^{0.5} 1^2 dt = 1$
\end{enumerate}

\subsection{Oplossingen Hoofdstuk 3}

\subsubsection*{Oplossing 3.1}

\textbf{Gegeven:} Vier functies.

\textbf{Vraag:} Bepaal Laplacetransformaties.

\textbf{Oplossing:}

\begin{enumerate}[label=(\alph*)]
\item $\mathcal{L}\{5u(t)\} = \frac{5}{s}$, ROC: $\sigma > 0$

\item $\mathcal{L}\{e^{-3t}u(t)\} = \frac{1}{s+3}$, ROC: $\sigma > -3$

\item $\mathcal{L}\{t \cdot u(t)\} = \frac{1}{s^2}$, ROC: $\sigma > 0$

\item $\mathcal{L}\{\cos(5t) \cdot u(t)\} = \frac{s}{s^2 + 25}$, ROC: $\sigma > 0$
\end{enumerate}

\subsubsection*{Oplossing 3.2}

\textbf{Gegeven:} $F(s) = \frac{3}{s+2} + \frac{5}{s^2 + 4}$.

\textbf{Vraag:} Vind inverse Laplacetransformatie.

\textbf{Oplossing:}

\[
F(s) = \frac{3}{s+2} + \frac{5}{s^2 + 2^2}
\]

Inverse Laplacetransformatie:

\[
f(t) = 3e^{-2t}u(t) + \frac{5}{2}\sin(2t)u(t)
\]

\subsubsection*{Oplossing 3.3}

\textbf{Gegeven:} $\frac{dy}{dt} + 4y = 8u(t)$, $y(0) = 2$.

\textbf{Vraag:} Los op met Laplacetransformatie.

\textbf{Oplossing:}

Laplacetransformatie van beide zijden:

\[
sY(s) - y(0) + 4Y(s) = \frac{8}{s}
\]

\[
sY(s) - 2 + 4Y(s) = \frac{8}{s}
\]

\[
Y(s)(s+4) = \frac{8}{s} + 2 = \frac{8 + 2s}{s}
\]

\[
Y(s) = \frac{8 + 2s}{s(s+4)} = \frac{2}{s} + \frac{2}{s+4}
\]

Inverse Laplacetransformatie:

\[
y(t) = 2u(t) + 2e^{-4t}u(t) = (2 + 2e^{-4t})u(t)
\]

\subsubsection*{Oplossing 3.4}

\textbf{Gegeven:} $\frac{d^2y}{dt^2} + 4\frac{dy}{dt} + 3y = 0$, $y(0) = 1$, $y'(0) = 0$.

\textbf{Vraag:} 
\begin{enumerate}[label=(\alph*)]
\item Karakteristieke vergelijking.
\item Wortels.
\item Los op.
\end{enumerate}

\textbf{Oplossing:}

\begin{enumerate}[label=(\alph*)]
\item Karakteristieke vergelijking: $\lambda^2 + 4\lambda + 3 = 0$

\item Wortels:

\[
\lambda = \frac{-4 \pm \sqrt{16-12}}{2} = \frac{-4 \pm 2}{2}
\]

$\lambda_1 = -1$, $\lambda_2 = -3$

\item Algemene oplossing:

\[
y(t) = c_1 e^{-t} + c_2 e^{-3t}
\]

Met beginvoorwaarden:
- $y(0) = 1$: $c_1 + c_2 = 1$
- $y'(0) = 0$: $-c_1 - 3c_2 = 0 \Rightarrow c_1 = -3c_2$

Oplossen: $c_2 = -\frac{1}{2}$, $c_1 = \frac{3}{2}$

\[
y(t) = \frac{3}{2}e^{-t} - \frac{1}{2}e^{-3t}
\]
\end{enumerate}

\subsubsection*{Oplossing 3.5}

\textbf{Gegeven:} $F(s) = \frac{2}{s^2 + 4}$.

\textbf{Vraag:} Bepaal $f(t)$ en $g(t) = \mathcal{L}^{-1}\{e^{-2s}F(s)\}$.

\textbf{Oplossing:}

\begin{enumerate}[label=(\alph*)]
\item Inverse Laplacetransformatie:

\[
f(t) = \mathcal{L}^{-1}\left\{\frac{2}{s^2 + 4}\right\} = \sin(2t)u(t)
\]

\item Gebruikmakend van de tijdsverschuivingsstelling $\mathcal{L}^{-1}\{e^{-as}F(s)\} = f(t-a)u(t-a)$:

\[
g(t) = f(t-2)u(t-2) = \sin(2(t-2))u(t-2) = \sin(2t-4)u(t-2)
\]
\end{enumerate}

\subsubsection*{Oplossing 3.6}

\textbf{Gegeven:} $F(s) = \frac{10}{(s+1)(s+2)(s+3)}$.

\textbf{Vraag:} Bepaal inverse via partieelbreuken.

\textbf{Oplossing:}

Partieelbreukontwikkeling:

\[
\frac{10}{(s+1)(s+2)(s+3)} = \frac{A}{s+1} + \frac{B}{s+2} + \frac{C}{s+3}
\]

Vermenigvuldigen met $(s+1)(s+2)(s+3)$:

\[
10 = A(s+2)(s+3) + B(s+1)(s+3) + C(s+1)(s+2)
\]

Voor $s = -1$: $10 = A(1)(2) = 2A \Rightarrow A = 5$

Voor $s = -2$: $10 = B(-1)(1) = -B \Rightarrow B = -10$

Voor $s = -3$: $10 = C(-2)(-1) = 2C \Rightarrow C = 5$

Inverse Laplacetransformatie:

\[
f(t) = 5e^{-t}u(t) - 10e^{-2t}u(t) + 5e^{-3t}u(t) = (5e^{-t} - 10e^{-2t} + 5e^{-3t})u(t)
\]

\subsection{Oplossingen Hoofdstuk 4}

\subsubsection*{Oplossing 4.1}

\textbf{Gegeven:} Rechthoekpuls $f(t) = A$ voor $-T/2 < t < T/2$, 0 elders.

\textbf{Vraag:} Bepaal Fouriertransformatie; schrijf als sinc; bepaal nulpunten.

\textbf{Oplossing:}

\begin{enumerate}[label=(\alph*)]
\item Fouriertransformatie:

\[
F(j\omega) = \int_{-T/2}^{T/2} A e^{-j\omega t} dt = A \left[ \frac{e^{-j\omega t}}{-j\omega} \right]_{-T/2}^{T/2}
\]

\[
= A \frac{e^{j\omega T/2} - e^{-j\omega T/2}}{j\omega} = A \frac{2\sin(\omega T/2)}{\omega} = AT \cdot \frac{\sin(\omega T/2)}{\omega T/2}
\]

\item Sinc-vorm:

\[
F(j\omega) = AT \cdot \text{sinc}\left(\frac{\omega T}{2}\right)
\]

\item Eerste nulpunten: $\frac{\omega T}{2} = \pm\pi, \pm 2\pi, \ldots$

Dit geeft: $\omega = \pm\frac{2\pi}{T}, \pm\frac{4\pi}{T}, \ldots$
\end{enumerate}

\subsubsection*{Oplossing 4.2}

\textbf{Gegeven:} Fouriertransformatie verschuivingsstelling.

\textbf{Vraag:} Bewijs; pas toe; bespreek effect op spektra.

\textbf{Oplossing:}

\begin{enumerate}[label=(\alph*)]
\item Verschuivingsstelling:
\[
\mathcal{F}\{f(t-t_0)\} = e^{-j\omega t_0} F(j\omega)
\]

\textbf{Bewijs:}
\[
\mathcal{F}\{f(t-t_0)\} = \int_{-\infty}^{\infty} f(t-t_0) e^{-j\omega t} dt
\]

Substitutie $\tau = t - t_0$:
\[
= \int_{-\infty}^{\infty} f(\tau) e^{-j\omega(\tau+t_0)} d\tau = e^{-j\omega t_0} F(j\omega)
\]

\item Voor puls met $t_0 = 1$, $A = 2$, $T = 2$:

\[
F(j\omega) = 2 \cdot 2 \cdot \text{sinc}(\omega) \cdot e^{-j\omega} = 4\text{sinc}(\omega) e^{-j\omega}
\]

\item Amplitude-spectrum: onveranderd (blijft $4|\text{sinc}(\omega)|$)

Fase-spectrum: lineair met $-\omega$ (effect van verschuiving)
\end{enumerate}

\subsubsection*{Oplossing 4.3}

\textbf{Gegeven:} $x(t) = \cos(10\pi t) \cdot \text{rect}(t)$ met $\text{rect}(t) = u(t+1) - u(t-1)$.

\textbf{Vraag:} Pas modulatiestelling toe; schets spektra.

\textbf{Oplossing:}

\begin{enumerate}[label=(\alph*)]
\item Modulatiestelling:
\[
\mathcal{F}\{\cos(\omega_0 t) f(t)\} = \frac{1}{2}[F(j(\omega-\omega_0)) + F(j(\omega+\omega_0))]
\]

Voor $\text{rect}(t)$: $F_{\text{rect}}(j\omega) = 2\text{sinc}(\omega)$

Voor $x(t)$ met $\omega_0 = 10\pi$:
\[
X(j\omega) = \frac{1}{2}[2\text{sinc}(\omega-10\pi) + 2\text{sinc}(\omega+10\pi)] = \text{sinc}(\omega-10\pi) + \text{sinc}(\omega+10\pi)
\]

\item Het spectrum bestaat uit twee verschoven sinc-functies gecentreerd op $\omega = \pm 10\pi$.
\end{enumerate}

\subsubsection*{Oplossing 4.4}

\textbf{Gegeven:} $f(t) = e^{-t}u(t)$.

\textbf{Vraag:} Bepaal energie in tijdsdomein; controleer in frequentiedomein; verifieer Parsevals stelling.

\textbf{Oplossing:}

\begin{enumerate}[label=(\alph*)]
\item Energie in tijdsdomein:

\[
E = \int_0^{\infty} e^{-2t} dt = \left[-\frac{1}{2}e^{-2t}\right]_0^{\infty} = \frac{1}{2}
\]

\item Fouriertransformatie: $F(j\omega) = \frac{1}{1+j\omega}$

\[
|F(j\omega)|^2 = \frac{1}{1+\omega^2}
\]

\item Energie in frequentiedomein (Parsevals stelling):

\[
E = \frac{1}{2\pi} \int_{-\infty}^{\infty} |F(j\omega)|^2 d\omega = \frac{1}{2\pi} \int_{-\infty}^{\infty} \frac{1}{1+\omega^2} d\omega
\]

\[
= \frac{1}{2\pi} [\arctan(\omega)]_{-\infty}^{\infty} = \frac{1}{2\pi} \cdot \pi = \frac{1}{2} \quad \checkmark
\]
\end{enumerate}

\subsubsection*{Oplossing 4.5}

\textbf{Gegeven:} $f(t) = e^{-a|t|}$ met $a > 0$.

\textbf{Vraag:} Bepaal Fouriertransformatie; schets spektra; bepaal bandbreedte.

\textbf{Oplossing:}

\begin{enumerate}[label=(\alph*)]
\item Fouriertransformatie:

\[
F(j\omega) = \int_{-\infty}^{\infty} e^{-a|t|} e^{-j\omega t} dt = \int_{-\infty}^{0} e^{at} e^{-j\omega t} dt + \int_{0}^{\infty} e^{-at} e^{-j\omega t} dt
\]

\[
= \frac{1}{a-j\omega} + \frac{1}{a+j\omega} = \frac{2a}{a^2 + \omega^2}
\]

\item Amplitude-spectrum: $|F(j\omega)| = \frac{2a}{a^2 + \omega^2}$

Fase-spectrum: $\angle F(j\omega) = 0$ (zuiver reeel en positief)

\item Eerste nulpunt: geen nulpunten, maar bandbreedte op halve amplitude: $|F(j\omega)| = |F(0)|/2 = a$

Dit geeft: $\frac{2a}{a^2 + \omega^2} = a \Rightarrow \omega = a$ (3dB bandbreedte = $2a$)
\end{enumerate}

\subsubsection*{Oplossing 4.6}

\textbf{Gegeven:} Drie signalen.

\textbf{Vraag:} Bepaal Fouriertransformaties.

\textbf{Oplossing:}

\begin{enumerate}[label=(\alph*)]
\item $f(t) = \delta(t)$: $F(j\omega) = 1$ (constant spectrum)

\item $f(t) = \delta(t-t_0)$: $F(j\omega) = e^{-j\omega t_0}$ (lineaire fase)

\item $f(t) = \cos(\omega_0 t) = \frac{1}{2}(e^{j\omega_0 t} + e^{-j\omega_0 t})$:

\[
F(j\omega) = \pi[\delta(\omega-\omega_0) + \delta(\omega+\omega_0)]
\]

(twee impulsen op $\pm\omega_0$)
\end{enumerate}

\subsection{Oplossingen Hoofdstuk 5}

\subsubsection*{Oplossing 5.1}

\textbf{Gegeven:} Blokgolf met periode $T = 2$: $f(t) = 1$ voor $0 < t < 1$, $f(t) = -1$ voor $1 < t < 2$.

\textbf{Vraag:} Bepaal symmetrie; bereken Fourierco\"effici\"enten; schrijf reeks tot 3e harmonische.

\textbf{Oplossing:}

\begin{enumerate}[label=(\alph*)]
\item De functie is \textbf{oneven}: $f(-t) = -f(t)$ (na periodieke uitbreiding)

\item Fourierco\"effici\"enten ($\omega_0 = \pi$ rad/s):

\[
a_0 = \frac{1}{2}\int_0^2 f(t)dt = \frac{1}{2}[1 - 1] = 0
\]

Voor oneven functie: $a_n = 0$

\[
b_n = \int_0^1 \sin(n\pi t)dt - \int_1^2 \sin(n\pi t)dt = \frac{4}{n\pi} \text{ voor oneven } n
\]

\item Fourierreeks tot 3e harmonische:

\[
f(t) \approx \frac{4}{\pi}\sin(\pi t) + \frac{4}{3\pi}\sin(3\pi t)
\]
\end{enumerate}

\subsubsection*{Oplossing 5.2}

\textbf{Gegeven:} Zaagtandgolf: $f(t) = 2t$ voor $0 < t < 1$ met periode $T = 1$.

\textbf{Vraag:} Bereken $a_0$, $b_1$, $b_2$; schrijf Fouriersom met 2 termen.

\textbf{Oplossing:}

\begin{enumerate}[label=(\alph*)]
\item $a_0 = \int_0^1 2t \, dt = [t^2]_0^1 = 1$

\item Voor $b_n$:

\[
b_n = 2\int_0^1 2t\sin(2\pi n t)dt = -\frac{2}{n\pi}
\]

Dus: $b_1 = -\frac{2}{\pi}$, $b_2 = -\frac{1}{\pi}$

\item Benaderende Fouriersom:

\[
f(t) \approx 1 - \frac{2}{\pi}\sin(2\pi t) - \frac{1}{\pi}\sin(4\pi t)
\]
\end{enumerate}

\subsubsection*{Oplossing 5.3}

\textbf{Gegeven:} Driehoekgolf: $f(t) = t$ voor $0 \le t < 1$, $f(t) = 2-t$ voor $1 \le t < 2$, periode $T = 2$.

\textbf{Vraag:} Bepaal symmetrie; bereken co\"effici\"enten; schrijf eerste drie niet-nul termen.

\textbf{Oplossing:}

\begin{enumerate}[label=(\alph*)]
\item Symmetrie: Dit signaal is \textbf{even}: $f(2-t) = f(t)$

\item Fourierco\"effici\"enten ($\omega_0 = \pi$):

\[
a_0 = \frac{1}{2}\int_0^2 f(t)dt = \frac{1}{2} \cdot 1 = 0.5
\]

Voor even functie: $b_n = 0$

\[
a_n = \frac{4}{n^2\pi^2} \sin(n\pi/2) \text{ (na integratie)}
\]

\item Eerste drie niet-nul termen:

\[
f(t) \approx 0.5 + \frac{4}{\pi^2}\cos(\pi t) - \frac{4}{9\pi^2}\cos(3\pi t)
\]
\end{enumerate}

\subsubsection*{Oplossing 5.4}

\textbf{Gegeven:} Blokgolf uit oefening 5.1.

\textbf{Vraag:} Bepaal gemiddelde macht; controlleer met Parsevals stelling.

\textbf{Oplossing:}

\begin{enumerate}[label=(\alph*)]
\item Gemiddelde macht via integratie:

\[
P = \frac{1}{2}\int_0^2 f^2(t)dt = \frac{1}{2}[1 + 1] = 1
\]

\item Parsevals stelling voor Fourierreeksen:

\[
P = a_0^2 + \frac{1}{2}\sum_{n=1}^{\infty} b_n^2 = 0 + \frac{1}{2}\sum_{n=1,3,5,...} \frac{16}{n^2\pi^2}
\]

\[
= \frac{8}{\pi^2}[1 + \frac{1}{9} + \frac{1}{25} + ...] = \frac{8}{\pi^2} \cdot \frac{\pi^2}{8} = 1 \quad \checkmark
\]
\end{enumerate}

\subsection{Oplossingen Hoofdstuk 6}

\subsubsection*{Oplossing 6.1}

\textbf{Gegeven:} Eerste-orde systeem met impulsrespons $h(t) = 2e^{-5t}u(t)$.

\textbf{Vraag:} Bepaal systeemrespons op stapingang via convolutie; verifieer met Laplace.

\textbf{Oplossing:}

\begin{enumerate}[label=(\alph*)]
\item Convolutie met stapingang:

\[
y(t) = h(t) * u(t) = \int_0^t 2e^{-5\tau}d\tau = 2\left[-\frac{1}{5}e^{-5\tau}\right]_0^t = \frac{2}{5}(1 - e^{-5t})u(t)
\]

\item Via Laplacetransformatie:

\[
H(s) = \frac{2}{s+5}, \quad F(s) = \frac{1}{s}, \quad Y(s) = \frac{2}{s(s+5)}
\]

Partieelbreuken: $\frac{2}{s(s+5)} = \frac{2/5}{s} - \frac{2/5}{s+5}$

\[
y(t) = \frac{2}{5}(1 - e^{-5t})u(t) \quad \checkmark
\]
\end{enumerate}

\subsubsection*{Oplossing 6.2}

\textbf{Gegeven:} Massa-veer-dempersysteem: $m = 2$ kg, $k = 8$ N/m, $c = 4$ Ns/m.

\textbf{Vraag:} Schrijf DV; bepaal $\omega_0$; bepaal type demping.

\textbf{Oplossing:}

\begin{enumerate}[label=(\alph*)]
\item Differentiaalvergelijking:

\[
2\frac{d^2y}{dt^2} + 4\frac{dy}{dt} + 8y = f(t)
\]

\item Natuurlijke eigenfrequentie:

\[
\omega_0 = \sqrt{\frac{k}{m}} = \sqrt{\frac{8}{2}} = 2 \text{ rad/s}
\]

\item Karakteristieke vergelijking: $2\lambda^2 + 4\lambda + 8 = 0 \Rightarrow \lambda^2 + 2\lambda + 4 = 0$

\[
\lambda = -1 \pm j\sqrt{3}
\]

Complexe wortels $\Rightarrow$ Het systeem is \textbf{ondergedempt}.
\end{enumerate}

\subsubsection*{Oplossing 6.3}

\textbf{Gegeven:} LTC-systeem met $H(s) = \frac{10}{s+5}$.

\textbf{Vraag:} Bepaal frequentierespons; bepaal amplitude/faserespons; bepaal 3dB bandbreedte.

\textbf{Oplossing:}

\begin{enumerate}[label=(\alph*)]
\item Frequentierespons:

\[
H(j\omega) = \frac{10}{j\omega+5} = \frac{10(5-j\omega)}{25+\omega^2}
\]

\item Amplitude- en faserespons:

\[
|H(j\omega)| = \frac{10}{\sqrt{25+\omega^2}}, \quad \angle H(j\omega) = -\arctan(\omega/5)
\]

\item 3dB bandbreedte waar $|H(j\omega)| = |H(0)|/\sqrt{2} = 2/\sqrt{2} = \sqrt{2}$:

\[
\frac{10}{\sqrt{25+\omega^2}} = \sqrt{2} \Rightarrow 25 + \omega^2 = 50 \Rightarrow \omega_{3dB} = 5 \text{ rad/s}
\]
\end{enumerate}

\subsubsection*{Oplossing 6.4}

\textbf{Gegeven:} Twee systemen in cascade: $H_1(s) = \frac{5}{s+2}$, $H_2(s) = \frac{3}{s+3}$.

\textbf{Vraag:} Bepaal totale overdracht; bepaal impulsrespons.

\textbf{Oplossing:}

\begin{enumerate}[label=(\alph*)]
\item Totale overdracht:

\[
H(s) = H_1(s) \cdot H_2(s) = \frac{5}{s+2} \cdot \frac{3}{s+3} = \frac{15}{(s+2)(s+3)}
\]

\item Impulsrespons via partieelbreuken:

\[
\frac{15}{(s+2)(s+3)} = \frac{A}{s+2} + \frac{B}{s+3}
\]

Voor $s = -2$: $15 = A(1) \Rightarrow A = 15$

Voor $s = -3$: $15 = B(-1) \Rightarrow B = -15$

\[
h(t) = 15e^{-2t}u(t) - 15e^{-3t}u(t) = 15(e^{-2t} - e^{-3t})u(t)
\]
\end{enumerate}

\subsection{Oplossingen Hoofdstuk 7}

\subsubsection*{Oplossing 7.1}

\textbf{Gegeven:} $A = \begin{pmatrix} 4 & 1 \\ 2 & 3 \end{pmatrix}$.

\textbf{Vraag:} Bepaal karakteristieke veelterm; vind eigenwaarden; bereken eigenvectoren.

\textbf{Oplossing:}

\begin{enumerate}[label=(\alph*)]
\item Karakteristieke veelterm:

\[
f(\lambda) = \begin{vmatrix} 4-\lambda & 1 \\ 2 & 3-\lambda \end{vmatrix} = (4-\lambda)(3-\lambda) - 2 = \lambda^2 - 7\lambda + 10
\]

\item Eigenwaarden: $\lambda^2 - 7\lambda + 10 = (\lambda - 5)(\lambda - 2) = 0$

$\lambda_1 = 5$, $\lambda_2 = 2$

\item Eigenvectoren:

Voor $\lambda_1 = 5$:
\[
\begin{pmatrix} -1 & 1 \\ 2 & -2 \end{pmatrix} \begin{pmatrix} v_1 \\ v_2 \end{pmatrix} = 0 \Rightarrow v_1 = v_2 \Rightarrow \mathbf{v}_1 = \begin{pmatrix} 1 \\ 1 \end{pmatrix}
\]

Voor $\lambda_2 = 2$:
\[
\begin{pmatrix} 2 & 1 \\ 2 & 1 \end{pmatrix} \begin{pmatrix} v_1 \\ v_2 \end{pmatrix} = 0 \Rightarrow v_2 = -2v_1 \Rightarrow \mathbf{v}_2 = \begin{pmatrix} 1 \\ -2 \end{pmatrix}
\]
\end{enumerate}

\subsubsection*{Oplossing 7.2}

\textbf{Gegeven:} Tweede-orde systeem uit oefening 3.4.

\textbf{Vraag:} Schrijf als matrixvergelijking; bepaal eigenwaarden; verifieer; bepaal eigenvectoren.

\textbf{Oplossing:}

\begin{enumerate}[label=(\alph*)]
\item Matrixvergelijking:
\[
\frac{d}{dt}\begin{pmatrix} y \\ \dot{y} \end{pmatrix} = \begin{pmatrix} 0 & 1 \\ -3 & -4 \end{pmatrix} \begin{pmatrix} y \\ \dot{y} \end{pmatrix}
\]

\item Karakteristieke vergelijking:
\[
f(\lambda) = \begin{vmatrix} -\lambda & 1 \\ -3 & -4-\lambda \end{vmatrix} = \lambda^2 + 4\lambda + 3 = 0
\]

\item Eigenwaarden: $\lambda_1 = -1$, $\lambda_2 = -3$ (hetzelfde als uit oefening 3.4) \checkmark

\item Eigenvectoren:

Voor $\lambda_1 = -1$:
\[
\begin{pmatrix} 1 & 1 \\ -3 & -3 \end{pmatrix} \begin{pmatrix} v_1 \\ v_2 \end{pmatrix} = 0 \Rightarrow \mathbf{v}_1 = \begin{pmatrix} 1 \\ -1 \end{pmatrix}
\]

Voor $\lambda_2 = -3$:
\[
\begin{pmatrix} 3 & 1 \\ -3 & -1 \end{pmatrix} \begin{pmatrix} v_1 \\ v_2 \end{pmatrix} = 0 \Rightarrow \mathbf{v}_2 = \begin{pmatrix} 1 \\ -3 \end{pmatrix}
\]
\end{enumerate}

\subsubsection*{Oplossing 7.3}

\textbf{Gegeven:} $A = \begin{pmatrix} 5 & -2 \\ -2 & 2 \end{pmatrix}$.

\textbf{Vraag:} Bepaal eigenwaarden/eigenvectoren; controlleer orthogonaliteit; bepaal $P^{-1}AP = D$.

\textbf{Oplossing:}

\begin{enumerate}[label=(\alph*)]
\item Karakteristieke vergelijking:

\[
f(\lambda) = \lambda^2 - 7\lambda + 6 = (\lambda - 6)(\lambda - 1) = 0
\]

Eigenwaarden: $\lambda_1 = 6$, $\lambda_2 = 1$

Eigenvectoren:
- Voor $\lambda_1 = 6$: $\mathbf{v}_1 = \begin{pmatrix} 2 \\ -1 \end{pmatrix}$ (genormaliseerd: $\mathbf{u}_1 = \frac{1}{\sqrt{5}}\begin{pmatrix} 2 \\ -1 \end{pmatrix}$)
- Voor $\lambda_2 = 1$: $\mathbf{v}_2 = \begin{pmatrix} 1 \\ 2 \end{pmatrix}$ (genormaliseerd: $\mathbf{u}_2 = \frac{1}{\sqrt{5}}\begin{pmatrix} 1 \\ 2 \end{pmatrix}$)

\item Orthogonaliteit:
\[
\mathbf{u}_1 \cdot \mathbf{u}_2 = \frac{1}{5}(2 - 2) = 0 \quad \checkmark
\]

\item Diagonalisatie:

\[
P = \frac{1}{\sqrt{5}}\begin{pmatrix} 2 & 1 \\ -1 & 2 \end{pmatrix}, \quad P^{-1} = P^T = \frac{1}{\sqrt{5}}\begin{pmatrix} 2 & -1 \\ 1 & 2 \end{pmatrix}
\]

\[
D = \begin{pmatrix} 6 & 0 \\ 0 & 1 \end{pmatrix}
\]
\end{enumerate}

\subsubsection*{Oplossing 7.4}

\textbf{Gegeven:} $A = \begin{pmatrix} 4 & 0.5 & 0.2 \\ 0.3 & -2 & 0.1 \\ 0.2 & 0.4 & 3 \end{pmatrix}$.

\textbf{Vraag:} Bepaal Gerschgorin-cirkels; geef grenzen; controlleer eigenwaarden.

\textbf{Oplossing:}

\begin{enumerate}[label=(\alph*)]
\item Gerschgorin-cirkels:

Rij 1: Centrum 4, radius $0.5 + 0.2 = 0.7$, dus $\lambda \in [3.3, 4.7]$

Rij 2: Centrum -2, radius $0.3 + 0.1 = 0.4$, dus $\lambda \in [-2.4, -1.6]$

Rij 3: Centrum 3, radius $0.2 + 0.4 = 0.6$, dus $\lambda \in [2.4, 3.6]$

\item Alle eigenwaarden liggen in de unie van deze cirkels.

\item De eigenwaarden zijn ongeveer: $\lambda_1 \approx 4.3$, $\lambda_2 \approx -2.1$, $\lambda_3 \approx 2.8$ (numeriek bepaald)

Alle drie liggen inderdaad in hun respectieve cirkels. \checkmark
\end{enumerate}

\subsubsection*{Oplossing 7.5}

\textbf{Gegeven:} $B = \begin{pmatrix} 3 & 1 \\ 1 & 3 \end{pmatrix}$.

\textbf{Vraag:} Bepaal eigenwaarden; bepaal eigenvectoren; toon orthogonaliteit aan; normaliseer.

\textbf{Oplossing:}

\begin{enumerate}[label=(\alph*)]
\item Karakteristieke vergelijking:

\[
f(\lambda) = (3-\lambda)^2 - 1 = \lambda^2 - 6\lambda + 8 = (\lambda - 4)(\lambda - 2) = 0
\]

Eigenwaarden: $\lambda_1 = 4$, $\lambda_2 = 2$

\item Eigenvectoren:

Voor $\lambda_1 = 4$:
\[
\begin{pmatrix} -1 & 1 \\ 1 & -1 \end{pmatrix} \begin{pmatrix} v_1 \\ v_2 \end{pmatrix} = 0 \Rightarrow \mathbf{v}_1 = \begin{pmatrix} 1 \\ 1 \end{pmatrix}
\]

Voor $\lambda_2 = 2$:
\[
\begin{pmatrix} 1 & 1 \\ 1 & 1 \end{pmatrix} \begin{pmatrix} v_1 \\ v_2 \end{pmatrix} = 0 \Rightarrow \mathbf{v}_2 = \begin{pmatrix} 1 \\ -1 \end{pmatrix}
\]

\item Orthogonaliteit:
\[
\mathbf{v}_1 \cdot \mathbf{v}_2 = 1 - 1 = 0 \quad \checkmark
\]

\item Genormaliseerde eigenvectoren:

\[
\mathbf{u}_1 = \frac{1}{\sqrt{2}}\begin{pmatrix} 1 \\ 1 \end{pmatrix}, \quad \mathbf{u}_2 = \frac{1}{\sqrt{2}}\begin{pmatrix} 1 \\ -1 \end{pmatrix}
\]

Verificatie: $|\mathbf{u}_1| = |\mathbf{u}_2| = 1$, $\mathbf{u}_1 \cdot \mathbf{u}_2 = 0$ \checkmark
\end{enumerate}

\end{document}
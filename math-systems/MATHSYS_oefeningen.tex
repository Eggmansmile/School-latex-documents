\documentclass[a4paper,11pt]{article}
\usepackage[dutch]{babel}
\usepackage[utf8]{inputenc}
\usepackage{amsmath}
\usepackage{amssymb}
\usepackage{geometry}
\usepackage{enumitem}
\usepackage{graphicx}

\geometry{margin=2.5cm}
\title{Oefeningen Wiskunde voor Systemen}
\author{KU Leuven - ESAT}
\date{\today}

\begin{document}

\maketitle
\tableofcontents
\newpage

\section{Hoofdstuk 1: Signalen en Systemen - Eerste Kennismaking}

\subsection{Oefening 1.1: Lineaire Systemen}
Gegeven een systeem met operator $\mathcal{T}$ gedefinieerd als $\mathcal{T}\{x(t)\} = 3x(t) + 2$.

\textbf{Vraag:} Onderzoek of dit systeem lineair is.

\subsection{Oefening 1.2: RC-Circuit}
Een RC-circuit heeft $R = 1000$ $\Omega$ en $C = 10$ $\mu$F. De ingangsspanning is een stapfunctie $v_{\text{in}}(t) = 5u(t)$ V. 

\textbf{Vraag:}
\begin{enumerate}[label=(\alph*)]
\item Schrijf de differentiaalvergelijking op voor de uitgangsspanning $v_{\text{uit}}(t)$.
\item Bereken de tijdsconstante $\tau$ van het circuit.
\item Bepaal de uitgangsspanning na $10$ ms als $v_{\text{uit}}(0) = 0$ V.
\end{enumerate}

\subsection{Oefening 1.3: Radioactief Verval}
Een radio-isotoop heeft een halveringstijd van $6$ uur. Om 08:00 uur wordt 20 mg geproduceerd.

\textbf{Vraag:} Hoeveel milligram blijft over om 14:00 uur?

\newpage
\section{Hoofdstuk 2: Basissignalen en Bewerkingen}

\subsection{Oefening 2.1: Exponentiële Functies}
Gegeven de signalen $x_1(t) = e^{0.2t}$ en $x_2(t) = e^{-0.5t}$.

\textbf{Vraag:}
\begin{enumerate}[label=(\alph*)]
\item Bepaal welk signaal exponentiële groei en welk exponentieel verval vertoont.
\item Bereken de waarde van elk signaal op $t = 5$ s.
\end{enumerate}

\subsection{Oefening 2.2: Sinus en Cosinus}
Een sinusgolf is gegeven door $x(t) = 3\sin(4\pi t + \frac{\pi}{6})$.

\textbf{Vraag:}
\begin{enumerate}[label=(\alph*)]
\item Bepaal de amplitude, hoekfrequentie $\omega$, frequentie $f$, en fasehoek.
\item Schrijf dit signaal als een cosinusfunctie.
\end{enumerate}

\subsection{Oefening 2.3: Complexe Exponentiële Functie}
Gegeven $z(t) = e^{j2\pi t}$.

\textbf{Vraag:}
\begin{enumerate}[label=(\alph*)]
\item Schrijf dit signaal in termen van sinus en cosinus gebruikmakend van de formule van Euler.
\item Bepaal de waarde op $t = 0.25$ s.
\end{enumerate}

\subsection{Oefening 2.4: Convolutie}
Bereken de convolutie van twee pulssignalen:
\[
f(t) = u(t) - u(t-1), \quad g(t) = u(t) - u(t-1)
\]

\textbf{Vraag:} Bepaal $(f * g)(t)$ en schets het resultaat.

\newpage
\section{Hoofdstuk 3: Laplacetransformatie}

\subsection{Oefening 3.1: Eenvoudige Laplacetransformaties}
Bepaal de Laplacetransformatie van de volgende functies:

\textbf{Vraag:}
\begin{enumerate}[label=(\alph*)]
\item $f(t) = 5u(t)$
\item $f(t) = e^{-3t}u(t)$
\item $f(t) = t \cdot u(t)$
\item $f(t) = \cos(5t) \cdot u(t)$
\end{enumerate}

\subsection{Oefening 3.2: Inverse Laplacetransformatie}
Bepaal de inverse Laplacetransformatie van:
\[
F(s) = \frac{3}{s+2} + \frac{5}{s^2 + 4}
\]

\textbf{Vraag:} Vind $f(t)$.

\subsection{Oefening 3.3: Eerste-Orde Systeem}
Los de volgende differentiaalvergelijking op met Laplacetransformatie:
\[
\frac{dy}{dt} + 4y = 8u(t), \quad y(0) = 2
\]

\textbf{Vraag:} Bepaal $y(t)$.

\subsection{Oefening 3.4: Tweede-Orde Systeem}
Een massa-veer-dempersysteem wordt beschreven door:
\[
\frac{d^2y}{dt^2} + 4\frac{dy}{dt} + 3y = 0, \quad y(0) = 1, \quad y'(0) = 0
\]

\textbf{Vraag:}
\begin{enumerate}[label=(\alph*)]
\item Bepaal de karakteristieke vergelijking.
\item Vind de wortels van de karakteristieke vergelijking.
\item Los de differentiaalvergelijking op voor $y(t)$.
\end{enumerate}

\newpage
\section{Hoofdstuk 5: Fourierreeksen}

\subsection{Oefening 5.1: Blokgolf}
Een periodieke blokgolf met periode $T = 2$ s is gedefinieerd als:
\[
f(t) = \begin{cases}
1, & 0 < t < 1 \\
-1, & 1 < t < 2
\end{cases}
\]

\textbf{Vraag:}
\begin{enumerate}[label=(\alph*)]
\item Bepaal of de functie even, oneven, of geen van beide is.
\item Bereken de Fouriercoëfficiënten $a_0$, $a_n$, en $b_n$.
\item Schrijf de Fourierreeks tot de 3e harmonische.
\end{enumerate}

\subsection{Oefening 5.2: Zaagtandgolf}
Een zaagtandgolf met periode $T = 1$ s is gegeven door $f(t) = 2t$ voor $0 < t < 1$.

\textbf{Vraag:}
\begin{enumerate}[label=(\alph*)]
\item Bereken $a_0$.
\item Bepaal $b_1$ en $b_2$.
\item Schrijf de benaderende Fouriersom met 2 termen.
\end{enumerate}

\newpage
\section{Hoofdstuk 6: LTC-Systemen}

\subsection{Oefening 6.1: Impulsrespons}
Een eerste-orde systeem heeft impulsrespons $h(t) = 2e^{-5t}u(t)$.

\textbf{Vraag:}
\begin{enumerate}[label=(\alph*)]
\item Bereken de systeemrespons op een stapingang $f(t) = u(t)$ door convolutie.
\item Verifieer je antwoord met de Laplacetransformatie.
\end{enumerate}

\subsection{Oefening 6.2: Massa-Veer-Demper}
Een massa-veer-dempersysteem heeft $m = 2$ kg, $k = 8$ N/m, en $c = 4$ Ns/m.

\textbf{Vraag:}
\begin{enumerate}[label=(\alph*)]
\item Schrijf de differentiaalvergelijking.
\item Bepaal de natuurlijke eigenfrequentie $\omega_0$.
\item Is het systeem ondergedempt, kritisch gedempt, of overgedempt?
\end{enumerate}

\newpage
\section{Oplossingen}

\subsection{Oplossingen Hoofdstuk 1}

\subsubsection*{Oplossing 1.1}
Het systeem is \textbf{niet lineair}.

\textbf{Bewijs:}
\begin{itemize}
\item Test eigenschap 1 (schaling): $\mathcal{T}\{ax(t)\} = 3ax(t) + 2 \neq a\mathcal{T}\{x(t)\} = a(3x(t) + 2)$
\item Omdat de schalingsvoorwaarde niet voldaan is, is het systeem niet lineair.
\end{itemize}

\subsubsection*{Oplossing 1.2}
\begin{enumerate}[label=(\alph*)]
\item De differentiaalvergelijking: $RC\frac{dv_{\text{uit}}}{dt} + v_{\text{uit}} = v_{\text{in}}$
\[
(1000)(10 \times 10^{-6})\frac{dv_{\text{uit}}}{dt} + v_{\text{uit}} = 5u(t)
\]
\[
0.01\frac{dv_{\text{uit}}}{dt} + v_{\text{uit}} = 5u(t)
\]

\item Tijdsconstante: $\tau = RC = (1000)(10 \times 10^{-6}) = 0.01$ s $= 10$ ms

\item Voor een staprespons met $v_{\text{uit}}(0) = 0$:
\[
v_{\text{uit}}(t) = 5(1 - e^{-t/\tau})u(t) = 5(1 - e^{-100t})u(t)
\]
Op $t = 10$ ms $= 0.01$ s:
\[
v_{\text{uit}}(0.01) = 5(1 - e^{-1}) = 5(1 - 0.368) = 5(0.632) = 3.16 \text{ V}
\]
\end{enumerate}

\subsubsection*{Oplossing 1.3}
Model radioactief verval: $N(t) = N_0 e^{-kt}$

Halveringstijd $t_{1/2} = 6$ uur: $\frac{1}{2} = e^{-6k} \Rightarrow k = \frac{\ln(2)}{6} = 0.1155$ h$^{-1}$

Van 08:00 tot 14:00 is $\Delta t = 6$ uur:
\[
N(6) = 20 e^{-0.1155 \times 6} = 20 e^{-0.693} = 20 \times 0.5 = 10 \text{ mg}
\]

\subsection{Oplossingen Hoofdstuk 2}

\subsubsection*{Oplossing 2.1}
\begin{enumerate}[label=(\alph*)]
\item $x_1(t) = e^{0.2t}$: exponentiële \textbf{groei} (positieve exponent)\\
$x_2(t) = e^{-0.5t}$: exponentieel \textbf{verval} (negatieve exponent)

\item Op $t = 5$ s:
\begin{align*}
x_1(5) &= e^{0.2 \times 5} = e^{1} = 2.718\\
x_2(5) &= e^{-0.5 \times 5} = e^{-2.5} = 0.082
\end{align*}
\end{enumerate}

\subsubsection*{Oplossing 2.2}
\begin{enumerate}[label=(\alph*)]
\item Van $x(t) = 3\sin(4\pi t + \frac{\pi}{6})$:
\begin{itemize}
\item Amplitude: $A = 3$
\item Hoekfrequentie: $\omega = 4\pi$ rad/s
\item Frequentie: $f = \frac{\omega}{2\pi} = \frac{4\pi}{2\pi} = 2$ Hz
\item Fasehoek: $\phi = \frac{\pi}{6}$ rad $= 30°$
\end{itemize}

\item Als cosinusfunctie: $\sin(\theta) = \cos(\theta - \frac{\pi}{2})$
\[
x(t) = 3\cos\left(4\pi t + \frac{\pi}{6} - \frac{\pi}{2}\right) = 3\cos\left(4\pi t - \frac{\pi}{3}\right)
\]
\end{enumerate}

\subsubsection*{Oplossing 2.3}
\begin{enumerate}[label=(\alph*)]
\item Formule van Euler: $e^{j\theta} = \cos(\theta) + j\sin(\theta)$
\[
z(t) = e^{j2\pi t} = \cos(2\pi t) + j\sin(2\pi t)
\]

\item Op $t = 0.25$ s:
\[
z(0.25) = \cos(2\pi \times 0.25) + j\sin(2\pi \times 0.25) = \cos(\frac{\pi}{2}) + j\sin(\frac{\pi}{2}) = 0 + j \cdot 1 = j
\]
\end{enumerate}

\subsubsection*{Oplossing 2.4}
Voor twee identieke pulsen van breedte 1:
\[
(f * g)(t) = \begin{cases}
0, & t < 0\\
t, & 0 \le t < 1\\
2-t, & 1 \le t < 2\\
0, & t \ge 2
\end{cases}
\]
Dit is een driehoeksfunctie met maximum 1 op $t=1$ en breedte 2.

\subsection{Oplossingen Hoofdstuk 3}

\subsubsection*{Oplossing 3.1}
\begin{enumerate}[label=(\alph*)]
\item $\mathcal{L}\{5u(t)\} = \frac{5}{s}$, ROC: $\sigma > 0$

\item $\mathcal{L}\{e^{-3t}u(t)\} = \frac{1}{s+3}$, ROC: $\sigma > -3$

\item $\mathcal{L}\{t \cdot u(t)\} = \frac{1}{s^2}$, ROC: $\sigma > 0$

\item $\mathcal{L}\{\cos(5t) \cdot u(t)\} = \frac{s}{s^2 + 25}$, ROC: $\sigma > 0$
\end{enumerate}

\subsubsection*{Oplossing 3.2}
\[
F(s) = \frac{3}{s+2} + \frac{5}{s^2 + 4} = \frac{3}{s+2} + \frac{5}{s^2 + 2^2}
\]

Inverse Laplacetransformatie:
\[
f(t) = 3e^{-2t}u(t) + \frac{5}{2}\sin(2t)u(t)
\]

\subsubsection*{Oplossing 3.3}
Laplacetransformatie van beide zijden:
\[
sY(s) - y(0) + 4Y(s) = \frac{8}{s}
\]
\[
sY(s) - 2 + 4Y(s) = \frac{8}{s}
\]
\[
Y(s)(s+4) = \frac{8}{s} + 2
\]
\[
Y(s) = \frac{8}{s(s+4)} + \frac{2}{s+4}
\]

Partieelbreuken voor eerste term:
\[
\frac{8}{s(s+4)} = \frac{A}{s} + \frac{B}{s+4} = \frac{2}{s} - \frac{2}{s+4}
\]

Dus:
\[
Y(s) = \frac{2}{s} - \frac{2}{s+4} + \frac{2}{s+4} = \frac{2}{s}
\]

Inverse Laplacetransformatie:
\[
y(t) = 2u(t)
\]

\textbf{Verificatie:} Steady-state waarde: $y_{\infty} = \frac{8}{4} = 2$ ✓

\subsubsection*{Oplossing 3.4}
\begin{enumerate}[label=(\alph*)]
\item Karakteristieke vergelijking: $\lambda^2 + 4\lambda + 3 = 0$

\item Wortels: 
\[
\lambda = \frac{-4 \pm \sqrt{16-12}}{2} = \frac{-4 \pm 2}{2}
\]
$\lambda_1 = -1$, $\lambda_2 = -3$

\item Algemene oplossing:
\[
y(t) = c_1 e^{-t} + c_2 e^{-3t}
\]

Met beginvoorwaarden $y(0) = 1$ en $y'(0) = 0$:
\[
c_1 + c_2 = 1
\]
\[
-c_1 - 3c_2 = 0 \Rightarrow c_1 = -3c_2
\]

Oplossen: $c_2 = -\frac{1}{2}$, $c_1 = \frac{3}{2}$

\[
y(t) = \frac{3}{2}e^{-t} - \frac{1}{2}e^{-3t}
\]
\end{enumerate}

\subsection{Oplossingen Hoofdstuk 5}

\subsubsection*{Oplossing 5.1}
\begin{enumerate}[label=(\alph*)]
\item De functie is \textbf{oneven}: $f(-t) = -f(t)$

\item Fouriercoëfficiënten ($\omega_0 = \frac{2\pi}{T} = \pi$ rad/s):
\[
a_0 = \frac{1}{T}\int_0^T f(t)dt = \frac{1}{2}\left[\int_0^1 1\,dt + \int_1^2 (-1)dt\right] = \frac{1}{2}[1 - 1] = 0
\]

Voor oneven functie: $a_n = 0$ voor alle $n$

\[
b_n = \frac{2}{T}\int_0^T f(t)\sin(n\omega_0 t)dt = \int_0^1 \sin(n\pi t)dt - \int_1^2 \sin(n\pi t)dt
\]

\[
b_n = \begin{cases}
\frac{4}{n\pi}, & n \text{ oneven}\\
0, & n \text{ even}
\end{cases}
\]

\item Fourierreeks tot 3e harmonische:
\[
f(t) \approx \frac{4}{\pi}\sin(\pi t) + \frac{4}{3\pi}\sin(3\pi t)
\]
\end{enumerate}

\subsubsection*{Oplossing 5.2}
\begin{enumerate}[label=(\alph*)]
\item $a_0 = \frac{1}{1}\int_0^1 2t\,dt = 2\left[\frac{t^2}{2}\right]_0^1 = 1$

\item Voor $b_n$ (oneven functie, dus $a_n = 0$):
\[
b_n = 2\int_0^1 2t\sin(2\pi n t)dt
\]
Met partiële integratie:
\[
b_n = -\frac{2}{n\pi}
\]

Dus: $b_1 = -\frac{2}{\pi}$, $b_2 = -\frac{1}{\pi}$

\item Benaderende Fouriersom:
\[
f(t) \approx 1 - \frac{2}{\pi}\sin(2\pi t) - \frac{1}{\pi}\sin(4\pi t)
\]
\end{enumerate}

\subsection{Oplossingen Hoofdstuk 6}

\subsubsection*{Oplossing 6.1}
\begin{enumerate}[label=(\alph*)]
\item Convolutie:
\[
y(t) = h(t) * u(t) = \int_0^t 2e^{-5\tau}d\tau = 2\left[-\frac{1}{5}e^{-5\tau}\right]_0^t = \frac{2}{5}(1 - e^{-5t})u(t)
\]

\item Via Laplacetransformatie:
\[
H(s) = \frac{2}{s+5}, \quad F(s) = \frac{1}{s}
\]
\[
Y(s) = H(s)F(s) = \frac{2}{s(s+5)}
\]
Partieelbreuken: $\frac{2}{s(s+5)} = \frac{2/5}{s} - \frac{2/5}{s+5}$

\[
y(t) = \frac{2}{5}(1 - e^{-5t})u(t)
\]
Beide methoden geven hetzelfde resultaat ✓
\end{enumerate}

\subsubsection*{Oplossing 6.2}
\begin{enumerate}[label=(\alph*)]
\item Differentiaalvergelijking:
\[
m\frac{d^2y}{dt^2} + c\frac{dy}{dt} + ky = f(t)
\]
\[
2\frac{d^2y}{dt^2} + 4\frac{dy}{dt} + 8y = f(t)
\]

\item Natuurlijke eigenfrequentie:
\[
\omega_0 = \sqrt{\frac{k}{m}} = \sqrt{\frac{8}{2}} = 2 \text{ rad/s}
\]

\item Karakteristieke vergelijking: $2\lambda^2 + 4\lambda + 8 = 0 \Rightarrow \lambda^2 + 2\lambda + 4 = 0$
\[
\lambda = \frac{-2 \pm \sqrt{4-16}}{2} = -1 \pm j\sqrt{3}
\]
Complexe wortels → Het systeem is \textbf{ondergedempt}.
\end{enumerate}

\end{document}

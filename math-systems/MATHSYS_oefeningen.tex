\documentclass[a4paper,11pt]{article}

\usepackage[dutch]{babel}
\usepackage[utf8]{inputenc}
\usepackage{amsmath}
\usepackage{amssymb}
\usepackage{geometry}
\usepackage{enumitem}
\usepackage{graphicx}
\usepackage{tikz}
\usepackage{pgfplots}
\pgfplotsset{compat=1.18}
\usetikzlibrary{patterns,decorations.markings,arrows.meta,shapes}
\usepackage[hidelinks]{hyperref}

\geometry{margin=2.5cm}

\title{Wiskunde voor Systemen\\Oefeningen}
\author{KU Leuven -- ESAT\\Campus Groep T\\[0.5em]\small Gebaseerd op materiaal van Prof. Toon van Waterschoot\\Opgefrist en uitgebreid -- 2025--2026}
\date{\today}

\begin{document}

\maketitle
\tableofcontents
\newpage

% ===================================================================
% CHAPTER 1: AFGELEIDEN EN INTEGRALEN
% ===================================================================
\section{Afgeleiden en integralen}

\subsection{Beknopte theorie}

\subsubsection{Afgeleiden}

De afgeleide van een functie $f(t)$ naar de onafhankelijke variabele $t$ (meestal tijd in contexten van signalen en systemen) wordt aangeduid met $\frac{df(t)}{dt}$ of $f'(t)$. De afgeleide geeft aan hoe snel de functie verandert.

\textbf{Vier fundamentele afgeleiden:}
\begin{enumerate}
\item Machtsfunctie: $(t^n)' = nt^{n-1}$
\item Sinusfunctie: $(\sin t)' = \cos t$
\item Cosinusfunctie: $(\cos t)' = -\sin t$
\item Exponentiële functie: $(e^t)' = e^t$
\end{enumerate}

\textbf{Zes differentiatie-eigenschappen:}
\begin{enumerate}
\item Constante: $(c)' = 0$ voor $c \in \mathbb{R}$
\item Som: $(f+g)'(t) = f'(t) + g'(t)$
\item Product: $(fg)'(t) = f'(t)g(t) + f(t)g'(t)$
\item Quotiënt: $\left(\frac{f}{g}\right)'(t) = \frac{f'(t)g(t) - f(t)g'(t)}{g^2(t)}$ voor $g(t) \neq 0$
\item Samenstelling (kettingregel): $(f \circ g)'(t) = g'(t) \cdot f'\left(g(t)\right)$
\item Inverse functie: $(f^{-1})'(t) = \frac{1}{f'\left(f^{-1}(t)\right)}$
\end{enumerate}

\subsubsection{Integratie}

Integratie is het omgekeerde van differentiatie. De \textbf{onbepaalde integraal} van $f(t)$ naar $t$ is de verzameling van alle primitieve functies:

\[
\int f(t)\,dt = g(t) + c, \quad \text{waarbij } g'(t) = f(t) \text{ en } c \in \mathbb{R}
\]

De \textbf{bepaalde integraal} van $f(t)$ over het interval $[t_1, t_2]$ berekenen we via de hoofdstelling van de integraalrekening:

\[
\int_{t_1}^{t_2} f(t)\,dt = \left[g(t)\right]_{t_1}^{t_2} = g(t_2) - g(t_1)
\]

\textbf{Lineariteit van integratie:}
\[
\int \left(af_1(t) + bf_2(t)\right) dt = a\int f_1(t)\,dt + b\int f_2(t)\,dt
\]

\subsection{Oefeningen}

\begin{enumerate}

\item Bereken de afgeleide $f'(t)$ voor de volgende functies:
\begin{enumerate}[label=(\alph*)]
\item $f(t) = \tan t$
\item $f(t) = t^2 \cos t$
\item $f(t) = \sin t \tan t$
\item $f(t) = t^3 \sin t \cos t$
\item $f(t) = \tan(5 - \sin^2 t)$
\item $f(t) = \frac{1}{(1-2t)^3}$
\item $f(t) = \sin\left(\cos(2t-5)\right)$
\item $f(t) = \sqrt{1 + \cos(t^2)}$
\end{enumerate}

\item Bereken de onbepaalde integraal $\int f(t)\,dt$ voor:
\begin{enumerate}[label=(\alph*)]
\item $f(t) = (t^3 + t)^5(3t^2 + 1)$
\item $f(t) = \sqrt{2t + 1}$
\item $f(t) = t^2 \cos(t^3)$
\item $f(t) = (1 + \sqrt{t})^{1/3} / \sqrt{t}$
\item $f(t) = \left(1 - \cos\frac{t}{2}\right)^2 \sin\frac{t}{2}$
\item $f(t) = \frac{1}{t^2} \sin\frac{1}{t} \cos\frac{1}{t}$
\end{enumerate}

\item Bereken de bepaalde integralen:
\begin{enumerate}[label=(\alph*)]
\item $\int_{-1}^{1} 3t^2\sqrt[3]{t^3+1}\,dt$
\item $\int_{\pi/4}^{\pi/2} \cot t \csc^2 t\,dt$
\item $\int_0^{\pi/2} \frac{2\sin t \cos t}{(1+\sin^2 t)^3}\,dt$
\item $\int_0^{+\infty} e^{-t^2}\,dt$
\item $\int_{-\infty}^{+\infty} \frac{dt}{1+t^2}$
\item $\int_{-\infty}^{+\infty} \frac{2t\,dt}{(t^2+1)^2}$
\item $\int_{-\infty}^{+\infty} 2te^{-t^2}\,dt$
\item $\int_0^{-\infty} e^{-|t|}\,dt$
\end{enumerate}

\end{enumerate}

% ===================================================================
% CHAPTER 2: DIFFERENTIAALVERGELIJKINGEN
% ===================================================================
\newpage
\section{Differentiaalvergelijkingen}

\subsection{Beknopte theorie}

Een lineaire differentiaalvergelijking met constante coëfficiënten (LDV) van orde $N$ heeft de vorm:

\[
a_N\frac{d^N y}{dt^N} + a_{N-1}\frac{d^{N-1}y}{dt^{N-1}} + \cdots + a_1\frac{dy}{dt} + a_0 y = f(t)
\]

waarbij $y(t)$ de onbekende functie is en $f(t)$ een gegeven functie (invoersignaal).

\subsubsection{Homogene oplossing}

De homogene oplossing $y_h(t)$ is de oplossing wanneer het rechterlid nul is:

\[
a_N\frac{d^N y_h}{dt^N} + \cdots + a_0 y_h = 0
\]

Dit wordt opgelost door de karakteristieke veelterm in te voeren:

\[
a_N\lambda^N + a_{N-1}\lambda^{N-1} + \cdots + a_0 = 0
\]

De vorm van $y_h(t)$ hangt af van de wortels van deze veelterm:

\begin{itemize}
\item \textbf{Verschillende reële wortels} $\lambda_i$: $y_h(t) = \sum_i C_i e^{\lambda_i t}$
\item \textbf{Samenvallende reële wortels} $\lambda$ (multipliciteit $m$): termen als $t^k e^{\lambda t}$ voor $k = 0,1,\ldots,m-1$
\item \textbf{Complexe geconjugeerde wortels} $\lambda = \alpha \pm j\beta$: $y_h(t) = e^{\alpha t}(C_1\cos(\beta t) + C_2\sin(\beta t))$
\end{itemize}

\subsubsection{Particuliere oplossing}

De particuliere oplossing $y_p(t)$ is een enige oplossing van de inhomogene vergelijking. Deze kan gevonden worden met de methode van onbepaalde coëfficiënten:

\begin{enumerate}
\item Kies een aansat voor $y_p(t)$ gebaseerd op de vorm van $f(t)$
\item Substitueer in de DV en los de onbekende coëfficiënten op
\end{enumerate}

\subsubsection{Algemene oplossing}

De algemene oplossing is:

\[
y(t) = y_h(t) + y_p(t)
\]

Beginvoorwaarden bepalen de constanten in $y_h(t)$.

\subsection{Oefeningen}

\begin{enumerate}

\item Los de volgende lineaire differentiaalvergelijkingen op:
\begin{enumerate}[label=(\alph*)]
\item $y'' + 6y' + 5y = 0$ met $y(0) = 4, y'(0) = 0$
\item $y'' + 6y' + 25y = 0$ met $y(0) = 4, y'(0) = 0$
\item $y'' + 6y' + 9y = 0$ met $y(0) = 2, y'(0) = 0$
\item $y' + 3y = 4e^{-t}$ met $y(0) = 1$
\item $y'' + 16y = 8\sin(2t)$ met $y(0) = 4, y'(0) = 0$
\item $y'' + 16y = 8\sin(4t)$ met $y(0) = 4, y'(0) = 0$
\item $y'' + 6y' + 25y = 689\sin(20t)$ met $y(0) = 2, y'(0) = 1$
\item $y'' + 6y' + 9y = 30\sin(3t)$ met $y(0) = 1, y'(0) = 0$
\item $y''' + 6y' = t$ met alle beginvoorwaarden nul
\end{enumerate}

\end{enumerate}

% ===================================================================
% CHAPTER 3-9: REMAINING CHAPTERS (STRUCTURE IN PLACE)
% ===================================================================
\newpage
\section{Signalen en systemen: een eerste kennismaking}

Oefeningen worden toegevoegd volgens dezelfde structuur als hierboven. 

\newpage
\section{Basissignalen en -bewerkingen}

Oefeningen worden toegevoegd volgens dezelfde structuur.

\newpage
\section{De Laplacetransformatie}

Oefeningen worden toegevoegd volgens dezelfde structuur.

\newpage
\section{De Fouriertransformatie}

Oefeningen worden toegevoegd volgens dezelfde structuur.

\newpage
\section{De Fourierreeks}

Oefeningen worden toegevoegd volgens dezelfde structuur.

\newpage
\section{Lineaire Tijdsinvariante Continue (LTC) systemen}

Oefeningen worden toegevoegd volgens dezelfde structuur.

\newpage
\section{Eigenwaarden en eigenvectoren}

Oefeningen worden toegevoegd volgens dezelfde structuur.

% ===================================================================
% SOLUTIONS
% ===================================================================
\newpage
\appendix
\section*{Oplossingen}

De oplossingen worden opgesteld volgens dezelfde structuur als in het officiële oefenboek, met volledige uitwerkingen en verwijzingen naar relevante theorie.

Voor volledige uitwerkingen, zie de aanvullende oplossingshandleiding.

\newpage

\section*{Verwijzingen}

Dit document maakt gebruik van het officiële KU Leuven Formularium ``Wiskunde voor Systemen'' en is gebaseerd op het cursusmateriaal van Prof. Toon van Waterschoot.

\textbf{Aanbevelingen voor aanvullende studie:}
\begin{itemize}
\item Signalen en Systemen: Oppenheim, Willsky, Nawab
\item Oefeningenbundel 2025--2026
\item WOLF online platform (KU Leuven)
\end{itemize}

\end{document}
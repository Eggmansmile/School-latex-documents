\documentclass[a4paper,11pt]{article}

\usepackage[dutch]{babel}
\usepackage[utf8]{inputenc}
\usepackage{amsmath}
\usepackage{amssymb}
\usepackage{geometry}
\usepackage{enumitem}
\usepackage{graphicx}
\usepackage{fancyhdr}

\geometry{margin=2.5cm}
\pagestyle{fancy}
\lhead{Oefeningen: Fourier, LTC & Eigenwaarden}
\rhead{Wiskunde voor Systemen}

\title{\textbf{Oefeningenmodule: De Kernconcepten}\\ \large Fourierreeksen, LTC Systemen & Eigenwaarden}
\author{Simpel, maar niet triviaal}
\date{\today}

\begin{document}

\maketitle
\tableofcontents
\newpage

\section{Oefeningen}

% =============================================================================
% FOURIERREEKSEN
% =============================================================================
\section{Onderwerp 1: Fourierreeksen}
\textit{Doel: Coëfficiënten berekenen en symmetrie gebruiken.}

\subsection*{Oefening 1.1: De Blokgolf (Symmetrie)}
Gegeven een periodieke functie $f(t)$ met periode $T=2\pi$:
\[
f(t) = \begin{cases}
    1 & \text{voor } 0 < t < \pi \\
    -1 & \text{voor } \pi < t < 2\pi
\end{cases}
\]
\begin{enumerate}[label=(\alph*)]
    \item Schets de functie over twee periodes (van $-2\pi$ tot $2\pi$).
    \item Is deze functie even, oneven, of geen van beide? Wat betekent dit voor de coëfficiënten $a_0$, $a_n$ en $b_n$?
    \item Bereken de niet-nul coëfficiënten en schrijf de eerste drie termen van de reeks op.
\end{enumerate}

\subsection*{Oefening 1.2: Fourierreeks door Inspectie}
Soms hoef je niet te integreren. Gegeven het signaal:
\[ f(t) = 3 + 2\cos(4t) - 5\sin(10t) \]
\begin{enumerate}[label=(\alph*)]
    \item Wat is de grondfrequentie $\omega_0$ van dit signaal? (Hint: zoek de grootste gemene deler van de frequenties).
    \item Bepaal direct de Fouriercoëfficiënten $a_0, a_n, b_n$.
\end{enumerate}

\subsection*{Oefening 1.3: Parseval (Energie)}
De stelling van Parseval zegt dat de vermogensinhoud in tijd gelijk is aan de som van de vermogens van de harmonischen: $P = a_0^2 + \frac{1}{2}\sum (a_n^2 + b_n^2)$.
\\Bereken het gemiddelde vermogen van het signaal uit Oefening 1.2:
\[ f(t) = 3 + 2\cos(4t) - 5\sin(10t) \]

\newpage

% =============================================================================
% LTC SYSTEMEN
% =============================================================================
\section{Onderwerp 2: LTC Systemen}
\textit{Doel: Van DV naar Transferfunctie, Polen & Stabiliteit.}

\subsection*{Oefening 2.1: Van DV naar $H(s)$}
Een systeem wordt beschreven door de differentiaalvergelijking:
\[ y''(t) + 3y'(t) + 2y(t) = x'(t) + 4x(t) \]
\begin{enumerate}[label=(\alph*)]
    \item Zet beide kanten om naar het Laplace-domein (neem aan dat alle beginvoorwaarden 0 zijn).
    \item Bepaal de transferfunctie $H(s) = \frac{Y(s)}{X(s)}$.
    \item Wat zijn de polen en nulpunten van dit systeem?
\end{enumerate}

\subsection*{Oefening 2.2: Impulsrespons en Stabiliteit}
Gegeven de transferfunctie:
\[ H(s) = \frac{1}{s^2 + 4s + 3} \]
\begin{enumerate}[label=(\alph*)]
    \item Splits de noemer in factoren $(s+a)(s+b)$.
    \item Gebruik partieelbreuksplitsing om $H(s)$ te schrijven als $\frac{A}{s+1} + \frac{B}{s+3}$.
    \item Bepaal de impulsrespons $h(t)$ (de inverse Laplace van $H(s)$).
    \item Is dit systeem BIBO-stabiel? Waarom?
\end{enumerate}

\subsection*{Oefening 2.3: Gedrag bij frequenties (Bode-intuïtie)}
Gegeven een eerste-orde systeem $H(s) = \frac{10}{s+2}$.
\begin{enumerate}[label=(\alph*)]
    \item Wat is de "DC-gain" (de versterking bij frequentie 0)? Hint: vul $s=0$ in.
    \item Wat gebeurt er met de versterking $|H(j\omega)|$ als de frequentie $\omega$ naar oneindig gaat?
    \item Wat voor type filter is dit? (Laagdoorlaat, Hoogdoorlaat?)
\end{enumerate}

\newpage

% =============================================================================
% EIGENWAARDEN
% =============================================================================
\section{Onderwerp 3: Eigenwaarden en Eigenvectoren}
\textit{Doel: Karakteristieke vergelijking en basis berekeningen.}

\subsection*{Oefening 3.1: Karakteristieke Vergelijking}
Gegeven de matrix $A$:
\[ A = \begin{pmatrix} 4 & 1 \\ 2 & 3 \end{pmatrix} \]
\begin{enumerate}[label=(\alph*)]
    \item Stel de karakteristieke vergelijking op: $\det(A - \lambda I) = 0$.
    \item Los deze kwadratische vergelijking op om de eigenwaarden $\lambda_1$ en $\lambda_2$ te vinden.
\end{enumerate}

\subsection*{Oefening 3.2: Eigenvectoren vinden}
Gebruik de eigenwaarden uit Oefening 3.1 ($\lambda_1=2, \lambda_2=5$).
\begin{enumerate}[label=(\alph*)]
    \item Bepaal de eigenvector $\mathbf{v}_1$ die hoort bij $\lambda_1 = 2$.
    (Los op: $(A - 2I)\mathbf{v} = \mathbf{0}$).
    \item Bepaal de eigenvector $\mathbf{v}_2$ die hoort bij $\lambda_2 = 5$.
\end{enumerate}

\subsection*{Oefening 3.3: Diagonaalmatrix en machten}
Gegeven de matrix $D = \begin{pmatrix} 2 & 0 \\ 0 & -1 \end{pmatrix}$.
\begin{enumerate}[label=(\alph*)]
    \item Wat zijn de eigenwaarden van $D$? (Dit zou je direct moeten zien).
    \item Bereken $D^5$ zonder de matrix 5 keer te vermenigvuldigen.
    (Hint: bij diagonaalmatrices mag je de macht per element nemen).
\end{enumerate}

\newpage

% =============================================================================
% OPLOSSINGEN
% =============================================================================
\section{Oplossingen}

\subsection*{Oplossing 1.1: De Blokgolf}
\begin{enumerate}[label=(\alph*)]
    \item \textbf{Symmetrie:} $f(t)$ is \textbf{oneven}, want $f(-t) = -f(t)$. (De grafiek is puntsymmetrisch rond de oorsprong).
    \item \textbf{Gevolg:} $a_0 = 0$ (gemiddelde is 0) en alle $a_n = 0$. We hoeven alleen $b_n$ te berekenen.
    \item \textbf{Berekening $b_n$:}
    \[ b_n = \frac{1}{\pi} \int_{0}^{2\pi} f(t)\sin(nt)dt = \frac{2}{\pi} \int_{0}^{\pi} (1)\sin(nt)dt \]
    (We integreren over de halve periode en doen $\times 2$ vanwege symmetrie).
    \[ = \frac{2}{\pi} \left[ \frac{-\cos(nt)}{n} \right]_0^\pi = \frac{2}{n\pi} (1 - \cos(n\pi)) \]
    Als $n$ even is ($2, 4, \dots$), is $1-1=0$.
    Als $n$ oneven is ($1, 3, \dots$), is $1-(-1)=2$.
    Dus: $b_n = \frac{4}{n\pi}$ voor oneven $n$.
    \[ f(t) \approx \frac{4}{\pi}\sin(t) + \frac{4}{3\pi}\sin(3t) + \frac{4}{5\pi}\sin(5t) \]
\end{enumerate}

\subsection*{Oplossing 1.2: Inspectie}
\begin{enumerate}[label=(\alph*)]
    \item Grondfrequentie van $4t$ en $10t$: De grootste gemene deler van 4 en 10 is 2. Dus $\omega_0 = 2$.
    \item De reeks is gewoon de som van sinussen/cosinussen.
    \[ a_0 = 3 \]
    \[ \text{Bij } 4t \ (2\omega_0): \quad a_2 = 2 \]
    \[ \text{Bij } 10t \ (5\omega_0): \quad b_5 = -5 \]
    Alle andere coëfficiënten zijn 0.
\end{enumerate}

\subsection*{Oplossing 1.3: Parseval}
Signaal: $3 + 2\cos(4t) - 5\sin(10t)$.
\[ P = a_0^2 + \frac{1}{2}a_2^2 + \frac{1}{2}b_5^2 \]
\[ P = 3^2 + \frac{1}{2}(2^2) + \frac{1}{2}(-5)^2 \]
\[ P = 9 + 2 + 12.5 = 23.5 \]

\subsection*{Oplossing 2.1: Van DV naar $H(s)$}
\begin{enumerate}[label=(\alph*)]
    \item Laplace: $s^2Y(s) + 3sY(s) + 2Y(s) = sX(s) + 4X(s)$
    \item $Y(s)(s^2+3s+2) = X(s)(s+4) \implies H(s) = \frac{s+4}{s^2+3s+2}$
    \item $H(s) = \frac{s+4}{(s+1)(s+2)}$.
    Nulpunt: $s=-4$. Polen: $s=-1, s=-2$.
\end{enumerate}

\subsection*{Oplossing 2.2: Impulsrespons}
\begin{enumerate}[label=(\alph*)]
    \item $s^2+4s+3 = (s+1)(s+3)$.
    \item $\frac{1}{(s+1)(s+3)} = \frac{A}{s+1} + \frac{B}{s+3}$.
    A vinden: bedek $(s+1)$, vul $s=-1$ in: $\frac{1}{-1+3} = 0.5$.
    B vinden: bedek $(s+3)$, vul $s=-3$ in: $\frac{1}{-3+1} = -0.5$.
    $H(s) = \frac{0.5}{s+1} - \frac{0.5}{s+3}$.
    \item $h(t) = 0.5e^{-t}u(t) - 0.5e^{-3t}u(t)$.
    \item Ja, stabiel. De polen (-1 en -3) zijn negatief. De exponentiële functies sterven uit.
\end{enumerate}

\subsection*{Oplossing 2.3: Frequentiegedrag}
\begin{enumerate}[label=(\alph*)]
    \item DC-gain ($s=0$): $H(0) = \frac{10}{2} = 5$.
    \item Als $\omega \to \infty$, dan $|H(j\omega)| = |\frac{10}{j\omega+2}| \approx \frac{10}{\omega} \to 0$.
    \item Laagdoorlaatfilter (laat lage frequenties door, blokkeert hoge).
\end{enumerate}

\subsection*{Oplossing 3.1: Karakteristieke vergelijking}
\begin{enumerate}[label=(\alph*)]
    \item $\det \begin{pmatrix} 4-\lambda & 1 \\ 2 & 3-\lambda \end{pmatrix} = (4-\lambda)(3-\lambda) - 2 = 0$.
    \item $\lambda^2 - 7\lambda + 12 - 2 = 0 \implies \lambda^2 - 7\lambda + 10 = 0$.
    Ontbinden: $(\lambda-2)(\lambda-5) = 0$.
    Eigenwaarden: $\lambda_1 = 2, \lambda_2 = 5$.
\end{enumerate}

\subsection*{Oplossing 3.2: Eigenvectoren}
\begin{enumerate}[label=(\alph*)]
    \item Voor $\lambda=2$:
    $\begin{pmatrix} 4-2 & 1 \\ 2 & 3-2 \end{pmatrix} \mathbf{v} = \begin{pmatrix} 2 & 1 \\ 2 & 1 \end{pmatrix} \begin{pmatrix} x \\ y \end{pmatrix} = \begin{pmatrix} 0 \\ 0 \end{pmatrix}$.
    Dit geeft $2x + y = 0 \implies y = -2x$. Kies $x=1$, dan $\mathbf{v}_1 = \begin{pmatrix} 1 \\ -2 \end{pmatrix}$.
    \item Voor $\lambda=5$:
    $\begin{pmatrix} 4-5 & 1 \\ 2 & 3-5 \end{pmatrix} \mathbf{v} = \begin{pmatrix} -1 & 1 \\ 2 & -2 \end{pmatrix} \begin{pmatrix} x \\ y \end{pmatrix} = \begin{pmatrix} 0 \\ 0 \end{pmatrix}$.
    Dit geeft $-x + y = 0 \implies y = x$. Kies $x=1$, dan $\mathbf{v}_2 = \begin{pmatrix} 1 \\ 1 \end{pmatrix}$.
\end{enumerate}

\subsection*{Oplossing 3.3: Diagonaalmatrix}
\begin{enumerate}[label=(\alph*)]
    \item Eigenwaarden zijn de diagonaalelementen: $\lambda_1 = 2, \lambda_2 = -1$.
    \item $D^5 = \begin{pmatrix} 2^5 & 0 \\ 0 & (-1)^5 \end{pmatrix} = \begin{pmatrix} 32 & 0 \\ 0 & -1 \end{pmatrix}$.
\end{enumerate}

\end{document}
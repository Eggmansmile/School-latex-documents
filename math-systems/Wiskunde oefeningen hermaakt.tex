\documentclass[12pt,a4paper,twoside]{book}
\usepackage[utf8]{inputenc}
\usepackage[dutch]{babel}
\usepackage{amsmath}
\usepackage{amssymb}
\usepackage{amsthm}
\usepackage{graphicx}
\usepackage{fancyhdr}
\usepackage{geometry}
\usepackage{xcolor}
\usepackage{tikz}
\usepackage{pgfplots}
\usepackage{subcaption}
\usepackage{float}
\usepackage{enumitem}
\usepackage{booktabs}
\usepackage{multirow}
\usepackage{array}
\usepackage{tcolorbox}
\usepackage{hyperref}
\usepackage{listings}

\pgfplotsset{compat=1.18}
\usetikzlibrary{positioning,arrows.meta,calc,patterns,decorations.pathreplacing}

\geometry{
    paper=a4paper,
    top=2.5cm,
    bottom=2.5cm,
    left=3cm,
    right=2.5cm,
    headheight=15pt,
    headsep=1.5cm
}

\raggedbottom

\pagestyle{fancy}
\fancyhf{}
\fancyhead[LE,RO]{\thepage}
\fancyhead[LO]{\rightmark}
\fancyhead[RE]{\leftmark}

\definecolor{hoofdkleur}{RGB}{0,51,102}
\definecolor{accentkleur}{RGB}{204,0,0}
\definecolor{achtergrond}{RGB}{240,248,255}
\definecolor{theoriecolor}{RGB}{230,242,255}
\definecolor{oefencolor}{RGB}{255,250,230}

\newtcolorbox{theorieblok}[1][]{
    colback=theoriecolor,
    colframe=hoofdkleur,
    fonttitle=\bfseries\large,
    title=#1,
    sharp corners,
    boxrule=1.5pt,
    left=8pt,
    right=8pt,
    top=8pt,
    bottom=8pt
}

\newtcolorbox{oefenblok}[1][]{
    colback=oefencolor,
    colframe=accentkleur,
    fonttitle=\bfseries\large,
    title=#1,
    sharp corners,
    boxrule=1pt,
    left=6pt,
    right=6pt,
    top=6pt,
    bottom=6pt
}

\newcommand{\concept}[1]{\textbf{\textcolor{hoofdkleur}{#1}}}
\newcommand{\belangrijk}[1]{\textbf{\textcolor{accentkleur}{#1}}}
\newcommand{\important}[1]{\textbf{\textcolor{accentkleur}{#1}}}
\newcommand{\wichtig}[1]{\emph{#1}}

\theoremstyle{definition}
\newtheorem{theorem}{Stelling}[chapter]
\newtheorem{lemma}[theorem]{Lemma}
\newtheorem{corollary}[theorem]{Gevolg}
\newtheorem{definition}[theorem]{Definitie}
\newtheorem{example}[theorem]{Voorbeeld}

\title{
    \Huge\textbf{Handboek Wiskunde voor Systemen} \\
    \LARGE Theorie en Toepassingen \\
    \vspace{1cm}
    \large Een Complete Gids voor Signaal- en Systeemanalyse
}
\author{Voor Ingenieursstudenten en Professionals}
\date{\today}

\begin{document}



\newpage

\tableofcontents

\mainmatter

\section{Formularium}
\label{sec:formularium}

Dit formularium is een compacte samenvatting van de standaardformules uit het ``Signals and Systems'' formularium. In de oefeningen wordt hiernaar verwezen.

\subsection{Laplace transform (LT)}
\label{form:laplace}

\subsubsection*{1.1 Definitie en eigenschappen}
\label{form:laplace-def}
\label{form:laplace-prop}
\begingroup
\setlength{\tabcolsep}{6pt}
\renewcommand{\arraystretch}{1.35}
\[
\begin{array}{@{}lcl@{}}
\text{Definitie:} & f(t)\,u(t) & \longleftrightarrow\; F(s)=\displaystyle\int_0^{\infty} f(t)e^{-st}\,dt\\
\text{translatie in } s: & f(t)e^{-at}\,u(t) & \longleftrightarrow\; F(s+a)\\
\text{translatie in } t: & f(t-a)\,u(t-a) & \longleftrightarrow\; e^{-as}F(s)\\
\text{afgeleide in } t: & \dfrac{d}{dt}\big(f(t)u(t)\big) & \longleftrightarrow\; sF(s)-f(0^+)\\
 & \dfrac{d^2}{dt^2}\big(f(t)u(t)\big) & \longleftrightarrow\; s^2F(s)-s f(0^+)-f'(0^+)\\

\text{afgeleide in } s: & t\,f(t)\,u(t) & \longleftrightarrow\; -\dfrac{d}{ds}F(s)\\
 & t^n f(t)\,u(t) & \longleftrightarrow\; (-1)^n\,\dfrac{d^n}{ds^n}F(s)\\
\text{convolutie:} & \big(f*g\big)(t)\,u(t) & \longleftrightarrow\; F(s)\,G(s)\\
\text{schaling:} & f(a t) & \longleftrightarrow\; \dfrac{1}{a}\,F\!\left(\dfrac{s}{a}\right)\\
\end{array}
\]
\endgroup

\noindent\textbf{Initial value theorem:}
\[
\lim_{t\to 0^+} f(t)=\lim_{s\to\infty} sF(s).
\]
\noindent\textbf{Final value theorem:}
\[
\lim_{t\to\infty} f(t)=\lim_{s\to 0} sF(s)\quad(\text{onder de gebruikelijke poolvoorwaarden}).
\]
\noindent\textbf{Link met FTC:}
\label{form:lt-ft-link}
\[
\mathrm{LT}(s=j\omega)=\mathrm{FTC}(j\omega)\quad \text{als } x(t) \text{ absoluut integreerbaar is.}
\]

\subsubsection*{1.2 Useful Laplace pairs}
\label{form:laplace-pairs}
\begingroup
\setlength{\tabcolsep}{10pt}
\renewcommand{\arraystretch}{1.35}
\[
\begin{array}{@{}lcl@{\qquad}lcl@{}}
e^{-at}u(t) & \longleftrightarrow & \dfrac{1}{s+a} & t^n u(t) & \longleftrightarrow & \dfrac{n!}{s^{n+1}}\\
\cos(a t)u(t) & \longleftrightarrow & \dfrac{s}{s^2+a^2} & \sin(a t)u(t) & \longleftrightarrow & \dfrac{a}{s^2+a^2}\\
\delta(t) & \longleftrightarrow & 1 & u(t) & \longleftrightarrow & \dfrac{1}{s}\\
t\cos(a t)u(t) & \longleftrightarrow & \dfrac{s^2-a^2}{(s^2+a^2)^2} & t\sin(a t)u(t) & \longleftrightarrow & \dfrac{2sa}{(s^2+a^2)^2}
\end{array}
\]
\endgroup

\subsection{Fourier transform (FTC)}
\label{form:ft}

\subsubsection*{2.1 Basic formulae}
\label{form:ft-def}
\label{form:ft-prop}
\[
X(\omega)=\int_{-\infty}^{\infty} x(t)e^{-j\omega t}\,dt,\qquad
x(t)=\frac{1}{2\pi}\int_{-\infty}^{\infty} X(\omega)e^{j\omega t}\,d\omega.
\]

\begingroup
\setlength{\tabcolsep}{6pt}
\renewcommand{\arraystretch}{1.35}
\[
\begin{array}{@{}lcl@{}}
\text{convolution theorem (tijd):} & f(t)*g(t) & \longleftrightarrow\; F(\omega)\,G(\omega)\\
\text{convolution theorem (freq.):} & f(t)\,g(t) & \longleftrightarrow\; \dfrac{1}{2\pi}\,(F*G)(\omega)\\
\text{translatie:} & x(t-t_0) & \longleftrightarrow\; X(\omega)e^{-j\omega t_0}\\
\text{symmetry (reëel $x$):} &  & X(-\omega)=X^*(\omega)\\
\text{time symmetry (reëel $x$):} & x(-t) & \longleftrightarrow\; X(-\omega)=X^*(\omega)\\
\text{link FS--FTC:} &  & X(k\omega_0)=T\,c_k\quad(\omega_0=2\pi/T)
\end{array}
\]
\endgroup

\subsubsection*{2.2 Useful Fourier pairs}
\label{form:ft-pairs}
We gebruiken $\mathrm{sinc}(x)=\dfrac{\sin(x)}{x}$.

\begingroup
\setlength{\tabcolsep}{8pt}
\renewcommand{\arraystretch}{1.4}
\[
\begin{array}{@{}lcl@{}}
\text{Block:} & x(t)=\begin{cases}A,& t\in[-L/2,L/2]\\0,& \text{elders}\end{cases}
& \longleftrightarrow\; X(\omega)=AL\,\mathrm{sinc}\!\left(\dfrac{\omega L}{2}\right)\\
\text{Sinc:} & x(t)=A\,\mathrm{sinc}(\omega_0 t)
& \longleftrightarrow\; X(\omega)=\begin{cases}\dfrac{A\pi}{\omega_0},& |\omega|<\omega_0\\0,& \text{elders}\end{cases}\\
\text{Impuls:} & \delta(t-t_0) & \longleftrightarrow\; e^{-j\omega t_0}\\
\text{Complex expon.:} & e^{j\omega_0 t} & \longleftrightarrow\; 2\pi\,\delta(\omega-\omega_0)\\
\text{Cosine:} & \cos(\omega_0 t) & \longleftrightarrow\; \pi\,[\delta(\omega+\omega_0)+\delta(\omega-\omega_0)]\\
\text{Sine:} & \sin(\omega_0 t) & \longleftrightarrow\; j\pi\,[\delta(\omega+\omega_0)-\delta(\omega-\omega_0)]\\
\text{Delta train:} & \sum\limits_{k=-\infty}^{\infty}\delta(t-kT) & \longleftrightarrow\; \dfrac{2\pi}{T}\sum\limits_{k=-\infty}^{\infty}\delta\!\left(\omega-\dfrac{2\pi k}{T}\right)
\end{array}
\]
\endgroup

\subsection{Fourier series (FS)}
\label{form:fs}

\subsubsection*{3.1 Cartesian form}
\label{form:fs-cart}
Voor periode $T$ met $\omega_0=\dfrac{2\pi}{T}$:
\[
f(t)=\frac{a_0}{2}+\sum_{k=1}^{\infty}\left[a_k\cos\left(\frac{2\pi k}{T}t\right)+b_k\sin\left(\frac{2\pi k}{T}t\right)\right].
\]
\[
a_0=\frac{2}{T}\int_{0}^{T} f(t)\,dt,\qquad
a_k=\frac{2}{T}\int_{0}^{T} f(t)\cos\left(\frac{2\pi k}{T}t\right)\,dt,\qquad
b_k=\frac{2}{T}\int_{0}^{T} f(t)\sin\left(\frac{2\pi k}{T}t\right)\,dt.
\]

\subsubsection*{3.2 Complex form}
\label{form:fs-cplx}
\[
x(t)=\sum_{k=-\infty}^{\infty} c_k e^{jk\omega_0 t},\qquad \omega_0=\frac{2\pi}{T},\qquad
c_k=\frac{1}{T}\int_{0}^{T} f(t)\,e^{-jk\omega_0 t}\,dt.
\]
\noindent\textbf{Symmetrie (reëel $f$):}\; $c_{-k}=c_k^*$.

\noindent\textbf{Spectrum:} 
\[
X(\omega)=\frac{2\pi}{T}\sum_{k=-\infty}^{\infty} c_k\,\delta(\omega-k\omega_0).
\]

\subsubsection*{3.3 Links between cartesian and complex form}
\label{form:fs-ft-link}
\[
c_k=\frac{a_k-jb_k}{2},\qquad c_k^*=\frac{a_k+jb_k}{2}.
\]
\[
|c_k|=\frac{1}{2}\sqrt{a_k^2+b_k^2},\qquad \varphi_k=\mathrm{Arctan2}(a_k,-b_k).
\]
\[
a_k=2\,\mathrm{Re}\{c_k\},\qquad b_k=-2\,\mathrm{Im}\{c_k\}.
\]

\newpage

% ============================================================================
% HOOFDSTUK 1: AFGELEIDEN EN INTEGRALEN
% ============================================================================

\chapter{Afgeleiden en Integralen}

\section{Fundamentele Analyse in Systeemcontext}

De studie van dynamische systemen vereist een diepgaand begrip van de calculus, specifiek de differentiaal- en integraalrekening. In de context van signaalanalyse en systeemmodellering is de onafhankelijke variabele vrijwel zonder uitzondering de tijd \(t\). Calculus biedt ons de woordenschat om te spreken over verandering en accumulatie, twee concepten die essentieel zijn voor het begrijpen van de fysieke wereld.

Functies, die we noteren als \(f(t)\), representeren fysische grootheden zoals:
\begin{itemize}
    \item Elektrische spanning over een condensator (energieopslag in een elektrisch veld)
    \item De snelheid van een mechanische massa (kinetische energie)
    \item De temperatuur in een reactievat (thermische energie)
    \item De verplaatsing van een trillend voorwerp (potentiële energie)
\end{itemize}

De wiskundige operaties die op deze signalen worden uitgevoerd, vormen de taal waarin het gedrag van systemen wordt beschreven en voorspeld.

\subsection{De Taal van Verandering en Geheugen}

Calculus in systeemtheorie draait om twee kernconcepten:
\begin{enumerate}
    \item \textbf{Differentiatie (Verandering):} Hoe reageert het systeem \emph{nu} op wat er \emph{nu} gebeurt? Differentiatie kijkt naar de momentane verandering. Het is de wiskunde van het "nu". In fysieke systemen is dit vaak gekoppeld aan krachten die beweging tegenwerken (demping) of veroorzaken (versnelling).
    \item \textbf{Integratie (Geheugen):} Wat is de historie van het systeem? Integratie is het proces van accumulatie. De lading op een condensator is de integraal van de stroom die er in het verleden doorheen heeft gevloeid. Integratie is de wiskunde van het "verleden". Zonder integratie zouden systemen geen geheugen hebben en geen energie kunnen opslaan.
\end{enumerate}

\subsection{Differentiatie: De Snelheid van Verandering}

\begin{theorieblok}[Definitie van de Afgeleide]
Differentiatie is het wiskundige proces waarmee de momentane veranderingssnelheid van een grootheid wordt gekwantificeerd. De afgeleide, genoteerd als \(\frac{df(t)}{dt}\) of kortweg \(f'(t)\), is zelf een tijdsfunctie die op elk tijdstip \(t\) aangeeft hoe snel het oorspronkelijke signaal \(f(t)\) toeneemt of afneemt.

Geometrisch gezien correspondeert deze waarde met de helling van de raaklijn aan de curve van \(f(t)\) in het punt \((t, f(t))\).
\end{theorieblok}

In de systeemtheorie speelt de afgeleide een centrale rol omdat veel fysische wetten geformuleerd zijn in termen van veranderingen:

\begin{enumerate}
    \item \textbf{Tweede wet van Newton}: Relateert kracht aan versnelling, de tweede afgeleide van positie: \(F = ma = m\ddot{x}(t)\). Hier vertelt de afgeleide ons hoe de snelheid verandert.
    \item \textbf{Capacitor-stroomrelatie}: De stroom door een condensator is evenredig met de tijdsafgeleide van de spanning: \(i = C\frac{dv(t)}{dt}\). Hier vertelt de afgeleide ons hoe snel de energieopslag verandert.
    \item \textbf{Inductor-spanningsrelatie}: De spanning over een spoel is evenredig met de verandering van de stroom: \(v = L\frac{di(t)}{dt}\).
\end{enumerate}

Zonder beheersing van differentiatie is het onmogelijk om de differentiaalvergelijkingen op te stellen die deze systemen regeren.

\subsubsection{Fundamentele Afgeleiden}

Voor de analyse die in dit werk wordt gepresenteerd, zijn vier fundamentele afgeleiden van cruciaal belang:

\begin{definition}[Machtsfunctie]
\[
\frac{d}{dt}[t^n] = n t^{n-1}
\]
Dit gedrag is essentieel bij het bestuderen van polynomiale ingangssignalen zoals ramp-functies \(t \cdot u(t)\).
\end{definition}

\begin{definition}[Goniometrische functies]
\[
\frac{d}{dt}[\sin(t)] = \cos(t), \quad \frac{d}{dt}[\cos(t)] = -\sin(t)
\]
Deze relatie vormt de basis voor de analyse van oscillerende systemen en wisselstroomkringen.
\end{definition}

\begin{definition}[Exponentiële functie]
\[
\frac{d}{dt}[e^t] = e^t
\]
Dit is de unieke eigenschap waarbij de afgeleide gelijk is aan de functie zelf. Dit maakt de exponentiële functie tot de \concept{'eigenfunctie'} van lineaire tijdsinvariante systemen.
\end{definition}

\subsubsection{Rekenregels voor Afgeleiden}

De kracht van differentiatie ligt in de rekenregels die het mogelijk maken complexe signalen te ontleden:

\begin{theorem}[Lineariteit]
\[
\frac{d}{dt}[af(t) + bg(t)] = a\frac{df(t)}{dt} + b\frac{dg(t)}{dt}
\]
Dit staat centraal in de theorie van lineaire systemen.
\end{theorem}

\begin{theorem}[Productregel]
\[
\frac{d}{dt}[f(t)g(t)] = f'(t)g(t) + f(t)g'(t)
\]
Dit is onmisbaar bij de analyse van signalen met variërende amplitudes, zoals AM-modulatie \(A(t)\cos(\omega t)\).
\end{theorem}

\begin{theorem}[Kettingregel]
\[
\frac{d}{dt}[f(g(t))] = f'(g(t)) \cdot g'(t)
\]
Dit stelt ons in staat het effect van tijdschaling en tijdsverschuiving te analyseren, wat direct toepasbaar is op het Dopplereffect of vertraagde signalen.
\end{theorem}

\subsection{Integratie: Accumulatie en Oppervlakte}

\begin{theorieblok}[Definitie van de Integraal]
Integratie is de inverse operatie van differentiatie en vertegenwoordigt het proces van accumulatie. Als differentiatie ons vertelt hoe snel iets verandert, vertelt integratie ons hoeveel er in totaal is opgebouwd.
\end{theorieblok}

\begin{example}
Als \(f(t)\) de stroomsnelheid van water in een tank is (liters per seconde), dan is de integraal van \(f(t)\) het totale volume water in de tank (liters).
\end{example}

\begin{definition}
Wiskundig gezien: als \(f(t) = g'(t)\), dan is \(g(t)\) de primitieve functie van \(f(t)\). De onbepaalde integraal \(\int f(t) dt\) representeert de verzameling van alle mogelijke primitieve functies, die onderling slechts verschillen door een constante \(C\).
\end{definition}

\subsubsection{Integratietechnieken}

\begin{theorem}[Substitutiemethode]
Deze techniek, die in wezen de omgekeerde kettingregel is, laat ons toe een complexe integrand \(f(g(t))g'(t)\) te vereenvoudigen door de variabele \(u = g(t)\) te introduceren, wat leidt tot \(\int f(u) du\).
\end{theorem}

Dit is bijzonder nuttig bij het berekenen van de energie van signalen met complexe argumenten.

\begin{definition}[Bepaalde Integraal]
\[
\int_{t_1}^{t_2} f(t) dt
\]
kwantificeert de ``netto oppervlakte'' onder de curve tussen twee tijdstippen. In systeemcontext kan dit:
\begin{itemize}
    \item De totale energie zijn die gedissipeerd wordt in een weerstand
    \item De totale verplaatsing van een voertuig
    \item De hoeveelheid opgeslagen lading in een condensator
\end{itemize}
\end{definition}

\begin{theorem}[Hoofdstelling van de Integraalrekening]
\[
\int_{t_1}^{t_2} f(t) dt = g(t_2) - g(t_1)
\]
waarbij \(g(t) = \int f(t) dt\) de primitieve is. Dit legt de cruciale link: de totale verandering van een grootheid over een interval is gelijk aan het verschil in de waarden van de primitieve functie aan de grenzen.
\end{theorem}

\begin{definition}[Oneigenlijke Integraal]
Wanneer een integratiegrens oneindig is, bijvoorbeeld \(\int_0^\infty f(t) dt\), spreken we van een oneigenlijke integraal. Deze zijn fundamenteel voor transformatiemethoden zoals Laplace en Fourier, waarbij we het gedrag van signalen over hun gehele levensduur beschouwen.
\end{definition}

\section{Praktische Toepassingen en Oefeningen}

\subsection{Oefening 1.1: Analyse van Goniometrische Transformaties}

\begin{oefenblok}[Probleem]
Beschouw de functie \(f(t) = \tan(5 - \sin^2 t)\). Dit is een samengestelde functie die modellerend kan optreden in niet-lineaire fase-modulatie vraagstukken. Bepaal de afgeleide \(f'(t)\).
\end{oefenblok}

\begin{oefenblok}[Oplossing]
Om de veranderingssnelheid te vinden, passen we de kettingregel systematisch toe.

De buitenste functie is de tangens, waarvan de afgeleide \(\sec^2(\cdot)\) is. Het argument is \(u(t) = 5 - \sin^2 t\).

De afgeleide van dit argument vereist opnieuw de kettingregel voor de sinus-term:
\[
u'(t) = -\cos(2t) \cdot 2 = -2\cos(2t)
\]

Door deze componenten te combineren, vinden we de totale afgeleide:
\[
\boxed{f'(t) = \sec^2(5 - \sin^2 t) \cdot (-2\cos 2t)}
\]

Dit resultaat toont aan hoe de frequentie van de modulerende term (2t) direct de amplitude van de afgeleide beïnvloedt. Dit is essentieel in frequentiemodulatie (FM)-systemen.
\end{oefenblok}

\subsection{Oefening 1.2: Integratie van Gemoduleerde Signalen}

\begin{oefenblok}[Probleem]
Bepaal de onbepaalde integraal:
\[
\int t^2 \cos(t^3) \, dt
\]
Dit type integraal treedt op bij energieberekeningen van signalen met complexe modulatie.
\end{oefenblok}

\begin{oefenblok}[Oplossing]
Hoewel dit op het eerste gezicht complex lijkt, onthult de structuur een duidelijke substitutiemogelijkheid.

De term \(t^2\) is (op een factor na) de afgeleide van het argument \(t^3\). Door de substitutie \(u = t^3\) te kiezen, volgt:
\[
du = 3t^2 dt \quad \Rightarrow \quad t^2 dt = \frac{1}{3} du
\]

De integraal transformeert dan naar:
\[
\int \frac{1}{3} \cos(u) du = \frac{1}{3} \sin(u) + C
\]

Terugsubstitutie geeft de oplossing:
\[
\boxed{\int t^2 \cos(t^3) dt = \frac{1}{3} \sin(t^3) + C}
\]

Dit illustreert hoe patroonherkenning cruciaal is bij het oplossen van systeemintegralen.
\end{oefenblok}

\subsection{Oefening 1.3: Afgeleiden van Gedempte Trillingen}

\begin{oefenblok}[Probleem]
In de analyse van RLC-kringen en mechanische dempers komen functies voor van de vorm $f(t) = e^{-\alpha t} \cos(\omega t)$.
Bereken de eerste en tweede afgeleide van $x(t) = e^{-2t} \sin(10t)$.
Toon aan dat de verhouding tussen de amplitudes van de trilling en zijn afgeleide frequentie-afhankelijk is.
\end{oefenblok}

\begin{oefenblok}[Oplossing]
\textbf{Eerste afgeleide (Productregel):}
\[
x'(t) = (e^{-2t})' \sin(10t) + e^{-2t} (\sin(10t))'
\]
\[
x'(t) = -2e^{-2t} \sin(10t) + 10e^{-2t} \cos(10t)
\]
\[
x'(t) = e^{-2t} (10 \cos 10t - 2 \sin 10t)
\]

\textbf{Tweede afgeleide:}
Differentieer $x'(t)$ opnieuw.
Term 1: $(-2e^{-2t} \sin 10t)' = 4e^{-2t} \sin 10t - 20e^{-2t} \cos 10t$
Term 2: $(10e^{-2t} \cos 10t)' = -20e^{-2t} \cos 10t - 100e^{-2t} \sin 10t$

Som:
\[
x''(t) = e^{-2t} (4 \sin 10t - 20 \cos 10t - 20 \cos 10t - 100 \sin 10t)
\]
\[
x''(t) = e^{-2t} (-96 \sin 10t - 40 \cos 10t)
\]

\textbf{Inzicht:} Bij hoge frequenties ($\omega=10 \gg \alpha=2$) domineert de term met $\omega$ in de afgeleide. De snelheid ($x'$) is dan ongeveer $\omega$ keer zo groot als de uitwijking.
\end{oefenblok}

\subsection{Oefening 1.4: RMS Waarde (Effectieve Waarde)}

\begin{oefenblok}[Probleem]
De Root Mean Square (RMS) waarde van een signaal is een maat voor het vermogen. Bereken de RMS-waarde van $f(t) = A \cos(\omega t)$.
Formule: $V_{rms} = \sqrt{\frac{1}{T} \int_0^T f(t)^2 dt}$.
\end{oefenblok}

\begin{oefenblok}[Oplossing]
Kwadrateer: $f(t)^2 = A^2 \cos^2(\omega t)$.
Gebruik de dubbele hoek formule: $\cos^2 x = \frac{1 + \cos 2x}{2}$.
\[
\int_0^T A^2 \frac{1 + \cos(2\omega t)}{2} dt = \frac{A^2}{2} \left[ t + \frac{\sin(2\omega t)}{2\omega} \right]_0^T
\]
Omdat $T = 2\pi/\omega$, is $\sin(2\omega T) = \sin(4\pi) = 0$.
De integraal wordt $\frac{A^2}{2} \cdot T$.

Delen door $T$ en wortel trekken:
\[
V_{rms} = \sqrt{\frac{1}{T} \cdot \frac{A^2 T}{2}} = \sqrt{\frac{A^2}{2}} = \frac{A}{\sqrt{2}}
\]
Dit is de standaard factor $0.707$ die gebruikt wordt in het elektriciteitsnet.
\end{oefenblok}

\subsection{Oefening 1.5: Oneigenlijke Integraal}

\begin{oefenblok}[Probleem]
Bepaal de oneigenlijke integraal:
\[
\int_0^{\infty} e^{-t/2} dt
\]
Dit type integraal is de hoeksteen van de Laplace-transformatie.
\end{oefenblok}

\begin{oefenblok}[Oplossing]
De functie \(e^{-t/2}\) representeert een exponentieel uitdempend signaal met een tijdsconstante \(\tau = 2\).

De primitieve is:
\[
\int e^{-t/2} dt = -2e^{-t/2} + C
\]

Bij het evalueren van de grenzen nemen we de limiet voor \(t \to \infty\):
\[
\int_0^{\infty} e^{-t/2} dt = \lim_{T \to \infty} \left[-2e^{-T/2} - (-2e^0)\right] = 0 + 2 = \boxed{2}
\]

\textbf{Fysische interpretatie:} De totale oppervlakte (of ``lading'') van dit oneindig lange signaal is eindig (gelijk aan 2). Dit is een noodzakelijke voorwaarde voor stabiliteit in veel systemen.
\end{oefenblok}

\subsection{Oefening 1.6: Partiële Integratie}

\begin{oefenblok}[Probleem]
Bepaal de integraal:
\[
\int t e^{-2t} \, dt
\]
Dit type integraal komt vaak voor bij het berekenen van het zwaartepunt van een signaal of bij Laplace-transformaties.
\end{oefenblok}

\begin{oefenblok}[Oplossing]
We passen partiële integratie toe met de formule:
\[
\int u \, dv = uv - \int v \, du
\]

Kies:
\[
u = t \quad \Rightarrow \quad du = dt
\]
\[
dv = e^{-2t} dt \quad \Rightarrow \quad v = -\frac{1}{2}e^{-2t}
\]

Invullen in de formule geeft:
\[
\int t e^{-2t} dt = t \left(-\frac{1}{2}e^{-2t}\right) - \int \left(-\frac{1}{2}e^{-2t}\right) dt
\]
\[
= -\frac{1}{2}te^{-2t} + \frac{1}{2} \int e^{-2t} dt
\]

De resterende integraal is standaard:
\[
\int e^{-2t} dt = -\frac{1}{2}e^{-2t}
\]

Dus de volledige oplossing is:
\[
-\frac{1}{2}te^{-2t} + \frac{1}{2} \left(-\frac{1}{2}e^{-2t}\right) + C
\]

\[
\boxed{-\frac{1}{2}te^{-2t} - \frac{1}{4}e^{-2t} + C}
\]

Of gefactoriseerd:
\[
-\frac{1}{4}e^{-2t}(2t + 1) + C
\]
\end{oefenblok}

\subsection{Oefening 1.7: Productregel en Kettingregel}

\begin{oefenblok}[Probleem]
Bereken de afgeleide \(f'(t)\) voor de functie:
\[
f(t) = t^2 \cos(t)
\]
\end{oefenblok}

\begin{oefenblok}[Oplossing]
We gebruiken de productregel: \((uv)' = u'v + uv'\).
Stel \(u = t^2\) en \(v = \cos(t)\).
Dan is \(u' = 2t\) en \(v' = -\sin(t)\).

\[
f'(t) = (2t)(\cos(t)) + (t^2)(-\sin(t))
\]
\[
\boxed{f'(t) = 2t \cos(t) - t^2 \sin(t)}
\]
\end{oefenblok}



% ============================================================================
% HOOFDSTUK 2: DIFFERENTIAALVERGELIJKINGEN
% ============================================================================

\chapter{Differentiaalvergelijkingen}

\section{Modellering van Dynamische Systemen}

\begin{theorieblok}[Kern van Systeemtheorie]
Differentiaalvergelijkingen vormen het hart van de systeemtheorie. Ze beschrijven niet de toestand van een systeem op één moment, maar hoe de toestand verandert in de tijd. Een differentiaalvergelijking is in essentie een balansvergelijking van krachten, stromen of energieën.
\end{theorieblok}

Waarom gebruiken we differentiaalvergelijkingen? Omdat de natuurwetten lokaal zijn. Ze vertellen ons wat er \emph{nu} gebeurt op basis van de huidige toestand. Bijvoorbeeld: een veer oefent een kracht uit die afhangt van de huidige uitrekking, en een demper oefent een kracht uit die afhangt van de huidige snelheid.

Een Lineair Tijdsinvariant Continu (LTC) systeem wordt wiskundig gemodelleerd door een lineaire differentiaalvergelijking (LDV) met constante coëfficiënten. De algemene vorm voor een tweede-orde systeem, zoals een massa-veer-demper of een RLC-kring, is:

\[
a_2 \frac{d^2y(t)}{dt^2} + a_1 \frac{dy(t)}{dt} + a_0 y(t) = x(t)
\]

Hierin is:
\begin{itemize}
    \item \(y(t)\) de respons (output), bijvoorbeeld positie of lading.
    \item \(x(t)\) de excitatie (input), bijvoorbeeld een externe kracht of spanningsbron.
    \item \(a_i\) coëfficiënten bepaald door fysische parameters.
\end{itemize}

\subsection{Fysische Interpretatie van de Termen}

Elke term in de differentiaalvergelijking correspondeert met een specifiek fysisch mechanisme:

\begin{enumerate}
    \item \textbf{Traagheidsterm} (\(a_2 \frac{d^2y}{dt^2}\)):
    Deze term verzet zich tegen verandering van snelheid.
    \begin{itemize}
        \item Mechanisch: Massa \(m\) (wet van Newton \(F=ma\)).
        \item Elektrisch: Inductantie \(L\) (verzet zich tegen stroomverandering).
    \end{itemize}
    
    \item \textbf{Dempingsterm} (\(a_1 \frac{dy}{dt}\)):
    Deze term dissipeert energie en remt beweging af.
    \begin{itemize}
        \item Mechanisch: Demper \(c\) (wrijving evenredig met snelheid).
        \item Elektrisch: Weerstand \(R\) (dissipeert energie als warmte).
    \end{itemize}
    
    \item \textbf{Stijfheidsterm} (\(a_0 y(t)\)):
    Deze term wil het systeem terugbrengen naar de evenwichtsstand.
    \begin{itemize}
        \item Mechanisch: Veerconstante \(k\) (wet van Hooke).
        \item Elektrisch: Inverse capaciteit \(1/C\) (spanning bouwt op door lading).
    \end{itemize}
\end{enumerate}

Deze analogie tussen mechanische en elektrische domeinen toont de kracht van abstracte wiskunde: één vergelijking beschrijft talloze fysische realiteiten.

\subsection{De Homogene Oplossing: Natuurlijk Gedrag}

\begin{definition}[Homogene Vergelijking]
De eerste stap in de analyse is het bestuderen van het systeem zonder externe invloeden (\(x(t) = 0\)). Dit levert de homogene oplossing \(y_h(t)\), die het natuurlijke gedrag of de vrije respons van het systeem beschrijft.
\end{definition}

Dit gedrag wordt volledig bepaald door de interne dynamiek. We vinden deze oplossing door de wortels \(\lambda\) van de karakteristieke vergelijking op te lossen:

\[
a_2 \lambda^2 + a_1 \lambda + a_0 = 0
\]

\subsubsection{Drie Gedragstypes}

\begin{definition}[Overgedempt]
\textbf{Twee verschillende reële wortels} \(\lambda_1, \lambda_2\) (beide negatief voor stabiliteit).

De oplossing is een som van twee vervallende exponentiële functies:
\[
y_h(t) = C_1 e^{\lambda_1 t} + C_2 e^{\lambda_2 t}
\]

Het systeem keert traag terug naar evenwicht zonder oscillaties. Dit gedrag treedt op in sterk gedempt systemen, zoals een voertuig met sterke dempers.
\end{definition}

\begin{definition}[Kritisch Gedempt]
\textbf{Twee samenvallende reële wortels} \(\lambda\) (negatief).

De oplossing bevat een term \(te^{\lambda t}\):
\[
y_h(t) = (C_1 + C_2 t) e^{\lambda t}
\]

Dit representeert de snelst mogelijke terugkeer naar evenwicht zonder overshoot. Dit is optimaal voor veel praktische toepassingen.
\end{definition}

\begin{definition}[Ondergedempt]
\textbf{Twee complex toegevoegde wortels} \(\alpha \pm j\beta\).

De oplossing beschrijft een gedempte oscillatie:
\[
y_h(t) = e^{\alpha t}(C_1 \cos \beta t + C_2 \sin \beta t)
\]

Hierbij bepaalt:
\begin{itemize}
    \item \(\alpha\) (het reële deel) de dempingssnelheid
    \item \(\beta\) (het imaginaire deel) de oscillatiefrequentie
\end{itemize}

Dit is kenmerkend voor systemen met weinig wrijving, zoals een geluid-trillende snaarvork.
\end{definition}

\subsection{De Particuliere Oplossing: Gedwongen Respons}

\begin{definition}
Wanneer het systeem wordt aangedreven door een externe input \(x(t)\), ontstaat er een gedwongen respons, beschreven door de particuliere oplossing \(y_p(t)\).
\end{definition}

De vorm van deze oplossing volgt doorgaans de vorm van de input:
\begin{itemize}
    \item Een sinusoïde-input leidt tot een sinusoïdale respons (met andere fase/amplitude)
    \item Een constante input leidt tot een constante respons
    \item Een stap-input leidt tot een respons die naar een constant niveau convergeert
\end{itemize}

De \concept{methode van onbepaalde coëfficiënten} is een systematische manier om \(y_p(t)\) te vinden door een testfunctie in te vullen in de LDV en de constanten op te lossen.

\subsection{De Algemene Oplossing en Beginvoorwaarden}

\begin{theorem}
De totale respons van een lineair systeem is de superpositie van de natuurlijke en de gedwongen respons:
\[
y(t) = y_h(t) + y_p(t)
\]
\end{theorem}

Deze algemene oplossing bevat nog onbepaalde constanten (\(C_1, C_2\)) afkomstig uit het homogene deel. Deze worden vastgelegd door de initiële toestand van het systeem, de zogenaamde \concept{beginvoorwaarden} (\(y(0)\) en \(y'(0)\)).

Fysisch representeren deze bijvoorbeeld:
\begin{itemize}
    \item De initiële positie en snelheid van een massa
    \item De initiële spanning en stroom in een circuit
    \item De initiële concentratie in een chemisch proces
\end{itemize}

\wichtig{Opmerking:} Het is cruciaal om de constanten pas te bepalen nadat de volledige algemene oplossing is samengesteld, omdat de particuliere oplossing op \(t=0\) ook bijdraagt aan de beginwaarden.

\section{Analyse van Systeemgedrag}

\subsection{Oefening 2.1: Vrij Verval van een Eerste-Orde Systeem}

\begin{oefenblok}[Probleem]
Beschouw de differentiaalvergelijking:
\[
y'' + 6y' + 5y = 0
\]
met beginvoorwaarden \(y(0) = 4\) en \(y'(0) = 0\).

Dit modelleert een gedempt systeem zonder externe aandrijving dat wordt losgelaten vanuit een uitwijking.
\end{oefenblok}

\begin{oefenblok}[Oplossing]
\textbf{Stap 1: Karakteristieke Vergelijking}

\(\lambda^2 + 6\lambda + 5 = 0\), wat factoriseert tot \((\lambda + 1)(\lambda + 5) = 0\).

De wortels zijn reëel en verschillend: \(\lambda_1 = -1\) en \(\lambda_2 = -5\).

Dit duidt op een \textbf{overgedempt} systeem.

\vspace{0.3cm}
\textbf{Stap 2: Algemene Oplossing}

\[
y(t) = C_1 e^{-t} + C_2 e^{-5t}
\]

\vspace{0.3cm}
\textbf{Stap 3: Bepalen van Constanten}

Uit \(y(0) = 4\):
\[
C_1 + C_2 = 4
\]

De afgeleide is \(y'(t) = -C_1 e^{-t} - 5C_2 e^{-5t}\).

Uit \(y'(0) = 0\):
\[
-C_1 - 5C_2 = 0 \quad \Rightarrow \quad C_1 = -5C_2
\]

Substitutie in de eerste vergelijking:
\[
-5C_2 + C_2 = 4 \quad \Rightarrow \quad -4C_2 = 4 \quad \Rightarrow \quad C_2 = -1
\]

Daaruit volgt \(C_1 = 5\).

\vspace{0.3cm}
\textbf{Specifieke Systeemrespons:}

\[
\boxed{y(t) = 5e^{-t} - e^{-5t}}
\]

\textbf{Interpretatie:} Dit signaal begint op 4, stijgt initieel lichtjes (omdat de snelheid 0 moet zijn, maar de versnelling negatief is), en vervalt dan exponentieel naar nul. De trage component (\(e^{-t}\)) domineert voor grotere \(t\).
\end{oefenblok}

\subsection{Oefening 2.2: Resonantie en Gedwongen Oscillatie}

\begin{oefenblok}[Probleem]
Beschouw een ongedempt systeem aangedreven door een harmonische kracht:
\[
y'' + 16y = 8\sin 2t
\]
met \(y(0) = 4, y'(0) = 0\).
\end{oefenblok}

\begin{oefenblok}[Oplossing]
\textbf{Stap 1: Karakteristieke Vergelijking}

\(\lambda^2 + 16 = 0\) geeft imaginaire wortels \(\lambda = \pm 4j\).

De homogene oplossing is dus een ongedempte trilling:
\[
y_h(t) = C_1 \cos 4t + C_2 \sin 4t
\]
met een eigenfrequentie van 4 rad/s.

\vspace{0.3cm}
\textbf{Stap 2: Particuliere Oplossing}

De excitatie heeft een frequentie van 2 rad/s, wat verschilt van de eigenfrequentie (geen resonantie).

We stellen voor: \(y_p(t) = A \sin 2t + B \cos 2t\).

Tweemaal differentiëren geeft \(y_p''(t) = -4A \sin 2t - 4B \cos 2t\).

Invullen in de DV:
\[
(-4A \sin 2t - 4B \cos 2t) + 16(A \sin 2t + B \cos 2t) = 8 \sin 2t
\]

Groeperen van termen:
\[
12A \sin 2t + 12B \cos 2t = 8 \sin 2t
\]

Hieruit volgt: \(12B = 0 \Rightarrow B = 0\) en \(12A = 8 \Rightarrow A = 2/3\).

\vspace{0.3cm}
\textbf{Stap 3: Algemene Oplossing en Beginvoorwaarden}

\[
y(t) = C_1 \cos 4t + C_2 \sin 4t + \frac{2}{3} \sin 2t
\]

Uit \(y(0) = 4\): \(C_1 = 4\).

Afgeleide: \(y'(t) = -4C_1 \sin 4t + 4C_2 \cos 4t + \frac{4}{3} \cos 2t\).

Uit \(y'(0) = 0\): \(4C_2 + \frac{4}{3} = 0 \Rightarrow C_2 = -\frac{1}{3}\).

\vspace{0.3cm}
\textbf{Totale Respons:}

\[
\boxed{y(t) = 4 \cos 4t - \frac{1}{3} \sin 4t + \frac{2}{3} \sin 2t}
\]

\textbf{Fysische Interpretatie:} Het systeem vertoont een \concept{zweving} (``beating'') door de interactie tussen de eigenfrequentie (4 rad/s) en de aandrijffrequentie (2 rad/s). Dit fenomeen treedt op bij veel praktische systemen, van geluidversterkingssystemen tot trillingen in bruggen.
\end{oefenblok}

\subsection{Oefening 2.3: Mengtank (Eerste Orde Systeem)}

\begin{oefenblok}[Probleem]
Een tank bevat $V = 100$ liter zuiver water. Op $t=0$ begint er zout water in te stromen met een concentratie $C_{in} = 0.5$ kg/L en een debiet $Q = 10$ L/min. Het mengsel wordt goed geroerd en stroomt met hetzelfde debiet $Q$ weer uit de tank.
Stel de differentiaalvergelijking op voor de hoeveelheid zout $x(t)$ in de tank en los deze op.
\end{oefenblok}

\begin{oefenblok}[Oplossing]
\textbf{Massabalans:} Verandering = Instroom - Uitstroom.
\[
\frac{dx}{dt} = (\text{Instroom massa}) - (\text{Uitstroom massa})
\]

\textbf{Instroom:} $Q \cdot C_{in} = 10 \cdot 0.5 = 5$ kg/min.

\textbf{Uitstroom:} $Q \cdot C_{uit}(t)$. Concentratie in tank is $C_{uit}(t) = \frac{x(t)}{V}$.
Dus uitstroom = $10 \cdot \frac{x(t)}{100} = 0.1 x(t)$.

\textbf{DV:} $\frac{dx}{dt} = 5 - 0.1x(t) \Rightarrow x' + 0.1x = 5$.

\textbf{Oplossing:} Dit is een eerste orde lineaire DV.
Homogeen ($x' + 0.1x = 0$): $x_h(t) = K e^{-0.1t}$.
Particulier (voor constante 5): Stel $x_p = A$. Invullen: $0 + 0.1A = 5 \Rightarrow A = 50$.
Algemeen: $x(t) = 50 + K e^{-0.1t}$.

\textbf{Beginvoorwaarde:} $x(0) = 0$ (zuiver water).
$0 = 50 + K \cdot 1 \Rightarrow K = -50$.

\[
x(t) = 50 (1 - e^{-0.1t})
\]

Na verloop van tijd ($t \to \infty$) nadert de hoeveelheid zout 50 kg (wat logisch is: $100 \text{ L} \cdot 0.5 \text{ kg/L}$). De tijdsconstante is $\tau = 1/0.1 = 10$ minuten.
\end{oefenblok}

\subsection{Oefening 2.4: Opladen van een Condensator}

\begin{oefenblok}[Probleem]
Een RC-kring bestaat uit een weerstand \(R = 1 \, \Omega\) en een condensator \(C = 1 \, \text{F}\) in serie. Op \(t=0\) wordt een gelijkspanning van \(10 \, \text{V}\) aangelegd. De condensator is initieel ongeladen (\(v_c(0) = 0\)).

De differentiaalvergelijking voor de spanning \(v_c(t)\) over de condensator is:
\[
RC \frac{dv_c(t)}{dt} + v_c(t) = V_{in}(t)
\]
Bepaal \(v_c(t)\) voor \(t \ge 0\).
\end{oefenblok}

\begin{oefenblok}[Oplossing]
Met de gegeven waarden wordt de DV:
\[
1 \cdot \frac{dv_c}{dt} + v_c = 10 \quad (\text{voor } t > 0)
\]

\textbf{Stap 1: Homogene Oplossing}
\[
\frac{dv_c}{dt} + v_c = 0 \quad \Rightarrow \quad \lambda + 1 = 0 \quad \Rightarrow \quad \lambda = -1
\]
\[
v_{c,h}(t) = K e^{-t}
\]

\textbf{Stap 2: Particuliere Oplossing}
De input is constant (10). We stellen een constante oplossing voor: \(v_{c,p}(t) = A\).
Invullen:
\[
0 + A = 10 \quad \Rightarrow \quad A = 10
\]

\textbf{Stap 3: Algemene Oplossing}
\[
v_c(t) = K e^{-t} + 10
\]

\textbf{Stap 4: Beginvoorwaarde}
Gegeven \(v_c(0) = 0\):
\[
K e^0 + 10 = 0 \quad \Rightarrow \quad K + 10 = 0 \quad \Rightarrow \quad K = -10
\]

\textbf{Resultaat:}
\[
\boxed{v_c(t) = 10(1 - e^{-t}) \, \text{V}}
\]

\textbf{Interpretatie:} De spanning stijgt exponentieel van 0 V naar de eindwaarde van 10 V. De tijdsconstante is \(\tau = RC = 1\) seconde. Na \(5\tau\) (5 seconden) is de condensator voor meer dan 99\% opgeladen.
\end{oefenblok}

\subsection{Oefening 2.2: Dempingsgedrag van Tweede-Orde Systemen}

\begin{oefenblok}[Probleem]
Los de volgende homogene differentiaalvergelijkingen op met de gegeven beginvoorwaarden. Bepaal het type demping (ondergedempt, kritisch gedempt of overgedempt).
\begin{enumerate}
    \item \(y'' + 6y' + 25y = 0\) met \(y(0)=4\) en \(y'(0)=0\).
    \item \(y'' + 6y' + 9y = 0\) met \(y(0)=2\) en \(y'(0)=0\).
\end{enumerate}
\end{oefenblok}

\begin{oefenblok}[Oplossing]
\textbf{1. Ondergedempt Systeem}
Karakteristieke vergelijking: \(\lambda^2 + 6\lambda + 25 = 0\).
Discriminant \(D = 36 - 100 = -64\).
Wortels: \(\lambda = \frac{-6 \pm j8}{2} = -3 \pm 4j\).
Omdat de wortels complex toegevoegd zijn, is het systeem \textbf{ondergedempt}.

Algemene oplossing:
\[
y(t) = e^{-3t}(A \cos(4t) + B \sin(4t))
\]
Beginvoorwaarden:
\(y(0) = A = 4\).
\(y'(t) = -3e^{-3t}(A \cos 4t + B \sin 4t) + e^{-3t}(-4A \sin 4t + 4B \cos 4t)\).
\(y'(0) = -3A + 4B = 0 \Rightarrow 4B = 3(4) = 12 \Rightarrow B = 3\).

Oplossing:
\[
y(t) = e^{-3t}(4 \cos(4t) + 3 \sin(4t))
\]

\textbf{2. Kritisch Gedempt Systeem}
Karakteristieke vergelijking: \(\lambda^2 + 6\lambda + 9 = 0\).
Discriminant \(D = 36 - 36 = 0\).
Wortels: \(\lambda = -3\) (dubbele wortel).
Het systeem is \textbf{kritisch gedempt}.

Algemene oplossing:
\[
y(t) = (A + Bt)e^{-3t}
\]
Beginvoorwaarden:
\(y(0) = A = 2\).
\(y'(t) = B e^{-3t} - 3(A + Bt)e^{-3t}\).
\(y'(0) = B - 3A = 0 \Rightarrow B = 3(2) = 6\).

Oplossing:
\[
y(t) = (2 + 6t)e^{-3t}
\]
\end{oefenblok}

% ============================================================================
% HOOFDSTUK 3: SIGNALEN EN SYSTEMEN
% ============================================================================

\chapter{Signalen en Systemen: Een Eerste Kennismaking}

\section{Systeemeigenschappen en Classificatie}

In de ingenieurswetenschappen maken we een fundamenteel onderscheid tussen twee entiteiten:
\begin{itemize}
    \item \textbf{Signalen:} Dragers van informatie of energie (bijv. spraak, spanning, beurskoersen).
    \item \textbf{Systemen:} Entiteiten die signalen verwerken, manipuleren of erop reageren (bijv. versterker, filter, de economie).
\end{itemize}

We beschouwen een systeem vaak als een "Black Box": we weten niet noodzakelijk wat er binnenin gebeurt, maar we kunnen het gedrag beschrijven door te kijken naar de relatie tussen input en output.

\begin{theorieblok}[Formele Systeembeschrijving]
Een systeem is een operator \(T\) die een ingangssignaal \(x(t)\) transformeert naar een uitgangssignaal \(y(t)\), genoteerd als:
\[
y(t) = T\{x(t)\}
\]
\end{theorieblok}

Om systemen effectief te analyseren en ontwerpen, classificeren we ze op basis van fundamentele eigenschappen. De heilige graal van de systeemtheorie zijn systemen die zowel \textbf{Lineair} als \textbf{Tijdsinvariant} zijn (LTI-systemen), omdat we voor deze klasse een krachtig en compleet wiskundig raamwerk hebben.

\subsection{Lineariteit}

\begin{definition}
Een systeem is \concept{lineair} als het voldoet aan het superpositiebeginsel. Dit impliceert twee voorwaarden:

\textbf{Additiviteit:} De respons op een som van inputs is de som van de individuele responsen:
\[
T\{x_1 + x_2\} = T\{x_1\} + T\{x_2\}
\]

\textbf{Homogeniteit (Schaling):} Als de input schaalt met een factor \(a\), schaalt de output met dezelfde factor:
\[
T\{ax\} = aT\{x\}
\]
\end{definition}

\wichtig{Waarom is dit belangrijk?} Lineariteit is de sleutel tot "verdeel en heers". Het stelt ons in staat complexe signalen op te breken in eenvoudige componenten (zoals sinussen of impulsen). We berekenen de respons op elke simpele component afzonderlijk (wat makkelijk is) en tellen de resultaten op om de totale respons te vinden. Zonder lineariteit zou Fourieranalyse onmogelijk zijn.

\subsection{Tijdsinvariantie}

\begin{definition}
\concept{Tijdsinvariantie} betekent dat de eigenschappen van het systeem niet veranderen in de tijd. Als een input \(x(t)\) leidt tot output \(y(t)\), dan moet een vertraagde input \(x(t - t_0)\) leiden tot een identiek vertraagde output:
\[
\text{Als } x(t) \to y(t), \text{ dan } x(t-t_0) \to y(t-t_0)
\]
\end{definition}

De vorm van de respons hangt niet af van \emph{wanneer} de input wordt aangeboden, alleen van \emph{wat} de input is. Dit is karakteristiek voor fysische systemen waarvan de parameters (massa, weerstand, etc.) constant blijven tijdens het experiment.

\subsection{Causaliteit}

\begin{definition}
\concept{Causaliteit} is een beperking die stelt dat de output op tijdstip \(t\) alleen afhangt van de input op tijdstippen \(\tau \leq t\). Een systeem kan niet reageren op toekomstige input.
\end{definition}

\important{Fysische Opmerking:} Alle fysisch realiseerbare real-time systemen zijn causaal. Systemen die data achteraf verwerken (zoals beeldverwerking van opgeslagen foto's) hoeven niet causaal te zijn.

\section{Oefeningen: Systeemanalyse}

\subsection{Oefening 3.1: Analyse van Lineariteit}

\begin{oefenblok}[Probleem]
Beschouw het systeem beschreven door de vergelijking:
\[
y(t) = 3x(t) + 2
\]

Op het eerste gezicht lijkt dit een lineaire vergelijking, maar bepaal of dit systeem lineair is in de zin van systeemtheorie.
\end{oefenblok}

\begin{oefenblok}[Oplossing]
\textbf{Test van Homogeniteit:}

Als invoer \(x(t)\) geeft \(y(t) = 3x(t) + 2\).

Neem nu een geschaalde invoer \(x_{\text{nieuw}}(t) = a \cdot x(t)\). De nieuwe output is:
\[
y_{\text{nieuw}}(t) = 3(ax(t)) + 2 = 3ax(t) + 2
\]

Echter, als het systeem lineair zou zijn, verwachten we dat de oorspronkelijke output ook met \(a\) schaalt:
\[
a \cdot y(t) = a(3x(t) + 2) = 3ax(t) + 2a
\]

\textbf{Vergelijking:}
\[
3ax(t) + 2 \neq 3ax(t) + 2a \quad \text{(tenzij } a = 1\text{)}
\]

\boxed{\text{Conclusie:}} De constante offset ``2'' breekt de lineariteit. Dit is een \concept{affien} systeem, geen lineair systeem.

Dit komt veel voor in praktische toepassingen: sensoren hebben vaak een offset, versterkers hebben een gelijkcomponent, enzovoort. Voor lineaire analyse moeten we deze offsets eerst verwijderen.
\end{oefenblok}

\subsection{Oefening 3.2: Analyse van een Vertraagd Feedback Systeem}

\begin{oefenblok}[Probleem]
Beschouw het systeem:
\[
y(t) = x(t) + x(t-1)
\]

Bepaal of dit systeem lineair, tijdsinvariant, causaal en inverteerbaar is.
\end{oefenblok}

\begin{oefenblok}[Oplossing]
\textbf{1. Lineariteit:}

Test superpositie. Voor \(x_{\text{in}} = ax_1 + bx_2\):
\begin{align*}
y_{\text{in}} &= (ax_1(t) + bx_2(t)) + (ax_1(t-1) + bx_2(t-1)) \\
&= a(x_1(t) + x_1(t-1)) + b(x_2(t) + x_2(t-1)) \\
&= ay_1(t) + by_2(t)
\end{align*}

\(\checkmark\) \textbf{Het systeem is lineair.}

\vspace{0.3cm}
\textbf{2. Tijdsinvariantie:}

Als \(x(t)\) verschuift naar \(x(t - t_0)\), wordt de output:
\[
y(t) = x(t - t_0) + x((t-1) - t_0) = x(t-t_0) + x(t - t_0 - 1) = y_{\text{shift}}(t-t_0)
\]

\(\checkmark\) \textbf{Het systeem is tijdsinvariant.}

\vspace{0.3cm}
\textbf{3. Causaliteit:}

De output op \(t\) hangt af van input op \(t\) en \(t-1\) (verleden). Het hangt niet af van \(t+1\).

\(\checkmark\) \textbf{Het systeem is causaal.}

\vspace{0.3cm}
\textbf{4. Invertibiliteit:}

Kunnen we uit \(y(t)\) eenduidig \(x(t)\) terugvinden?

Stel dat \(x(t)\) een constante waarde \(C\) is, dan is \(y(t) = C + C = 2C\).

Stel nu dat \(x(t)\) een alternerende blokgolf is: \((+C, -C, +C, -C, \ldots)\).

Dan kan \(y(t)\) nul zijn op bepaalde tijdstippen.

Verschillende inputs kunnen dezelfde output genereren, dus het systeem is \textbf{niet inverteerbaar} zonder extra informatie.

Dit is typisch voor filters: ze comprimeren informatie.
\end{oefenblok}

\subsection{Oefening 3.3: Modellering van een RC-Kring}

\begin{oefenblok}[Probleem]
Een elektrisch RC-circuit bestaat uit:
\begin{itemize}
    \item Weerstand: \(R = 1000 \, \Omega\)
    \item Capaciteit: \(C = 10 \, \mu\text{F}\)
    \item Ingangsspanning: \(v_{\text{in}}(t) = 5u(t)\) V
\end{itemize}

Bepaal de systeemvergelijking en bereken de uitgangsspanning als functie van tijd.
\end{oefenblok}

\begin{oefenblok}[Oplossing]
\textbf{Stap 1: Kirchhoff's Spanningswet}

\[
v_{\text{in}} = v_R + v_C = iR + v_{\text{uit}}
\]

waarbij de stroomrelatie voor een condensator is:
\[
i = C \frac{dv_{\text{uit}}}{dt}
\]

\vspace{0.3cm}
\textbf{Stap 2: Differentiaalvergelijking}

\[
RC \frac{dv_{\text{uit}}}{dt} + v_{\text{uit}} = v_{\text{in}}
\]

De tijdsconstante is:
\[
\tau = RC = 1000 \times 10^{-5} = 0.01 \text{ s} = 10 \text{ ms}
\]

\vspace{0.3cm}
\textbf{Stap 3: Oplossing voor Staprespons}

Voor een stapinput (\(v_{\text{in}} = 5\) V voor \(t \geq 0\)) is de oplossing voor laden van de condensator:
\[
v_{\text{uit}}(t) = V_{\text{bron}}(1 - e^{-t/\tau})u(t) = 5(1 - e^{-t/0.01})u(t)
\]

\vspace{0.3cm}
\textbf{Stap 4: Numerieke Evaluatie}

Op \(t = 10 \text{ ms}\) (wat gelijk is aan \(\tau\)):
\[
v_{\text{uit}}(0.01) = 5(1 - e^{-1}) \approx 5(1 - 0.368) = 3.16 \text{ V}
\]

\vspace{0.3cm}
\boxed{\text{Conclusie:}} Na één tijdconstante heeft de spanning ongeveer 63% van zijn eindwaarde bereikt. Dit is een universeel kenmerk van eerste-orde systemen en zeer nuttig voor snelle schattingen in praktijk.
\end{oefenblok}

\subsection{Oefening 3.4: Stabiliteit van een Systeem}

\begin{oefenblok}[Probleem]
Beschouw het systeem gedefinieerd door de relatie:
\[
y(t) = t \cdot x(t)
\]
Onderzoek of dit systeem BIBO-stabiel (Bounded Input, Bounded Output) is.
\end{oefenblok}

\begin{oefenblok}[Oplossing]
Een systeem is BIBO-stabiel als voor elke begrensde input \(|x(t)| \le M < \infty\) geldt dat de output ook begrensd is (\(|y(t)| \le N < \infty\)).

Stel we nemen als input de eenheidsstap \(x(t) = u(t)\). Dit is een begrensd signaal, want \(|u(t)| \le 1\) voor alle \(t\).

De output wordt dan:
\[
y(t) = t \cdot u(t)
\]
Dit is de ramp-functie.

Als we kijken naar het gedrag voor \(t \to \infty\):
\[
\lim_{t \to \infty} y(t) = \lim_{t \to \infty} t = \infty
\]

De output groeit onbegrensd, hoewel de input begrensd is.

\boxed{\text{Conclusie:}} Het systeem is \textbf{niet BIBO-stabiel}.

Dit type systeem (een tijdsvermenigvuldiger) fungeert als een versterker waarvan de versterkingsfactor lineair toeneemt met de tijd, wat uiteindelijk altijd tot verzadiging of falen zal leiden in een fysieke implementatie.
\end{oefenblok}

\subsection{Oefening 3.5: Kwadratisch Modulatiesysteem}

\emph{Bron: Geavanceerde systeemanalyse.}

\begin{oefenblok}[Probleem]
Beschouw een systeem waarbij de relatie tussen ingang en uitgang wordt gegeven door:
\[
y(t) = x(t) \cdot \cos(2\pi t) + x^2(t)
\]

Onderzoek de lineariteit en tijdsinvariantie van dit systeem.
\end{oefenblok}

\begin{oefenblok}[Oplossing]
\textbf{Test 1: Lineariteit (Superpositie)}

Voor lineariteit moet gelden:
\[
\mathcal{T}\{ax_1(t) + bx_2(t)\} = a\mathcal{T}\{x_1(t)\} + b\mathcal{T}\{x_2(t)\}
\]

Invullen:
\begin{align*}
\mathcal{T}\{ax_1 + bx_2\} &= (ax_1 + bx_2)\cos(2\pi t) + (ax_1 + bx_2)^2 \\
&= (ax_1 + bx_2)\cos(2\pi t) + a^2x_1^2 + 2abx_1x_2 + b^2x_2^2
\end{align*}

Terwijl:
\begin{align*}
a\mathcal{T}\{x_1\} + b\mathcal{T}\{x_2\} &= a(x_1\cos(2\pi t) + x_1^2) + b(x_2\cos(2\pi t) + x_2^2) \\
&= (ax_1 + bx_2)\cos(2\pi t) + ax_1^2 + bx_2^2
\end{align*}

Deze zijn \textbf{niet gelijk} vanwege de kwadratische term $x^2(t)$!

\boxed{\text{Het systeem is NIET lineair.}}

\vspace{0.3cm}
\textbf{Test 2: Tijdsinvariantie}

Voor tijdsinvariantie: als $x(t) \to y(t)$, dan moet $x(t-t_0) \to y(t-t_0)$.

Voor verschoven input $x(t-t_0)$:
\[
y_{\text{nieuw}}(t) = x(t-t_0)\cos(2\pi t) + x^2(t-t_0)
\]

Terwijl de verschoven oorspronkelijke output:
\[
y(t-t_0) = x(t-t_0)\cos(2\pi(t-t_0)) + x^2(t-t_0)
\]

Deze zijn verschillend vanwege $\cos(2\pi t) \neq \cos(2\pi(t-t_0))$!

\boxed{\text{Het systeem is NIET tijdsinvariant.}}

\textbf{Conclusie:} Dit systeem is noch lineair noch tijdsinvariant. De cosinus-modulatie breekt de tijdsinvariantie, en de kwadratische term breekt de lineariteit. Zulke systemen komen voor in communicatietoepassingen (mixers, modulatoren).
\end{oefenblok}

\subsection{Oefening 3.6: Niet-causaal Integratorsysteem}

\emph{Bron: Causaliteitsanalyse.}

\begin{oefenblok}[Probleem]
Beschouw het systeem gedefinieerd door:
\[
y(t) = \int_{t-1}^{t+1} x(\tau) \, d\tau
\]

Is dit systeem causaal? Is het lineair?
\end{oefenblok}

\begin{oefenblok}[Oplossing]
\textbf{Test 1: Causaliteit}

Een systeem is causaal als de output op tijdstip $t$ alleen afhangt van input op tijdstippen $\tau \le t$.

In dit systeem hangt $y(t)$ af van $x(\tau)$ voor $\tau \in [t-1, t+1]$.

Dit betekent dat de output afhangt van \belangrijk{toekomstige waarden} $x(\tau)$ voor $\tau \in (t, t+1]$.

\boxed{\text{Het systeem is NIET causaal.}}

\textbf{Fysische betekenis:} Dit systeem zou in de toekomst moeten kunnen kijken om de huidige output te berekenen - onmogelijk in real-time toepassingen.

\vspace{0.3cm}
\textbf{Test 2: Lineariteit}

Voor twee signalen $x_1(t)$ en $x_2(t)$:
\begin{align*}
\mathcal{T}\{ax_1 + bx_2\} &= \int_{t-1}^{t+1} (ax_1(\tau) + bx_2(\tau)) d\tau \\
&= a\int_{t-1}^{t+1} x_1(\tau) d\tau + b\int_{t-1}^{t+1} x_2(\tau) d\tau \\
&= a\mathcal{T}\{x_1\} + b\mathcal{T}\{x_2\}
\end{align*}

\boxed{\text{Het systeem IS lineair.}}

\textbf{Toepassing:} Niet-causale systemen kunnen wel gebruikt worden in offline signaalverwerking, zoals bij het nabewerken van opgenomen audio of video.
\end{oefenblok}

\subsection{Oefening 3.7: Classificatie van Systemen}

\begin{oefenblok}[Probleem]
Bepaal voor elk van de volgende systemen of ze lineair, tijdsinvariant en causaal zijn. Motiveer je antwoord.
\begin{itemize}
    \item Systeem A: \(y(t) = x(t) \cos(t)\)
    \item Systeem B: \(y(t) = \int_{-\infty}^{t} x(\tau) d\tau\)
    \item Systeem C: \(y(t) = x(2t)\)
\end{itemize}
\end{oefenblok}

\begin{oefenblok}[Oplossing]
\textbf{Systeem A:}
\begin{itemize}
    \item \textbf{Lineair:} Ja. \(a x_1(t)\cos(t) + b x_2(t)\cos(t) = (a x_1(t) + b x_2(t))\cos(t)\).
    \item \textbf{Tijdsinvariant:} Nee. Een verschuiving in input \(x(t-t_0)\) geeft output \(x(t-t_0)\cos(t)\). De verschoven output zou \(x(t-t_0)\cos(t-t_0)\) moeten zijn. Deze zijn ongelijk (de \(\cos(t)\) term verschuift niet mee met de input).
    \item \textbf{Causaal:} Ja. De output op tijdstip \(t\) hangt enkel af van de input op datzelfde moment \(t\).
\end{itemize}

\textbf{Systeem B (Integrator):}
\begin{itemize}
    \item \textbf{Lineair:} Ja, integratie is een lineaire operatie.
    \item \textbf{Tijdsinvariant:} Ja. De integraal van een verschoven functie is de verschoven integraal.
    \item \textbf{Causaal:} Ja. De integraal loopt tot \(t\), dus er is geen kennis van de toekomst (\(t'>t\)) nodig.
\end{itemize}

\textbf{Systeem C (Tijdsschaling):}
\begin{itemize}
    \item \textbf{Lineair:} Ja.
    \item \textbf{Tijdsinvariant:} Nee. Als \(x(t) \to x(t-t_0)\), dan is de output \(y(t) = x(2t - t_0)\). Als we de oorspronkelijke output \(x(2t)\) verschuiven met \(t_0\), krijgen we \(x(2(t-t_0)) = x(2t - 2t_0)\). Deze zijn niet gelijk.
    \item \textbf{Causaal:} Nee. Stel \(t=1\), dan is \(y(1) = x(2)\). De output op \(t=1\) hangt af van de input in de toekomst (\(t=2\)). Dit is een niet-causaal systeem (voorspellend).
\end{itemize}

\begin{center}
\begin{tikzpicture}[scale=0.8]
    \begin{axis}[
        width=6cm, height=3cm,
        axis lines=middle,
        xlabel={$t$}, ylabel={$x(t)$},
        domain=-1:3, samples=50,
        ymin=0, ymax=1.5,
        title={Input}
    ]
    \addplot[thick, blue] coordinates {(-1,0) (0,0) (0,1) (2,1) (2,0) (3,0)};
    \end{axis}
\end{tikzpicture}
\hspace{0.5cm}
\begin{tikzpicture}[scale=0.8]
    \begin{axis}[
        width=6cm, height=3cm,
        axis lines=middle,
        xlabel={$t$}, ylabel={$y(t)=x(2t)$},
        domain=-1:3, samples=50,
        ymin=0, ymax=1.5,
        title={Output (versneld)}
    ]
    \addplot[thick, red] coordinates {(-1,0) (0,0) (0,1) (1,1) (1,0) (3,0)};
    \end{axis}
\end{tikzpicture}
\end{center}
\end{oefenblok}

\subsection{Oefening 3.8: Opladen van een Condensator}

\begin{oefenblok}[Probleem]
Een condensator van \(1 \, \mu\text{F}\) wordt opgeladen over een weerstand van \(200 \, \Omega\).
\begin{enumerate}
    \item Hoeveel bedraagt de maximale lading indien een constante spanning van \(5 \, \text{mV}\) wordt aangelegd op \(t=0\)?
    \item Na hoeveel tijd is de condensator voor 95\% opgeladen?
    \item Op het moment dat de spanning wordt ingeschakeld is de condensator leeg.
\end{enumerate}
\end{oefenblok}

\begin{oefenblok}[Oplossing]
\textbf{1. Maximale Lading}
De maximale lading \(Q_{\text{max}}\) wordt bereikt als de condensator volledig is opgeladen en de spanning erover gelijk is aan de bronspanning \(V\).
\[
Q_{\text{max}} = C \cdot V = 1 \cdot 10^{-6} \, \text{F} \cdot 5 \cdot 10^{-3} \, \text{V} = 5 \cdot 10^{-9} \, \text{C} = 5 \, \text{nC}
\]

\textbf{2. Oplaadtijd}
De tijdsconstante van het RC-circuit is:
\[
\tau = R \cdot C = 200 \, \Omega \cdot 1 \cdot 10^{-6} \, \text{F} = 200 \cdot 10^{-6} \, \text{s} = 0.2 \, \text{ms}
\]
De lading in functie van de tijd wordt gegeven door:
\[
q(t) = Q_{\text{max}} (1 - e^{-t/\tau})
\]
We zoeken \(t\) waarvoor \(q(t) = 0.95 Q_{\text{max}}\):
\[
0.95 = 1 - e^{-t/\tau} \Rightarrow e^{-t/\tau} = 0.05
\]
\[
-t/\tau = \ln(0.05) \approx -2.996
\]
\[
t \approx 3\tau = 3 \cdot 0.2 \, \text{ms} = 0.6 \, \text{ms}
\]

\begin{center}
\begin{tikzpicture}[scale=0.8]
    \begin{axis}[
        width=8cm, height=5cm,
        axis lines=middle,
        xlabel={$t$ (ms)}, ylabel={$q(t)$ (nC)},
        domain=0:1, samples=100,
        ymin=0, ymax=5.5,
        grid=both
    ]
    \addplot[thick, blue] {5 * (1 - exp(-x/0.2))};
    \draw[dashed] (axis cs:0, 5) -- (axis cs:1, 5);
    \node at (axis cs: 0.8, 5.2) {$Q_{\text{max}}$};
    \end{axis}
\end{tikzpicture}
\end{center}
\end{oefenblok}

% ============================================================================
% HOOFDSTUK 4: BASISSIGNALEN EN BEWERKINGEN
% ============================================================================

\chapter{Basissignalen en Bewerkingen}

\section{De Bouwstenen van Signaalanalyse}

In de praktijk zijn signalen vaak rommelig en complex. Om ze wiskundig te analyseren, benaderen we ze met geïdealiseerde modellen. Complexe signalen kunnen wiskundig worden ontleed in (of opgebouwd uit) elementaire basissignalen. De correcte manipulatie van deze signalen is essentieel voor verdere analyse.

Er zijn twee belangrijke klassen van signalen die we onderscheiden op basis van hun "grootte":
\begin{itemize}
    \item \textbf{Energiesignalen:} Signalen met een eindige totale energie (bijv. een puls). Deze sterven uit na verloop van tijd.
    \item \textbf{Vermogenssignalen:} Signalen met een oneindige energie maar een eindig gemiddeld vermogen (bijv. een periodieke sinus). Deze blijven eeuwig voortduren.
\end{itemize}

Hieronder bespreken we de belangrijkste bouwstenen.

\subsection{De Eenheidsstap \texorpdfstring{\(u(t)\)}{u(t)}}

\begin{definition}
De eenheidsstap (Heaviside-functie) wordt gedefinieerd als:
\[
u(t) = \begin{cases} 1 & \text{als } t > 0 \\ 0 & \text{als } t < 0 \end{cases}
\]

Dit signaal fungeert als een wiskundige schakelaar. Vermenigvuldiging van een signaal \(f(t)\) met \(u(t-t_0)\) zet het signaal "aan" op tijdstip \(t_0\). Dit is cruciaal voor het modelleren van \concept{causale} systemen, die pas reageren nadat ze zijn ingeschakeld.
\end{definition}

\subsection{De Impulsfunctie \texorpdfstring{\(\delta(t)\)}{delta(t)}}

\begin{definition}
De Dirac-impuls is een theoretische constructie (geen echte functie) met:
\begin{itemize}
    \item Oneindige hoogte
    \item Nul breedte
    \item Totale oppervlakte (integraal) exact gelijk aan 1
\end{itemize}

Fysisch modelleert dit een zeer korte, hevige gebeurtenis, zoals een hamerslag of een blikseminslag, die een eindige hoeveelheid energie overdraagt in een infinitesimaal korte tijd.

De cruciale eigenschap is de \concept{zeefeigenschap}:
\[
\int_{-\infty}^{\infty} x(t) \delta(t - t_0) dt = x(t_0)
\]
\end{definition}

Dit modelleert een onmiddellijke schok (zoals een hamerklap op een trillend systeem). Hoewel niet fysisch realiseerbaar, is het wiskundig zeer krachtig.

\subsection{De Exponentiële Functie \texorpdfstring{\(e^{st}\)}{e\textasciicircum st}}

\begin{definition}
Dit is de fundamentale eigenfunctie van LTI-systemen. Dit betekent dat als je een exponentieel signaal \(e^{st}\) door een lineair systeem stuurt, het uitgangssignaal ook exponentieel is (mogelijk met een ander amplitude en fase).
\end{definition}

Deze eigenschap is de basis voor Laplace- en Fourier-transformaties.

\section{Convolutie: De Kernoperatie}

\begin{definition}
Een cruciale bewerking tussen signalen is de \concept{convolutie}:
\[
(f * g)(t) = \int_{-\infty}^{\infty} f(\tau) g(t - \tau) d\tau
\]

Deze integraal beschrijft hoe een systeem met impulsrespons \(g(t)\) reageert op een input \(f(t)\).
\end{definition}

\wichtig{Geometrische Interpretatie:} Convolutie kan worden gezien als een ``gewogen voortschrijdend gemiddelde'', waarbij het ene signaal wordt omgekeerd, verschoven en vermenigvuldigd met het andere.

\section{Oefeningen: Convolutie en Transformatie}

\subsection{Oefening 4.1: Convolutie van Blokpulsen}

\begin{oefenblok}[Probleem]
Bereken de convolutie van twee identieke blokpulsen:
\[
f(t) = u(t) - u(t-1), \quad g(t) = u(t) - u(t-1)
\]

Beide zijn 1 op het interval \([0, 1]\) en 0 elders.
\end{oefenblok}

\begin{oefenblok}[Oplossing]
De convolutie-integraal \(y(t) = \int f(\tau) g(t - \tau) d\tau\) meet de overlap tussen de blokpuls \(f\) en de verschoven blokpuls \(g\).

\textbf{Geval 1: \(t < 0\)}
Geen overlap. \(y(t) = 0\).

\vspace{0.3cm}
\textbf{Geval 2: \(0 \le t < 1\) (Intrede)}

De pulsen schuiven in elkaar. De overlapduur is \(t\).
\[
y(t) = \int_0^t 1 \cdot 1 \, d\tau = t
\]

\vspace{0.3cm}
\textbf{Geval 3: \(1 \le t < 2\) (Volledige overlap)}

De pulsen schuiven uit elkaar. De overlappingsduur is \(2 - t\).
\[
y(t) = \int_{t-1}^1 1 \cdot 1 \, d\tau = 1 - (t-1) = 2 - t
\]

\vspace{0.3cm}
\textbf{Geval 4: \(t \ge 2\)}

Geen overlap meer. \(y(t) = 0\).

\vspace{0.3cm}
\boxed{\text{Resultaat:}}
\[
y(t) = \begin{cases}
0 & t < 0 \\
t & 0 \le t < 1 \\
2 - t & 1 \le t < 2 \\
0 & t \ge 2
\end{cases}
\]

\textbf{Fascinerende Opmerking:} Dit resultaat is een driehoekpuls! Dit is een fundamenteel resultaat: convolutie van twee rechthoeken geeft een driehoek; convolutie van meerdere rechthoeken benadert een Gauss-curve (centrale limietstelling).
\end{oefenblok}

\subsection{Oefening 4.2: Signaalmanipulatie}

\begin{oefenblok}[Probleem]
Gegeven \(x(t) = e^{-t} u(t)\). Bepaal:
\begin{enumerate}
    \item Verschuiving: \(y_1(t) = x(t-2)\)
    \item Schaling (Compressie): \(y_2(t) = x(2t)\)
\end{enumerate}
\end{oefenblok}

\begin{oefenblok}[Oplossing]
\textbf{1. Verschuiving:}

\(y_1(t) = x(t-2) = e^{-(t-2)} u(t-2) = e^{2} \cdot e^{-t} u(t-2)\)

Het signaal begint nu pas op \(t = 2\) en vervalt vanaf daar. De vorm is identiek, enkel verplaatst met 2 eenheden naar rechts.

\vspace{0.3cm}
\textbf{2. Schaling (Compressie):}

\(y_2(t) = x(2t) = e^{-2t} u(2t)\)

Aangezien \(u(2t) = u(t)\) (het teken van het argument verandert niet), is dit:
\[
y_2(t) = e^{-2t} u(t)
\]

Het signaal vervalt twee keer zo snel; de tijdsconstante is gehalveerd van 1 naar 0.5.

Dit illustreert een belangrijk principe: compressie in de tijd leidt tot versneld verval en hoger-frequente inhoud.
\end{oefenblok}

\subsection{Oefening 4.3: Convolutie van Exponentiëlen}

\emph{Bron: Analytische convolutie technieken.}

\begin{oefenblok}[Probleem]
Gegeven twee causale signalen:
\[
x(t) = e^{-3t} u(t) \quad \text{en} \quad h(t) = e^{-2t} u(t)
\]

Bepaal de convolutie $y(t) = (x * h)(t)$ analytisch (zonder Laplace).
\end{oefenblok}

\begin{oefenblok}[Oplossing]
De convolutie-integraal:
\[
y(t) = \int_{-\infty}^{\infty} x(\tau) h(t - \tau) d\tau
\]

Voor causale signalen ($u(t)$ zorgt dat beide functies 0 zijn voor negatieve argumenten):
\[
y(t) = \int_{0}^{t} e^{-3\tau} e^{-2(t-\tau)} d\tau \quad \text{voor } t \ge 0
\]

\vspace{0.3cm}
\textbf{Stap 1: Vereenvoudig de integraal}

\[
y(t) = \int_{0}^{t} e^{-3\tau} e^{-2t + 2\tau} d\tau = e^{-2t} \int_{0}^{t} e^{-3\tau + 2\tau} d\tau
\]
\[
= e^{-2t} \int_{0}^{t} e^{-\tau} d\tau
\]

\vspace{0.3cm}
\textbf{Stap 2: Integreer}

\[
= e^{-2t} \left[-e^{-\tau}\right]_0^t = e^{-2t}(1 - e^{-t})
\]

\vspace{0.3cm}
\textbf{Stap 3: Herschrijf}

\[
y(t) = (e^{-2t} - e^{-3t})u(t)
\]

\boxed{y(t) = (e^{-2t} - e^{-3t})u(t)}

\textbf{Fysische interpretatie:} Het resultaat is het verschil van twee exponentiële vervallen. Het signaal start bij 0, bereikt een maximum, en keert terug naar 0. Dit gedrag is typisch voor convolutie van twee vervallende exponenten.
\end{oefenblok}

\subsection{Oefening 4.4: Grafische Convolutie (Trapezium)}

\begin{oefenblok}[Probleem]
Bereken en teken de convolutie $y(t) = x(t) * h(t)$ van twee rechthoekige pulsen met verschillende breedtes:
$x(t) = u(t) - u(t-2)$ (Puls van 0 tot 2, hoogte 1)
$h(t) = u(t) - u(t-1)$ (Puls van 0 tot 1, hoogte 1)
\end{oefenblok}

\begin{oefenblok}[Oplossing]
\textbf{Stapsgewijs:}
We schuiven $h(t-\tau)$ over $x(\tau)$.

\textbf{$t < 0$:} Geen overlap. $y(t) = 0$.

\textbf{$0 \le t < 1$ (Inschuiven):}
De puls $h$ schuift de puls $x$ binnen. De overlap is gelijk aan de breedte van het deel van $h$ dat binnen is: breedte $= t$.
Integraal = $1 \cdot t = t$. (Lineaire stijging).

\textbf{$1 \le t < 2$ (Volledige overlap):}
De smalle puls $h$ (breedte 1) bevindt zich volledig binnen de brede puls $x$ (breedte 2).
De overlap is maximaal en constant: breedte $= 1$.
Integraal = $1 \cdot 1 = 1$. (Constant plateau).

\textbf{$2 \le t < 3$ (Uitschuiven):}
De puls $h$ verlaat $x$. De achterkant van $h$ is op $t-1$, de achterkant van $x$ is op $2$.
De overlap is van $t-1$ tot $2$. Breedte $= 2 - (t-1) = 3 - t$.
Integraal = $3 - t$. (Lineaire daling).

\textbf{$t \ge 3$:} Geen overlap meer. $y(t) = 0$.

\textbf{Resultaat:} Een trapeziumvormig signaal dat stijgt van 0 tot 1, constant blijft tot 2, en daalt tot 0 bij $t=3$.
\end{oefenblok}

\subsection{Oefening 4.5: Convolutie met een Impuls}

\begin{oefenblok}[Probleem]
Bereken de convolutie van een willekeurig signaal \(x(t)\) met een verschoven Dirac-impuls \(\delta(t - t_0)\):
\[
y(t) = x(t) * \delta(t - t_0)
\]
\end{oefenblok}

\begin{oefenblok}[Oplossing]
We gebruiken de definitie van convolutie:
\[
y(t) = \int_{-\infty}^{\infty} x(\tau) \delta((t - t_0) - \tau) d\tau
\]
Merk op dat \(\delta((t - t_0) - \tau) = \delta(t - t_0 - \tau)\).

De Dirac-impuls is nul overal, behalve waar het argument nul is.
Het argument is nul als:
\[
t - t_0 - \tau = 0 \quad \Rightarrow \quad \tau = t - t_0
\]

Volgens de zeefeigenschap (sifting property) van de integraal:
\[
\int_{-\infty}^{\infty} f(\tau) \delta(\tau - a) d\tau = f(a)
\]
In ons geval is de impuls verschoven naar \(\tau = t - t_0\).

Dus de integraal selecteert de waarde van \(x(\tau)\) op \(\tau = t - t_0\):
\[
y(t) = x(t - t_0)
\]

\boxed{\text{Conclusie:}} Convolutie met een verschoven impuls \(\delta(t - t_0)\) verschuift het oorspronkelijke signaal over een tijd \(t_0\). Dit bevestigt dat de impulsrespons van een ideaal vertragingssysteem een verschoven Dirac-impuls is.
\end{oefenblok}

\subsection{Oefening 4.6: Signaal Schetsen}

\begin{oefenblok}[Probleem]
Maak een schets van het volgende signaal voor \(t\) in \([0, 2\pi]\):
\[
x(t) = 2 - \cos(4t - \pi/2)
\]
Bepaal ook de periode.
\end{oefenblok}

\begin{oefenblok}[Oplossing]
\textbf{Analyse:}
\begin{itemize}
    \item \textbf{DC-offset:} Het signaal is verticaal verschoven met \(+2\).
    \item \textbf{Frequentie:} De hoekfrequentie is \(\omega = 4\). De periode is \(T = 2\pi / \omega = 2\pi / 4 = \pi/2\).
    \item \textbf{Faseverschuiving:} \(\cos(4t - \pi/2) = \cos(4(t - \pi/8))\). Dit is een verschuiving naar rechts met \(\pi/8\).
    \item \textbf{Alternatieve vorm:} We weten dat \(\cos(\theta - \pi/2) = \sin(\theta)\). Dus \(x(t) = 2 - \sin(4t)\).
\end{itemize}

\textbf{Schets:}
Het is een sinusgolf die geïnverteerd is (min-teken) en omhoog geschoven met 2.
Het signaal oscilleert tussen \(2-1=1\) en \(2+1=3\).

\begin{center}
\begin{tikzpicture}[scale=0.8]
    \begin{axis}[
        width=10cm, height=4cm,
        axis lines=middle,
        xlabel={$t$}, ylabel={$x(t)$},
        domain=0:6.28, samples=200,
        ymin=0, ymax=4,
        xtick={0, 1.57, 3.14, 4.71, 6.28},
        xticklabels={0, $\pi/2$, $\pi$, $3\pi/2$, $2\pi$},
        grid=both
    ]
    \addplot[thick, blue] {2 - cos(deg(4*x) - 90)};
    \end{axis}
\end{tikzpicture}
\end{center}
\end{oefenblok}

\subsection{Oefening 4.7: Convolutie van Gelijke Exponenten}

\begin{oefenblok}[Probleem]
Bepaal de convolutie van twee gelijke transiënte signalen:
\[
f_1(t) = f_2(t) = e^{-at}u(t)
\]
\end{oefenblok}

\begin{oefenblok}[Oplossing]
\[
y(t) = f_1(t) * f_2(t) = \int_{-\infty}^{\infty} e^{-a\tau}u(\tau) \cdot e^{-a(t-\tau)}u(t-\tau) d\tau
\]
De stapfuncties beperken het integratie-interval:
\begin{itemize}
    \item \(u(\tau)\) vereist \(\tau \ge 0\).
    \item \(u(t-\tau)\) vereist \(t-\tau \ge 0 \Rightarrow \tau \le t\).
\end{itemize}
Dus voor \(t < 0\) is de integraal 0. Voor \(t \ge 0\) integreren we van 0 tot \(t\).

\[
y(t) = \int_{0}^{t} e^{-a\tau} e^{-at + a\tau} d\tau = \int_{0}^{t} e^{-at} (e^{-a\tau} e^{a\tau}) d\tau
\]
\[
= e^{-at} \int_{0}^{t} 1 d\tau = e^{-at} [\tau]_0^t = t e^{-at}
\]

\boxed{y(t) = t e^{-at} u(t)}
\end{oefenblok}

\subsection{Oefening 4.8: Convolutie van Stap en Rechthoek}

\begin{oefenblok}[Probleem]
Bereken de convolutie \(z(t) = x(t) * y(t)\) voor de signalen:
\[
x(t) = u(t) \quad \text{en} \quad y(t) = \text{rect}\left(\frac{t-2}{2}\right)
\]
\end{oefenblok}

\begin{oefenblok}[Oplossing]
\textbf{Analyse van de signalen:}
\begin{itemize}
    \item \(x(t) = u(t)\) is de eenheidsstap (start op \(t=0\)).
    \item \(y(t) = \text{rect}\left(\frac{t-2}{2}\right)\) is een rechthoekpuls.
    \begin{itemize}
        \item Het argument \(\frac{t-2}{2}\) ligt tussen -0.5 en 0.5.
        \item Dus \(-0.5 \le \frac{t-2}{2} \le 0.5 \Rightarrow -1 \le t-2 \le 1 \Rightarrow 1 \le t \le 3\).
        \item De puls is 1 op het interval \([1, 3]\) en 0 elders.
    \end{itemize}
\end{itemize}

\textbf{Berekening:}
We gebruiken de integraal \(z(t) = \int_{-\infty}^{\infty} x(\tau) y(t-\tau) d\tau\).
Omdat \(x(\tau) = u(\tau)\), integreren we enkel voor \(\tau \ge 0\).
Het is vaak makkelijker om de eindige puls \(y\) vast te houden en \(x\) te verschuiven, of omgekeerd. Laten we hier de definitie volgen waarbij we \(y\) verschuiven:
Maar nog eenvoudiger is de eigenschap van convolutie met een stapfunctie:
\[
x(t) * u(t) = \int_{-\infty}^t x(\tau) d\tau
\]
Dus we moeten gewoon de integraal van \(y(t)\) berekenen van \(-\infty\) tot \(t\).

\[
z(t) = \int_{-\infty}^t y(\tau) d\tau = \int_{-\infty}^t \text{rect}\left(\frac{\tau-2}{2}\right) d\tau
\]

\textbf{Geval 1: \(t < 1\)}
De integraal loopt tot een punt voor de puls begint.
\[
z(t) = 0
\]

\textbf{Geval 2: \(1 \le t < 3\)}
We integreren over het begin van de puls tot \(t\).
\[
z(t) = \int_{1}^t 1 d\tau = [ \tau ]_1^t = t - 1
\]

\textbf{Geval 3: \(t \ge 3\)}
We hebben de volledige puls geïntegreerd.
\[
z(t) = \int_{1}^3 1 d\tau = 3 - 1 = 2
\]

\boxed{\text{Resultaat:}}
\[
z(t) = \begin{cases} 
0 & t < 1 \\
t - 1 & 1 \le t < 3 \\
2 & t \ge 3
\end{cases}
\]
Dit is een signaal dat lineair stijgt en daarna constant blijft (verzadiging).
\end{oefenblok}

\subsection{Oefening 4.9: Convolutie van Exponentieel en Rechthoek}

\begin{oefenblok}[Probleem]
Bereken de convolutie \(z(t) = x(t) * y(t)\) voor:
\[
x(t) = 3e^{-t}u(t) \quad \text{en} \quad y(t) = \text{rect}(t - 0.5)
\]
\end{oefenblok}

\begin{oefenblok}[Oplossing]
\textbf{Analyse:}
\begin{itemize}
    \item \(x(t)\) is een exponentieel verval startend op \(t=0\).
    \item \(y(t)\) is een rechthoek gecentreerd op 0.5 met breedte 1. Dus van \(t=0\) tot \(t=1\).
\end{itemize}

\textbf{Berekening:}
\[
z(t) = \int_{-\infty}^{\infty} x(\tau) y(t-\tau) d\tau
\]
De puls \(y(t-\tau)\) is niet-nul als \(0 \le t-\tau \le 1\), oftewel \(t-1 \le \tau \le t\).
Tevens moet \(\tau \ge 0\) zijn vanwege \(u(\tau)\) in \(x(\tau)\).

\textbf{Geval 1: \(t < 0\)}
Geen overlap. \(z(t) = 0\).

\textbf{Geval 2: \(0 \le t < 1\) (Gedeeltelijke overlap)}
Het integratie-interval wordt beperkt door \(\tau \ge 0\) (start van \(x\)) en \(\tau \le t\) (einde van verschoven \(y\)).
\[
z(t) = \int_0^t 3e^{-\tau} \cdot 1 d\tau = 3 [-e^{-\tau}]_0^t = 3(1 - e^{-t})
\]

\textbf{Geval 3: \(t \ge 1\) (Volledige overlap)}
Het integratie-interval is nu volledig bepaald door de breedte van de puls \(y\): van \(\tau = t-1\) tot \(\tau = t\).
\[
z(t) = \int_{t-1}^t 3e^{-\tau} d\tau = 3 [-e^{-\tau}]_{t-1}^t
\]
\[
z(t) = 3 (-e^{-t} - (-e^{-(t-1)})) = 3 (e^{-(t-1)} - e^{-t})
\]
\[
z(t) = 3 e^{-t} (e^1 - 1)
\]

\boxed{\text{Resultaat:}}
\[
z(t) = \begin{cases} 
0 & t < 0 \\
3(1 - e^{-t}) & 0 \le t < 1 \\
3(e - 1)e^{-t} & t \ge 1
\end{cases}
\]
\end{oefenblok}

\subsection{Oefening 4.10: Convolutie van Helling en Stap}

\begin{oefenblok}[Probleem]
Bereken de convolutie \(y(t) = r(t) * u(t)\), waarbij \(r(t) = t u(t)\) de hellingfunctie (ramp function) is en \(u(t)\) de eenheidsstap.
\end{oefenblok}

\begin{oefenblok}[Oplossing]
We gebruiken de definitie:
\[
y(t) = \int_{-\infty}^{\infty} r(\tau) u(t - \tau) d\tau
\]

Invullen van de functies:
\[
y(t) = \int_{-\infty}^{\infty} \tau u(\tau) u(t - \tau) d\tau
\]

De stapfuncties bepalen de integratiegrenzen:
\begin{itemize}
    \item \(u(\tau)\) is 1 als \(\tau \ge 0\).
    \item \(u(t - \tau)\) is 1 als \(t - \tau \ge 0 \Rightarrow \tau \le t\).
\end{itemize}

Dus de integrand is \(\tau\) op het interval \([0, t]\). Als \(t < 0\), is er geen overlap en is de integraal 0.

Voor \(t \ge 0\):
\[
y(t) = \int_{0}^{t} \tau d\tau = \left[ \frac{1}{2}\tau^2 \right]_0^t = \frac{1}{2}t^2
\]

\boxed{\text{Resultaat:}} \(y(t) = \frac{1}{2}t^2 u(t)\).

\textbf{Inzicht:} Convolutie met een stapfunctie \(u(t)\) is equivalent aan integratie. De integraal van een helling \(t\) is een parabool \(t^2/2\).
\end{oefenblok}

\subsection{Oefening 4.11: Energie en Vermogen}

\begin{oefenblok}[Probleem]
Bereken de totale energie \(E\) of het gemiddeld vermogen \(P\) van de volgende signalen en classificeer ze als energiesignaal of vermogenssignaal.
\begin{itemize}
    \item \(x_1(t) = 5 e^{-2t} u(t)\)
    \item \(x_2(t) = 4 \cos(3t)\)
\end{itemize}

\begin{center}
\begin{tikzpicture}[scale=0.8]
    \begin{axis}[
        width=6cm, height=4cm,
        axis lines=middle,
        xlabel={$t$}, ylabel={$x_1(t)$},
        domain=-1:2, samples=100,
        ymin=-1, ymax=6,
        grid=both
    ]
    \addplot[thick, blue] {5*exp(-2*x) * (x>=0)};
    \end{axis}
\end{tikzpicture}
\hspace{0.5cm}
\begin{tikzpicture}[scale=0.8]
    \begin{axis}[
        width=6cm, height=4cm,
        axis lines=middle,
        xlabel={$t$}, ylabel={$x_2(t)$},
        domain=0:4, samples=100,
        ymin=-5, ymax=5,
        grid=both
    ]
    \addplot[thick, red] {4*cos(deg(3*x))};
    \end{axis}
\end{tikzpicture}
\end{center}
\end{oefenblok}

\begin{oefenblok}[Oplossing]
\textbf{Energiesignaal:} Het signaal sterft uit naar oneindig.
\[
E = \int_{-\infty}^{\infty} |x_1(t)|^2 dt = \int_{0}^{\infty} (5 e^{-2t})^2 dt = 25 \int_{0}^{\infty} e^{-4t} dt
\]
\[
E = 25 \left[ \frac{-1}{4} e^{-4t} \right]_0^{\infty} = 25 (0 - (-0.25)) = 6.25 \text{ Joule}
\]

\textbf{Vermogenssignaal:} Het signaal is periodiek en duurt oneindig lang voort.
\[
P = \frac{1}{T} \int_0^T |x_2(t)|^2 dt
\]
Voor een sinusoïde \(A \cos(\omega t)\) is het gemiddeld vermogen \(P = A^2 / 2\).
Hier is \(A=4\), dus \(P = 4^2 / 2 = 16 / 2 = 8 \text{ Watt}\).
\end{oefenblok}

% ============================================================================
% HOOFDSTUK 5: LAPLACE TRANSFORMATIE (VERKORTE VERSIE)
% ============================================================================

\chapter{De Laplacetransformatie}

\section{Van Tijdsdomein naar s-Domein}

Het oplossen van differentiaalvergelijkingen in het tijdsdomein kan complex en rekenintensief zijn, vooral bij hogere ordes. De Laplacetransformatie biedt een elegante uitweg door het probleem te verplaatsen naar een ander domein: het \concept{s-domein} of frequentiedomein.

De filosofie achter deze transformatie is analoog aan het gebruik van logaritmen om vermenigvuldiging om te zetten in optelling. Laplace zet differentiatie (calculus) om in vermenigvuldiging (algebra).

\begin{theorieblok}[De Kracht van Laplace]
De Laplacetransformatie converteert lineaire differentiaalvergelijkingen (moeilijk) naar algebraïsche vergelijkingen (makkelijk).
\begin{itemize}
    \item \textbf{Tijdsdomein:} \( \frac{d}{dt} y(t) \) (Differentiatie)
    \item \textbf{s-Domein:} \( s Y(s) \) (Vermenigvuldiging)
\end{itemize}
Bovendien wordt convolutie in de tijd een simpele vermenigvuldiging in het s-domein: \( y(t) = x(t) * h(t) \Rightarrow Y(s) = X(s)H(s) \).
\end{theorieblok}

De variabele \(s\) is een \concept{complexe frequentie}: \(s = \sigma + j\omega\).
\begin{itemize}
    \item Het reële deel \(\sigma\) beschrijft exponentiële groei of demping.
    \item Het imaginaire deel \(\omega\) beschrijft oscillatie (sinusvormig gedrag).
\end{itemize}
Hiermee kunnen we signalen analyseren die niet alleen oscilleren, maar ook opblazen of uitsterven.

\begin{definition}
De eenzijdige Laplacetransformatie wordt gedefinieerd als:
\[
F(s) = \mathcal{L}\{f(t)\} = \int_0^\infty f(t) e^{-st} dt
\]
\end{definition}

Deze transformatie houdt expliciet rekening met beginvoorwaarden, wat het superieur maakt aan Fourier voor de analyse van transiënte verschijnselen.

\subsection{Belangrijkste Eigenschappen}

\begin{theorem}[Differentiatie]
\[
\mathcal{L}\{f'(t)\} = sF(s) - f(0)
\]
Elke afgeleide vermenigvuldigt de transformatie met \(s\), minus de beginwaarde. Dit mechanisme reduceert calculus tot algebra.
\end{theorem}

\begin{theorem}[Verschuiving in s-domein]
Vermenigvuldiging met \(e^{-at}\) in de tijd komt overeen met verschuiven in frequentie: \(F(s + a)\). Dit verklaart waarom gedempte sinussen \(e^{-at}\cos\omega t\) polen hebben op \(-a \pm j\omega\).
\end{theorem}

\begin{theorem}[Verschuiving in tijdsdomein]
Vertraging met \(T\) in de tijd komt overeen met vermenigvuldiging met \(e^{-sT}\):
\[
\mathcal{L}\{f(t-T)u(t-T)\} = e^{-sT}F(s)
\]
\end{theorem}

\section{Oefeningen: Toepassing van Laplace}

\subsection{Oefening 5.1: Transformatie van Basissignalen}

\begin{oefenblok}[Probleem]
Bepaal de Laplacetransformatie van \(f(t) = e^{-3t} u(t)\).
\end{oefenblok}

\begin{oefenblok}[Oplossing]
Volgens de definitie:
\[
\mathcal{L}\{e^{-3t} u(t)\} = \int_0^\infty e^{-3t} e^{-st} dt = \int_0^\infty e^{-(s+3)t} dt
\]

\[
= \left[ \frac{-1}{s+3} e^{-(s+3)t} \right]_0^\infty
\]

Als \(\text{Re}(s) > -3\), is de limiet op oneindig nul. Er rest:
\[
\boxed{\mathcal{L}\{e^{-3t} u(t)\} = \frac{1}{s+3}}
\]

\textbf{Belangrijk patroon:} Polen op \(s = -3\) corresponderen met exponentieel verval met snelheid 3.
\end{oefenblok}

\subsection{Oefening 5.2: Frequentieverschuiving}

\begin{oefenblok}[Probleem]
Bepaal de Laplacetransformatie van \(f(t) = t^2 e^{-3t} u(t)\).
\end{oefenblok}

\begin{oefenblok}[Oplossing]
We gebruiken de eigenschap van frequentieverschuiving:
\[
\mathcal{L}\{e^{-at} g(t)\} = G(s+a)
\]
Hier is \(g(t) = t^2\) en \(a = 3\).

De transformatie van \(t^n\) is \(\frac{n!}{s^{n+1}}\).
Voor \(t^2\) is dit:
\[
G(s) = \frac{2!}{s^{2+1}} = \frac{2}{s^3}
\]

Vervang nu \(s\) door \(s+3\):
\[
F(s) = G(s+3) = \frac{2}{(s+3)^3}
\]

\boxed{F(s) = \frac{2}{(s+3)^3}}
\end{oefenblok}

\subsection{Oefening 5.3: Oplossen van een Eerste-Orde DV}

\begin{oefenblok}[Probleem]
Los op:
\[
y' + 4y = 8u(t), \quad y(0) = 2
\]
\end{oefenblok}

\begin{oefenblok}[Oplossing]
\textbf{Stap 1: Transform naar s-domein}

\[
(sY(s) - y(0)) + 4Y(s) = \frac{8}{s}
\]

\vspace{0.3cm}
\textbf{Stap 2: Vul beginwaarde in}

\[
sY(s) - 2 + 4Y(s) = \frac{8}{s}
\]

\vspace{0.3cm}
\textbf{Stap 3: Groepeer termen met \(Y(s)\)}

\[
(s + 4)Y(s) = \frac{8}{s} + 2 = \frac{8 + 2s}{s}
\]

\vspace{0.3cm}
\textbf{Stap 4: Isoleer \(Y(s)\)}

\[
Y(s) = \frac{8 + 2s}{s(s+4)} = \frac{2(s+4)}{s(s+4)} = \frac{2}{s}
\]

\vspace{0.3cm}
\textbf{Stap 5: Inverse transformatie}

\[
\boxed{y(t) = 2u(t)}
\]

\textbf{Interpretatie:} De beginwaarde (\(y = 2\)) is exact gelijk aan de evenwichtswaarde die het systeem zou bereiken (want in steady state is \(y' = 0\), dus \(4y = 8 \Rightarrow y = 2\)). Omdat het systeem al in evenwicht start, vindt er geen transiënt verschijnsel plaats!
\end{oefenblok}

\subsection{Oefening 5.4: Tweede-Orde Systeem met Beginvoorwaarden}

\begin{oefenblok}[Probleem]
Bepaal de oplossing van:
\[
y'' + 4y' + 3y = 6u(t), \quad y(0) = 1, \quad y'(0) = -2
\]
\end{oefenblok}

\begin{oefenblok}[Oplossing]
\textbf{Stap 1: Transform}

\[
(s^2 Y(s) - s + 2) + 4(sY(s) - 1) + 3Y(s) = \frac{6}{s}
\]

\vspace{0.3cm}
\textbf{Stap 2: Herschik}

\[
(s^2 + 4s + 3)Y(s) = \frac{6}{s} + s + 2 = \frac{s^2 + 2s + 6}{s}
\]

\vspace{0.3cm}
\textbf{Stap 3: Factoriseer en vereenvoudig}

\[
Y(s) = \frac{s^2 + 2s + 6}{s(s+1)(s+3)}
\]

\vspace{0.3cm}
\textbf{Stap 4: Partieelbreuksplitsing}

\[
Y(s) = \frac{A}{s} + \frac{B}{s+1} + \frac{C}{s+3}
\]

Via afdekregel:
\begin{align*}
A &= \frac{1 + (-2) + 6}{(1)(3)} = 2 \\
B &= \frac{1 - 2 + 6}{(-1)(-2)} = -2.5 \\
C &= \frac{9 - 6 + 6}{(-3)(-2)} = 1.5
\end{align*}

\vspace{0.3cm}
\textbf{Stap 5: Inverse transformatie}

\[
\boxed{y(t) = (2 - 2.5e^{-t} + 1.5e^{-3t})u(t)}
\]

\textbf{Fysische Interpretatie:} De oplossing bevat een constant deel (evenwicht gelijk aan \(6/3 = 2\)) en twee exponentiële transientcomponenten die naar nul gaan.
\end{oefenblok}

\subsection{Oefening 5.5: Inverse Laplace met Complexe Polen}

\begin{oefenblok}[Probleem]
Bepaal de inverse Laplacetransformatie van:
\[
F(s) = \frac{1}{s^2 + 2s + 5}
\]
\end{oefenblok}

\begin{oefenblok}[Oplossing]
De noemer heeft complexe wortels, want de discriminant \(D = 2^2 - 4(1)(5) = 4 - 20 = -16 < 0\).
We herschrijven de noemer door kwadraatafsplitsing:
\[
s^2 + 2s + 5 = (s^2 + 2s + 1) + 4 = (s+1)^2 + 2^2
\]

De functie wordt:
\[
F(s) = \frac{1}{(s+1)^2 + 2^2}
\]

We herkennen hierin de vorm van een gedempte sinus:
\[
\mathcal{L}\{e^{-at} \sin(\omega t)\} = \frac{\omega}{(s+a)^2 + \omega^2}
\]
Hier is \(a = 1\) en \(\omega = 2\).

Om de teller correct te maken (we hebben 1, maar we hebben \(\omega = 2\) nodig), vermenigvuldigen en delen we door 2:
\[
F(s) = \frac{1}{2} \cdot \frac{2}{(s+1)^2 + 2^2}
\]

De inverse transformatie is dan:
\[
f(t) = \frac{1}{2} e^{-t} \sin(2t) u(t)
\]

\boxed{\text{Resultaat:}}
\[
f(t) = 0.5 e^{-t} \sin(2t) u(t)
\]

\textbf{Interpretatie:} Dit is een gedempte oscillatie met frequentie 2 rad/s en dempingsfactor 1.
\end{oefenblok}

\subsection{Oefening 5.6: Oplossen van een Differentiaalvergelijking}

\begin{oefenblok}[Probleem]
Los de volgende differentiaalvergelijking op met behulp van de Laplacetransformatie:
\[
y''(t) + 4y(t) = 0
\]
met beginvoorwaarden \(y(0) = 2\) en \(y'(0) = 0\).
\end{oefenblok}

\begin{oefenblok}[Oplossing]
We passen de Laplacetransformatie toe op beide zijden van de vergelijking.
Gebruik de eigenschap voor afgeleiden:
\[
\mathcal{L}\{y''(t)\} = s^2 Y(s) - s y(0) - y'(0)
\]
\[
\mathcal{L}\{y(t)\} = Y(s)
\]

Invullen in de differentiaalvergelijking:
\[
(s^2 Y(s) - s(2) - 0) + 4Y(s) = 0
\]

Groepeer de termen met \(Y(s)\):
\[
(s^2 + 4)Y(s) - 2s = 0
\]
\[
(s^2 + 4)Y(s) = 2s
\]

Los op naar \(Y(s)\):
\[
Y(s) = \frac{2s}{s^2 + 4} = 2 \cdot \frac{s}{s^2 + 2^2}
\]

We herkennen hierin direct een standaardtransformatiepaar (zie Formularium \ref{form:laplace-pairs}):
\[
\cos(at) \longleftrightarrow \frac{s}{s^2 + a^2}
\]
met \(a = 2\).

De inverse transformatie geeft:
\[
y(t) = 2 \cos(2t) u(t)
\]

\boxed{\text{Resultaat:}} \(y(t) = 2 \cos(2t)\) voor \(t \ge 0\).
\end{oefenblok}

\subsection{Oefening 5.7: Convolutie via Laplace}

\begin{oefenblok}[Probleem]
Bereken de convolutie \(y(t) = x(t) * h(t)\) van de volgende twee signalen met behulp van de Laplacetransformatie:
\[
x(t) = e^{-t}u(t) \quad \text{en} \quad h(t) = e^{-2t}u(t)
\]
\end{oefenblok}

\begin{oefenblok}[Oplossing]
Convolutie in het tijdsdomein komt overeen met vermenigvuldiging in het s-domein:
\[
Y(s) = X(s) \cdot H(s)
\]

Stap 1: Transformeer de signalen naar het s-domein.
\[
X(s) = \mathcal{L}\{e^{-t}u(t)\} = \frac{1}{s+1}
\]
\[
H(s) = \mathcal{L}\{e^{-2t}u(t)\} = \frac{1}{s+2}
\]

Stap 2: Vermenigvuldig in het s-domein.
\[
Y(s) = \frac{1}{s+1} \cdot \frac{1}{s+2} = \frac{1}{(s+1)(s+2)}
\]

Stap 3: Splits in partiële breuken.
\[
\frac{1}{(s+1)(s+2)} = \frac{A}{s+1} + \frac{B}{s+2}
\]
\[
1 = A(s+2) + B(s+1)
\]
Kies \(s = -1\): \(1 = A(1) \Rightarrow A = 1\).
Kies \(s = -2\): \(1 = B(-1) \Rightarrow B = -1\).

Dus:
\[
Y(s) = \frac{1}{s+1} - \frac{1}{s+2}
\]

Stap 4: Transformeer terug naar het tijdsdomein.
\[
y(t) = \mathcal{L}^{-1}\left\{\frac{1}{s+1}\right\} - \mathcal{L}^{-1}\left\{\frac{1}{s+2}\right\}
\]
\[
y(t) = (e^{-t} - e^{-2t})u(t)
\]

\boxed{\text{Resultaat:}} \(y(t) = (e^{-t} - e^{-2t})u(t)\).
\end{oefenblok}

\subsection{Oefening 5.7: RC-Circuit met Pulsingang}

\emph{Bron: Praktische toepassingen circuits.}

\begin{oefenblok}[Probleem]
Een RC-circuit heeft $R = 2$ k$\Omega$ en $C = 100$ $\mu$F. De ingangsspanning is een puls:
\[
v_{\text{in}}(t) = 10\big(u(t) - u(t-0.1)\big)\text{ V}
\]

Bepaal de uitgangsspanning $v_{\text{uit}}(t)$ met behulp van Laplacetransformatie (neem $v_{\text{uit}}(0) = 0$).
\end{oefenblok}

\begin{oefenblok}[Oplossing]
\textbf{Stap 1: Bepaal de tijdsconstante}

\[
\tau = RC = 2 \times 10^3 \times 100 \times 10^{-6} = 0.2 \text{ s}
\]

\vspace{0.3cm}
\textbf{Stap 2: Stel de differentiaalvergelijking op}

Voor een RC-circuit geldt:
\[
\tau \frac{dv_{\text{uit}}}{dt} + v_{\text{uit}} = v_{\text{in}}
\]

\vspace{0.3cm}
\textbf{Stap 3: Transformeer naar het s-domein}

\[
V_{\text{in}}(s) = 10\left(\frac{1}{s} - \frac{e^{-0.1s}}{s}\right) = \frac{10(1 - e^{-0.1s})}{s}
\]

\[
\tau(sV_{\text{uit}}(s) - 0) + V_{\text{uit}}(s) = V_{\text{in}}(s)
\]

\[
V_{\text{uit}}(s) = \frac{V_{\text{in}}(s)}{\tau s + 1} = \frac{10(1 - e^{-0.1s})}{s(0.2s + 1)}
\]

\vspace{0.3cm}
\textbf{Stap 4: Partieelbreuken en inverse transformatie}

Voor $0 < t < 0.1$ s:
\[
v_{\text{uit}}(t) = 10(1 - e^{-t/0.2})u(t)
\]

Voor $t > 0.1$ s (effect van $-u(t-0.1)$):
\[
v_{\text{uit}}(t) = 10\left[(1 - e^{-t/0.2}) - (1 - e^{-(t-0.1)/0.2})\right]u(t)
\]
\[
= 10\left[e^{-(t-0.1)/0.2} - e^{-t/0.2}\right]u(t)
\]

\boxed{\text{Het signaal laadt exponentieel op gedurende de puls en ontlaadt daarna exponentieel.}}
\end{oefenblok}

\subsection{Oefening 5.8: Transferfunctie en Stabiliteit}

\begin{oefenblok}[Probleem]
Gegeven een systeem met transferfunctie:
\[
H(s) = \frac{s+2}{s^2 - s - 6}
\]
\begin{enumerate}
    \item Bepaal de polen en nulpunten.
    \item Is het systeem stabiel?
    \item Bepaal de impulsrespons \(h(t)\).
\end{enumerate}
\end{oefenblok}

\begin{oefenblok}[Oplossing]
\textbf{1. Polen en Nulpunten}
Nulpunten (teller = 0): \(s + 2 = 0 \Rightarrow s = -2\).
Polen (noemer = 0): \(s^2 - s - 6 = 0\).
Discriminant \(D = (-1)^2 - 4(1)(-6) = 1 + 24 = 25\).
\[
s_{1,2} = \frac{1 \pm 5}{2} \Rightarrow s_1 = 3, \quad s_2 = -2
\]
Polen: \(s = 3\) en \(s = -2\).

\textbf{2. Stabiliteit}
Een systeem is stabiel als alle polen in het linkerhalfvlak liggen (\(\text{Re}(s) < 0\)).
Hier is er een pool op \(s = 3\) (rechterhalfvlak).
\boxed{\text{Het systeem is onstabiel.}}

\textbf{3. Impulsrespons}
\[
H(s) = \frac{s+2}{(s-3)(s+2)} = \frac{1}{s-3} \quad (\text{voor } s \neq -2)
\]
De pool op \(s = -2\) wordt weggedeeld door het nulpunt (pole-zero cancellation). Echter, in de fysieke realiteit blijft de modus \(e^{-2t}\) bestaan als hij wordt aangeslagen door beginvoorwaarden, maar in de transferfunctie van input naar output verdwijnt hij.

De inverse transformatie van de vereenvoudigde \(H(s)\):
\[
h(t) = \mathcal{L}^{-1}\left\{\frac{1}{s-3}\right\} = e^{3t}u(t)
\]

\boxed{h(t) = e^{3t}u(t)}
Dit bevestigt de instabiliteit: de impulsrespons groeit exponentieel naar oneindig.
\end{oefenblok}

\subsection{Oefening 5.8: Tweede-Orde Mechanisch Systeem}

\emph{Bron: Mechanica toepassingen.}

\begin{oefenblok}[Probleem]
Een massa-veer systeem voldoet aan de vergelijking:
\[
m\frac{d^2x}{dt^2} + c\frac{dx}{dt} + kx = F(t)
\]
met $m = 1$ kg, $c = 4$ Ns/m, $k = 5$ N/m, en externe kracht $F(t) = 10u(t)$ N.

Beginvoorwaarden: $x(0) = 0$ en $x'(0) = 0$.

Bepaal de verplaatsing $x(t)$.
\end{oefenblok}

\begin{oefenblok}[Oplossing]
\textbf{Stap 1: Bepaal systeemsparameters}

Natuurlijke eigenfrequentie:
\[
\omega_n = \sqrt{\frac{k}{m}} = \sqrt{\frac{5}{1}} = \sqrt{5} \approx 2.236 \text{ rad/s}
\]

Dempingsfactor:
\[
\zeta = \frac{c}{2\sqrt{km}} = \frac{4}{2\sqrt{5}} = \frac{2}{\sqrt{5}} \approx 0.894
\]

Aangezien $0 < \zeta < 1$, is het systeem \belangrijk{ondergedempt}.

\vspace{0.3cm}
\textbf{Stap 2: Transformeer naar het s-domein}

\[
(s^2 + 4s + 5)X(s) = \frac{10}{s}
\]

\[
X(s) = \frac{10}{s(s^2 + 4s + 5)}
\]

\vspace{0.3cm}
\textbf{Stap 3: Partieelbreuken}

\[
X(s) = \frac{A}{s} + \frac{Bs + C}{s^2 + 4s + 5}
\]

Voor $s = 0$: $10 = 5A \Rightarrow A = 2$

De karakteristieke wortels van $s^2 + 4s + 5 = 0$ zijn:
\[
s = -2 \pm j
\]

Na uitwerking:
\[
X(s) = \frac{2}{s} - \frac{2s + 8}{s^2 + 4s + 5} = \frac{2}{s} - \frac{2(s+2) + 4}{(s+2)^2 + 1}
\]

\vspace{0.3cm}
\textbf{Stap 4: Inverse transformatie}

\[
x(t) = 2 - 2e^{-2t}\cos(t) - 4e^{-2t}\sin(t)
\]
\[
= 2 - e^{-2t}(2\cos(t) + 4\sin(t))
\]

\vspace{0.3cm}
\textbf{Stap 5: Steady-state analyse}

Als $t \to \infty$:
\[
x(\infty) = 2 \text{ m}
\]

Dit klopt met de statische evenwichtswaarde: $kx = F \Rightarrow x = 10/5 = 2$ m.

\boxed{\text{Het systeem oscilleert gedempt naar zijn evenwichtspositie van 2 m.}}
\end{oefenblok}

\subsection{Oefening 5.9: Begin- en Eindwaardestelling}

\emph{Bron: Praktische analyse technieken.}

\begin{oefenblok}[Probleem]
Gegeven:
\[
F(s) = \frac{3s + 5}{s^2 + 4s + 3}
\]

Bepaal:
\begin{enumerate}[label=(\alph*)]
\item $f(0^+)$ met de beginwaardestelling
\item $\lim_{t \to \infty} f(t)$ met de eindwaardestelling
\item Verifieer door $f(t)$ expliciet te bepalen
\end{enumerate}
\end{oefenblok}

\begin{oefenblok}[Oplossing]
\textbf{(a) Beginwaarde:}

Beginwaardestelling: $f(0^+) = \lim_{s \to \infty} sF(s)$

\[
\lim_{s \to \infty} s \cdot \frac{3s + 5}{s^2 + 4s + 3} = \lim_{s \to \infty} \frac{3s^2 + 5s}{s^2 + 4s + 3} = 3
\]

\boxed{f(0^+) = 3}

\vspace{0.3cm}
\textbf{(b) Eindwaarde:}

Eindwaardestelling: $\lim_{t \to \infty} f(t) = \lim_{s \to 0} sF(s)$

\[
\lim_{s \to 0} s \cdot \frac{3s + 5}{s^2 + 4s + 3} = \lim_{s \to 0} \frac{3s^2 + 5s}{s^2 + 4s + 3} = \frac{0}{3} = 0
\]

\boxed{\lim_{t \to \infty} f(t) = 0}

\vspace{0.3cm}
\textbf{(c) Verificatie:}

Factoriseer: $s^2 + 4s + 3 = (s+1)(s+3)$

Partieelbreuken:
\[
F(s) = \frac{3s + 5}{(s+1)(s+3)} = \frac{A}{s+1} + \frac{B}{s+3}
\]

Voor $s = -1$: $3(-1) + 5 = 2A \Rightarrow A = 1$

Voor $s = -3$: $3(-3) + 5 = -2B \Rightarrow B = 2$

Dus:
\[
F(s) = \frac{1}{s+1} + \frac{2}{s+3}
\]

Inverse transformatie:
\[
f(t) = (e^{-t} + 2e^{-3t})u(t)
\]

Controle:
\[
f(0^+) = 1 + 2 = 3 \quad \checkmark
\]
\[
f(\infty) = 0 + 0 = 0 \quad \checkmark
\]

\textbf{Belangrijk:} De eindwaardestelling kan alleen gebruikt worden als alle polen van $sF(s)$ in het linker halfvlak liggen (stabiel systeem).
\end{oefenblok}

\subsection{Oefening 5.10: IVP met Externe Forcing}

\begin{oefenblok}[Probleem]
Los de volgende differentiaalvergelijking op met behulp van de Laplacetransformatie:
\[
y''(t) + 3y'(t) + 2y(t) = e^{-t}, \quad y(0) = 0, \quad y'(0) = 1
\]
\end{oefenblok}

\begin{oefenblok}[Oplossing]
Transformeren:
\[
(s^2 Y(s) - sy(0) - y'(0)) + 3(sY(s) - y(0)) + 2Y(s) = \frac{1}{s+1}
\]
Invullen beginvoorwaarden (\(y(0)=0, y'(0)=1\)):
\[
(s^2 Y - 1) + 3sY + 2Y = \frac{1}{s+1}
\]
Herschikken:
\[
(s^2 + 3s + 2)Y = \frac{1}{s+1} + 1 = \frac{1 + (s+1)}{s+1} = \frac{s+2}{s+1}
\]
Factoriseer de karakteristieke polynoom: \(s^2 + 3s + 2 = (s+1)(s+2)\).

Isoleren \(Y(s)\):
\[
Y(s) = \frac{s+2}{(s+1)(s+1)(s+2)} = \frac{1}{(s+1)^2}
\]
(Merk op: de factor \((s+2)\) valt weg, wat de berekening vereenvoudigt).

Terugtransformeren:
We herkennen hier het paar voor \(t e^{-at} u(t) \leftrightarrow \frac{1}{(s+a)^2}\).
Met \(a=1\) geeft dit:
\[
y(t) = t e^{-t} u(t)
\]

\begin{center}
\begin{tikzpicture}[scale=0.8]
    \begin{axis}[
        width=8cm, height=5cm,
        axis lines=middle,
        xlabel={$t$}, ylabel={$y(t) = t e^{-t}$},
        domain=0:6, samples=100,
        ymin=0, ymax=0.4,
        grid=both
    ]
    \addplot[thick, blue] {x * exp(-x)};
    \end{axis}
\end{tikzpicture}
\end{center}
\end{oefenblok}

% ============================================================================
% RESTERENDE HOOFDSTUKKEN - VERKORTE BEHANDELING
% ============================================================================

\chapter{De Fouriertransformatie}

\section{Frequentie-analyse van Niet-Periodieke Signalen}

Terwijl Laplace ideaal is voor transiënte analyse en stabiliteit (het volledige s-vlak), is de Fouriertransformatie (FT) geoptimaliseerd voor pure frequentie-analyse en signaalverwerking (de imaginaire as, \(s = j\omega\)).

De kernidee van Fourieranalyse is diepgaand: \textbf{Elk fysisch signaal, hoe complex ook, kan worden beschouwd als een som (of integraal) van sinusvormige golven.}

\begin{theorieblok}[Het Spectrum]
De Fouriertransformatie vertaalt een signaal van het tijdsdomein (hoe het signaal eruitziet op de oscilloscoop) naar het frequentiedomein (hoe het signaal eruitziet op de spectrumanalyzer).
\begin{itemize}
    \item \textbf{Tijdsdomein \(x(t)\):} Vertelt ons \emph{wanneer} iets gebeurt.
    \item \textbf{Frequentiedomein \(X(\omega)\):} Vertelt ons \emph{hoe vaak} iets gebeurt (welke frequenties aanwezig zijn).
\end{itemize}
\end{theorieblok}

\begin{theorieblok}[Filters en Systeemrespons]
Een van de krachtigste toepassingen van de Fouriertransformatie is het analyseren van filters.
Als we een signaal \(x(t)\) door een systeem (filter) sturen met impulsrespons \(h(t)\), dan is de output \(y(t)\) de convolutie in de tijd:
\[ y(t) = x(t) * h(t) \]
In het frequentiedomein wordt dit een eenvoudige vermenigvuldiging:
\[ Y(\omega) = X(\omega) \cdot H(\omega) \]
Hierbij is \(H(\omega)\) de \textbf{frequentierespons} van het filter.
\begin{itemize}
    \item \textbf{Laagdoorlaatfilter:} Laat lage frequenties door (\(H(\omega) \approx 1\) voor kleine \(\omega\)) en blokkeert hoge (\(H(\omega) \approx 0\) voor grote \(\omega\)).
    \item \textbf{Hoogdoorlaatfilter:} Blokkeert lage frequenties en laat hoge door.
    \item \textbf{Banddoorlaatfilter:} Laat enkel een specifiek frequentiegebied door.
\end{itemize}
Dit maakt het ontwerpen van equalizers en ruisonderdrukkers inzichtelijk: je "knipt" simpelweg de ongewenste frequenties weg in \(X(\omega)\).
\end{theorieblok}

\begin{definition}
\[
X(\omega) = \int_{-\infty}^{\infty} x(t) e^{-j\omega t} dt
\]

Dit ontleedt een signaal in een continu spectrum van frequenties. Het stelt ons in staat om te zien welke frequenties aanwezig zijn in een puls, een stem, of een digitaal signaal.
\end{definition}

Een fundamenteel inzicht in signaalverwerking is de inverse relatie tussen tijd en frequentie:
\begin{itemize}
    \item Een signaal dat zeer kort duurt in de tijd (zoals een klik), is zeer breed uitgesmeerd in frequentie.
    \item Een signaal dat zeer lang duurt en weinig verandert (zoals een constante toon), is zeer smal in frequentie.
\end{itemize}

\subsection{Oefening 6.1: Spectrum van een Rechthoekpuls}

\begin{oefenblok}[Probleem]
Bepaal het spectrum van een rechthoekpuls met amplitude \(A\) en breedte \(T\) (van \(-T/2\) tot \(T/2\)).
\end{oefenblok}

\begin{oefenblok}[Oplossing]
\[
X(\omega) = \int_{-T/2}^{T/2} A e^{-j\omega t} dt = A \left[ \frac{-1}{j\omega} e^{-j\omega t} \right]_{-T/2}^{T/2}
\]

\[
= \frac{A}{j\omega} \left( e^{j\omega T/2} - e^{-j\omega T/2} \right) = A \frac{2\sin(\omega T/2)}{\omega}
\]

\[
\boxed{X(\omega) = AT \cdot \frac{\sin(\omega T/2)}{\omega T/2} = AT \cdot \text{sinc}\left(\frac{\omega T}{2}\right)}
\]

\textbf{Inzicht:} Een rechthoekig signaal in de tijd resulteert in een sinc-functie in frequentie. Dit heeft oneindige bandbreedte. Hoe korter de puls in tijd (\(T \downarrow\)), hoe breder het spectrum in frequentie (bandbreedte \(\uparrow\)). Dit is de onzekerheidsrelatie van signaalverwerking.
\end{oefenblok}

\subsection{Oefening 6.2: Verschuivingseigenschap}

\begin{oefenblok}[Probleem]
Gegeven is dat de Fouriertransformatie van \(x(t)\) gelijk is aan \(X(\omega)\). Bepaal de Fouriertransformatie van \(y(t) = x(t - t_0)\).
\end{oefenblok}

\begin{oefenblok}[Oplossing]
We passen de definitie toe op \(y(t)\):
\[
Y(\omega) = \int_{-\infty}^{\infty} x(t - t_0) e^{-j\omega t} dt
\]

Stel een substitutie voor: \(\tau = t - t_0\), dus \(t = \tau + t_0\) en \(dt = d\tau\).
De grenzen blijven \(-\infty\) tot \(\infty\).

\[
Y(\omega) = \int_{-\infty}^{\infty} x(\tau) e^{-j\omega (\tau + t_0)} d\tau
\]
\[
= e^{-j\omega t_0} \int_{-\infty}^{\infty} x(\tau) e^{-j\omega \tau} d\tau
\]

De integraal is precies de definitie van \(X(\omega)\).

\[
\boxed{Y(\omega) = e^{-j\omega t_0} X(\omega)}
\]

\textbf{Interpretatie:} Een verschuiving in de tijd verandert de amplitude van het spectrum niet (\(|e^{-j\omega t_0}| = 1\)), maar voegt een lineaire faseverschuiving toe (\(-\omega t_0\)). Dit betekent dat alle frequentiecomponenten met een verschillende hoeveelheid worden vertraagd, proportioneel aan hun frequentie, om de golfvorm intact te houden.
\end{oefenblok}

\subsection{Oefening 6.3: Differentiatie in het Frequentiedomein}

\begin{oefenblok}[Probleem]
Bewijs de eigenschap voor vermenigvuldiging met \(t\):
Als \(\mathcal{F}\{x(t)\} = X(\omega)\), toon dan aan dat:
\[
\mathcal{F}\{t x(t)\} = j \frac{d}{d\omega} X(\omega)
\]
\end{oefenblok}

\begin{oefenblok}[Oplossing]
We vertrekken van de definitie van de Fouriertransformatie:
\[
X(\omega) = \int_{-\infty}^{\infty} x(t) e^{-j\omega t} dt
\]

We differentiëren beide zijden naar \(\omega\). Omdat de integraal over \(t\) loopt en convergeert, mogen we de afgeleide binnen de integraal brengen (Leibniz integral rule):
\[
\frac{d}{d\omega} X(\omega) = \int_{-\infty}^{\infty} x(t) \frac{\partial}{\partial \omega} (e^{-j\omega t}) dt
\]

De afgeleide van de exponentiële functie is:
\[
\frac{\partial}{\partial \omega} (e^{-j\omega t}) = (-jt) e^{-j\omega t}
\]

Invullen geeft:
\[
\frac{d}{d\omega} X(\omega) = \int_{-\infty}^{\infty} x(t) (-jt) e^{-j\omega t} dt
\]
\[
\frac{d}{d\omega} X(\omega) = -j \int_{-\infty}^{\infty} (t x(t)) e^{-j\omega t} dt
\]

De integraal aan de rechterkant is per definitie de Fouriertransformatie van het signaal \(t x(t)\).
\[
\frac{d}{d\omega} X(\omega) = -j \mathcal{F}\{t x(t)\}
\]

Vermenigvuldig beide zijden met \(j\) (merk op dat \(j \cdot -j = -j^2 = -(-1) = 1\)):
\[
j \frac{d}{d\omega} X(\omega) = \mathcal{F}\{t x(t)\}
\]

\boxed{\text{Q.E.D.}}
\end{oefenblok}

\subsection{Oefening 6.4: Impulsrespons via Fourier}

\begin{oefenblok}[Probleem]
Bepaal de impulsrespons \(h(t)\) van een causaal LTI-systeem dat wordt beschreven door de differentiaalvergelijking:
\[
y''(t) + 3y'(t) + 2y(t) = x(t)
\]
Gebruik de Fouriertransformatie.
\end{oefenblok}

\begin{oefenblok}[Oplossing]
De impulsrespons \(h(t)\) is de output \(y(t)\) wanneer de input \(x(t) = \delta(t)\).
We transformeren de vergelijking naar het frequentiedomein.
Gebruik de eigenschap \(\mathcal{F}\{y^{(n)}(t)\} = (j\omega)^n Y(\omega)\).

\[
(j\omega)^2 Y(\omega) + 3(j\omega) Y(\omega) + 2 Y(\omega) = X(\omega)
\]

Voor \(x(t) = \delta(t)\) is \(X(\omega) = 1\). En \(Y(\omega)\) is dan \(H(\omega)\) (de frequentierespons).
\[
(-\omega^2 + 3j\omega + 2) H(\omega) = 1
\]
\[
H(\omega) = \frac{1}{-\omega^2 + 3j\omega + 2} = \frac{1}{(j\omega)^2 + 3(j\omega) + 2}
\]

We factoriseren de noemer (stel \(s = j\omega\)):
\[
s^2 + 3s + 2 = (s+1)(s+2)
\]
Dus:
\[
H(\omega) = \frac{1}{(j\omega + 1)(j\omega + 2)}
\]

We splitsen in partiële breuken:
\[
\frac{1}{(j\omega + 1)(j\omega + 2)} = \frac{A}{j\omega + 1} + \frac{B}{j\omega + 2}
\]
\[
1 = A(j\omega + 2) + B(j\omega + 1)
\]
Stel \(j\omega = -1 \Rightarrow A = 1\).
Stel \(j\omega = -2 \Rightarrow B = -1\).

\[
H(\omega) = \frac{1}{j\omega + 1} - \frac{1}{j\omega + 2}
\]

We gebruiken het standaardpaar \(e^{-at}u(t) \longleftrightarrow \frac{1}{j\omega + a}\).
\[
h(t) = e^{-t}u(t) - e^{-2t}u(t)
\]

\boxed{\text{Resultaat:}} \(h(t) = (e^{-t} - e^{-2t})u(t)\).
\end{oefenblok}

\subsection{Oefening 6.5: Modulatie-eigenschap}

\begin{oefenblok}[Probleem]
Bewijs de modulatie-eigenschap:
Als \(\mathcal{F}\{x(t)\} = X(\omega)\), toon dan aan dat:
\[
\mathcal{F}\{x(t) \cos(\omega_0 t)\} = \frac{1}{2} [X(\omega - \omega_0) + X(\omega + \omega_0)]
\]
\end{oefenblok}

\begin{oefenblok}[Oplossing]
We gebruiken de formule van Euler voor de cosinus:
\[
\cos(\omega_0 t) = \frac{e^{j\omega_0 t} + e^{-j\omega_0 t}}{2}
\]

Dus:
\[
x(t) \cos(\omega_0 t) = \frac{1}{2} x(t) e^{j\omega_0 t} + \frac{1}{2} x(t) e^{-j\omega_0 t}
\]

We nemen de Fouriertransformatie van beide termen. We gebruiken de frequentieverschuivingseigenschap:
\[
\mathcal{F}\{x(t) e^{j\omega_0 t}\} = X(\omega - \omega_0)
\]
\[
\mathcal{F}\{x(t) e^{-j\omega_0 t}\} = X(\omega - (-\omega_0)) = X(\omega + \omega_0)
\]

Door lineariteit geldt dan direct:
\[
\mathcal{F}\{x(t) \cos(\omega_0 t)\} = \frac{1}{2} X(\omega - \omega_0) + \frac{1}{2} X(\omega + \omega_0)
\]

\boxed{\text{Q.E.D.}}

\textbf{Toepassing:} Dit is het basisprincipe van AM-radio (Amplitude Modulatie). Het spectrum van het audiosignaal \(X(\omega)\) wordt verschoven naar de draaggolffrequentie \(\omega_0\).
\end{oefenblok}

\subsection{Oefening 6.6: Tweezijdig Exponentieel Signaal}

\begin{oefenblok}[Probleem]
Bepaal de Fouriertransformatie van het signaal \(x(t) = e^{-a|t|}\) met \(a > 0\). Dit signaal is niet causaal (het bestaat voor \(t<0\) en \(t>0\)).

\begin{center}
\begin{tikzpicture}[scale=0.8]
    \begin{axis}[
        width=8cm, height=5cm,
        axis lines=middle,
        xlabel={$t$}, ylabel={$x(t) = e^{-a|t|}$},
        domain=-3:3, samples=100,
        ymin=0, ymax=1.2,
        grid=both
    ]
    \addplot[thick, blue] {exp(-abs(x))};
    \node at (axis cs: 1, 0.8) [anchor=west] {$a=1$};
    \end{axis}
\end{tikzpicture}
\end{center}
\end{oefenblok}

\begin{oefenblok}[Oplossing]
We splitsen de integraal in twee delen: \(t < 0\) en \(t > 0\).
\[
X(\omega) = \int_{-\infty}^{0} e^{at} e^{-j\omega t} dt + \int_{0}^{\infty} e^{-at} e^{-j\omega t} dt
\]
(Merk op: voor \(t<0\) is \(|t| = -t\), dus \(e^{-a|t|} = e^{at}\))

Deel 1 (\(t<0\)):
\[
\int_{-\infty}^{0} e^{(a-j\omega)t} dt = \left[ \frac{1}{a-j\omega} e^{(a-j\omega)t} \right]_{-\infty}^0 = \frac{1}{a-j\omega} (1 - 0) = \frac{1}{a-j\omega}
\]

Deel 2 (\(t>0\)):
\[
\int_{0}^{\infty} e^{-(a+j\omega)t} dt = \left[ \frac{-1}{a+j\omega} e^{-(a+j\omega)t} \right]_0^{\infty} = \frac{1}{a+j\omega}
\]

Optellen:
\[
X(\omega) = \frac{1}{a-j\omega} + \frac{1}{a+j\omega} = \frac{(a+j\omega) + (a-j\omega)}{(a-j\omega)(a+j\omega)} = \frac{2a}{a^2 + \omega^2}
\]
Dit is een reële, even functie in het frequentiedomein (een Lorentz-curve), wat consistent is met het feit dat \(x(t)\) reëel en even is in de tijd.
\end{oefenblok}

\subsection{Oefening 6.7: Inverse Fouriertransformatie van een Ideaal Laagdoorlaatfilter}

\begin{oefenblok}[Probleem]
Bepaal de inverse Fouriertransformatie van een ideaal laagdoorlaatfilter met bandbreedte \(\omega_0\) en amplitude \(A\):
\[
X(\omega) = \begin{cases} A & \text{als } -\omega_0 \le \omega \le \omega_0 \\ 0 & \text{anders} \end{cases}
\]
\end{oefenblok}

\begin{oefenblok}[Oplossing]
We gebruiken de definitie van de inverse Fouriertransformatie:
\[
x(t) = \frac{1}{2\pi} \int_{-\infty}^{\infty} X(\omega) e^{j\omega t} d\omega
\]
Omdat \(X(\omega)\) nul is buiten \([-\omega_0, \omega_0]\), worden de grenzen:
\[
x(t) = \frac{1}{2\pi} \int_{-\omega_0}^{\omega_0} A e^{j\omega t} d\omega = \frac{A}{2\pi} \left[ \frac{e^{j\omega t}}{jt} \right]_{-\omega_0}^{\omega_0}
\]
\[
x(t) = \frac{A}{2\pi jt} (e^{j\omega_0 t} - e^{-j\omega_0 t})
\]
Gebruik de formule van Euler: \(e^{j\theta} - e^{-j\theta} = 2j \sin(\theta)\).
\[
x(t) = \frac{A}{2\pi jt} (2j \sin(\omega_0 t)) = \frac{A \sin(\omega_0 t)}{\pi t}
\]
We kunnen dit herschrijven in termen van de sinc-functie (\(\text{sinc}(x) = \frac{\sin(x)}{x}\)):
\[
x(t) = \frac{A \omega_0}{\pi} \frac{\sin(\omega_0 t)}{\omega_0 t} = \frac{A \omega_0}{\pi} \text{sinc}(\omega_0 t)
\]

\textbf{Interpretatie:} Een ideale blok in het frequentiedomein (ideaal filter) correspondeert met een sinc-functie in het tijdsdomein. Omdat de sinc-functie zich uitstrekt van \(-\infty\) tot \(+\infty\) (niet causaal), is een ideaal laagdoorlaatfilter fysisch niet realiseerbaar.
\end{oefenblok}

\subsection{Oefening 6.8: Eenvoudige Inverse Transformaties}

\begin{oefenblok}[Probleem]
Bepaal de inverse Fouriertransformatie van de volgende spectra met behulp van de tabel (zie Formularium \ref{form:ft-pairs}):
\begin{enumerate}
    \item \(X_1(\omega) = \frac{1}{3 + j\omega}\)
    \item \(X_2(\omega) = \frac{10}{2 + j\omega} + 2\pi \delta(\omega)\)
    \item \(X_3(\omega) = \frac{4}{4 + \omega^2}\)
\end{enumerate}
\end{oefenblok}

\begin{oefenblok}[Oplossing]
\textbf{1. Exponentieel Verval}
We herkennen de vorm \(\frac{1}{a + j\omega}\).
Uit het formularium weten we dat:
\[
e^{-at}u(t) \longleftrightarrow \frac{1}{a + j\omega}
\]
Hier is \(a=3\).
\[
\boxed{x_1(t) = e^{-3t}u(t)}
\]

\textbf{2. Lineariteit en DC-component}
We gebruiken de lineariteitseigenschap: \(\mathcal{F}^{-1}\{aX + bY\} = a\mathcal{F}^{-1}\{X\} + b\mathcal{F}^{-1}\{Y\}\).
\begin{itemize}
    \item Term 1: \(\frac{10}{2 + j\omega} = 10 \cdot \frac{1}{2 + j\omega}\). Dit geeft \(10 e^{-2t}u(t)\).
    \item Term 2: \(2\pi \delta(\omega)\). Uit het formularium weten we dat \(1 \longleftrightarrow 2\pi \delta(\omega)\). Dit is een constante DC-waarde van 1.
\end{itemize}
\[
\boxed{x_2(t) = 10 e^{-2t}u(t) + 1}
\]

\textbf{3. Tweezijdig Exponentieel}
De functie \(\frac{4}{4 + \omega^2}\) lijkt op de Lorentz-curve \(\frac{2a}{a^2 + \omega^2}\).
Hier is \(a^2 = 4 \Rightarrow a = 2\).
De teller moet dan \(2a = 4\) zijn, wat klopt.
Het transformatiepaar is:
\[
e^{-a|t|} \longleftrightarrow \frac{2a}{a^2 + \omega^2}
\]
\[
\boxed{x_3(t) = e^{-2|t|}}
\]
\end{oefenblok}

\subsection{Oefening 6.9: Dualiteit}

\begin{oefenblok}[Probleem]
Gegeven is het spectrum \(X(\omega) = \text{rect}(\omega/2)\). Bepaal \(x(t)\).
\end{oefenblok}

\begin{oefenblok}[Oplossing]
We weten uit Oefening 6.1 dat een rechthoek in de tijd een sinc in frequentie geeft:
\[
\text{rect}(t/T) \longleftrightarrow T \cdot \text{sinc}(\omega T/2)
\]
De eigenschap van \concept{dualiteit} stelt dat de rollen van tijd en frequentie omkeerbaar zijn (op een factor \(2\pi\) en een minteken na).
\[
X(t) \longleftrightarrow 2\pi x(-\omega)
\]
Of eenvoudiger, we zoeken in de tabel naar een sinc in de tijd.
Het paar is:
\[
\frac{\omega_c}{\pi} \text{sinc}(\omega_c t) \longleftrightarrow \text{rect}\left(\frac{\omega}{2\omega_c}\right)
\]
We hebben \(\text{rect}(\omega/2)\). Dit komt overeen met \(2\omega_c = 2 \Rightarrow \omega_c = 1\).
Dus:
\[
x(t) = \frac{1}{\pi} \text{sinc}(t)
\]
\[
\boxed{x(t) = \frac{\sin(t)}{\pi t}}
\]
\end{oefenblok}

\subsection{Oefening 6.10: Limietstellingen en Oppervlakte-eigenschappen}

\begin{oefenblok}[Probleem]
Gegeven is het spectrum van een causaal signaal \(x(t)\):
\[
X(\omega) = \frac{4}{2 + j\omega}
\]
Bepaal zonder de inverse transformatie volledig uit te rekenen:
\begin{enumerate}[label=(\alph*)]
    \item De integraal van het signaal over de hele tijd: \(\int_{-\infty}^{\infty} x(t) dt\).
    \item De beginwaarde van het signaal: \(x(0^+)\).
\end{enumerate}
\end{oefenblok}

\begin{oefenblok}[Oplossing]
\textbf{(a) Integraal van het signaal (DC-waarde)}
De oppervlakte onder het tijdsignaal is gelijk aan de waarde van het spectrum bij \(\omega = 0\):
\[
\int_{-\infty}^{\infty} x(t) dt = X(0)
\]
Invullen van \(\omega = 0\) in \(X(\omega)\):
\[
X(0) = \frac{4}{2 + j(0)} = \frac{4}{2} = 2
\]
\boxed{\text{Oppervlakte} = 2}

\textbf{(b) Beginwaarde (Initial Value Theorem)}
Voor een causaal signaal geldt de limietstelling (analoog aan Laplace):
\[
x(0^+) = \lim_{\omega \to \infty} j\omega X(\omega)
\]
\[
x(0^+) = \lim_{\omega \to \infty} j\omega \frac{4}{2 + j\omega} = \lim_{\omega \to \infty} \frac{4j\omega}{j\omega(2/j\omega + 1)} = \lim_{\omega \to \infty} \frac{4}{1} = 4
\]
\emph{Controle:} Het signaal is \(x(t) = 4e^{-2t}u(t)\). Op \(t=0\) is dit inderdaad 4.

\boxed{x(0^+) = 4}
\end{oefenblok}

\chapter{De Fourierreeks}

\section{Analyse van Periodieke Signalen}

Waar de Fouriertransformatie elk signaal kan analyseren, is de Fourierreeks specifiek bedoeld voor \concept{periodieke} signalen: signalen die zichzelf exact herhalen na een vaste tijd \(T\).

Het centrale inzicht van Joseph Fourier was revolutionair: \textbf{Elke periodieke golfvorm, hoe hoekig of grillig ook, kan worden opgebouwd door een oneindig aantal perfecte sinussen en cosinussen bij elkaar op te tellen.}

Deze bouwstenen hebben specifieke frequenties:
\begin{itemize}
    \item \textbf{Grondfrequentie (\(\omega_0 = 2\pi/T\)):} Bepaalt de toonhoogte of de periode van het signaal.
    \item \textbf{Harmonischen (\(k\omega_0\)):} Veelvouden van de grondfrequentie. Deze bepalen de vorm of de "klankkleur" van het signaal.
\end{itemize}

Dit verklaart waarom een viool anders klinkt dan een piano, zelfs als ze dezelfde noot (dezelfde grondfrequentie) spelen: de verhouding van hun harmonischen (de Fouriercoëfficiënten) is anders.

Er zijn twee gangbare manieren om de Fourierreeks te noteren: de complexe vorm en de trigonometrische (of Cartesische) vorm.

\subsection{Complexe Exponentiële Vorm}

Dit is de meest compacte notatie, die veel wordt gebruikt in systeemanalyse.

\begin{definition}
\[
x(t) = \sum_{k=-\infty}^{\infty} c_k e^{jk\omega_0 t}
\]
Waarbij de complexe coëfficiënten \(c_k\) worden berekend als:
\[
c_k = \frac{1}{T} \int_{0}^{T} x(t) e^{-jk\omega_0 t} dt
\]
\end{definition}

\subsection{Trigonometrische (Cartesische) Vorm}

Voor reële signalen is het vaak intuïtiever om te werken met sinussen en cosinussen. Dit wordt de Cartesische variant genoemd omdat we het signaal ontbinden in orthogonale componenten.

\begin{definition}
\[
x(t) = a_0 + \sum_{n=1}^{\infty} \left( a_n \cos(n\omega_0 t) + b_n \sin(n\omega_0 t) \right)
\]
De coëfficiënten worden als volgt berekend:
\begin{align*}
a_0 &= \frac{1}{T} \int_{0}^{T} x(t) dt \quad \text{(de gemiddelde waarde of DC-component)} \\
a_n &= \frac{2}{T} \int_{0}^{T} x(t) \cos(n\omega_0 t) dt \\
b_n &= \frac{2}{T} \int_{0}^{T} x(t) \sin(n\omega_0 t) dt
\end{align*}
\end{definition}

\subsubsection{Verband tussen beide vormen}
De complexe coëfficiënten \(c_k\) en de reële coëfficiënten \(a_n, b_n\) zijn direct aan elkaar gerelateerd:
\begin{itemize}
    \item \(c_0 = a_0\)
    \item \(c_n = \frac{1}{2}(a_n - j b_n)\) voor \(n > 0\)
    \item \(c_{-n} = \frac{1}{2}(a_n + j b_n)\) voor \(n > 0\)
\end{itemize}

\subsection{Oefening 7.1: Blokgolf Analyse}

\begin{oefenblok}[Probleem]
Beschouw een blokgolf met periode \(T = 2\), die schakelt tussen \(+1\) (voor \(0 < t < 1\)) en \(-1\) (voor \(1 < t < 2\)). Bepaal de Fouriercoëfficiënten.
\end{oefenblok}

\begin{oefenblok}[Oplossing]
Grondfrequentie \(\omega_0 = \pi\).

De functie is oneven, dus enkel sinustermen met \(b_k = \frac{2}{T} \int_0^T f(t) \sin(k\omega_0 t) dt\).

Na berekening krijgen we dat voor even \(k\), \(b_k = 0\), en voor oneven \(k\):
\[
b_k = \frac{4}{k\pi}
\]

De Fourierreeks is:
\[
f(t) = \frac{4}{\pi} \sin(\pi t) + \frac{4}{3\pi} \sin(3\pi t) + \frac{4}{5\pi} \sin(5\pi t) + \cdots
\]

\textbf{Conclusie:} Een blokgolf is opgebouwd uit oneindig veel oneven harmonischen waarvan de amplitude afneemt als \(1/k\).
\end{oefenblok}

\subsection{Oefening 7.2: Vermogen van een Signaal}

\begin{oefenblok}[Probleem]
Bereken het gemiddelde vermogen van de blokgolf uit Oefening 7.1 met behulp van de stelling van Parseval. De blokgolf heeft amplitude 1.
\end{oefenblok}

\begin{oefenblok}[Oplossing]
\textbf{Methode 1: Tijdsdomein}
Het gemiddelde vermogen \(P\) van een periodiek signaal \(x(t)\) met periode \(T\) is:
\[
P = \frac{1}{T} \int_0^T |x(t)|^2 dt
\]
Voor onze blokgolf is \(|x(t)| = 1\) (want het is \(1\) of \(-1\)).
\[
P = \frac{1}{2} \int_0^2 1^2 dt = \frac{1}{2} [t]_0^2 = \frac{1}{2} (2 - 0) = 1
\]

\textbf{Methode 2: Frequentiedomein (Parseval)}
Parseval stelt dat het totale vermogen gelijk is aan de som van de vermogens van de harmonischen:
\[
P = \sum_{k=-\infty}^{\infty} |c_k|^2
\]
Voor de reële Fourierreeks (met \(a_k, b_k\)) is dit:
\[
P = a_0^2 + \frac{1}{2} \sum_{n=1}^{\infty} (a_n^2 + b_n^2)
\]
Hier is \(a_0 = 0\) (gemiddelde is 0) en \(a_n = 0\).
We vonden \(b_n = \frac{4}{n\pi}\) voor oneven \(n\), en 0 voor even \(n\).

\[
P = \frac{1}{2} \sum_{n \in \{1,3,5,\dots\}} \left(\frac{4}{n\pi}\right)^2 = \frac{1}{2} \cdot \frac{16}{\pi^2} \sum_{k=0}^{\infty} \frac{1}{(2k+1)^2}
\]
\[
P = \frac{8}{\pi^2} \left( 1 + \frac{1}{9} + \frac{1}{25} + \cdots \right)
\]
Het is een bekend wiskundig resultaat dat \(\sum_{k=0}^{\infty} \frac{1}{(2k+1)^2} = \frac{\pi^2}{8}\).

\[
P = \frac{8}{\pi^2} \cdot \frac{\pi^2}{8} = 1
\]

\boxed{\text{Resultaat:}} Het vermogen is 1 Watt (aannemende 1 Ohm belasting). Parseval bevestigt dat de energie behouden blijft bij transformatie naar het frequentiedomein.
\end{oefenblok}

\subsection{Oefening 7.3: Fourierreeks van een Zaagtandgolf}

\begin{oefenblok}[Probleem]
Bepaal de Fourierreeks van de functie \(f(t) = t\) gedefinieerd op het interval \(-\pi < t < \pi\) met periode \(T = 2\pi\).
\end{oefenblok}

\begin{oefenblok}[Oplossing]
De fundamentele frequentie is \(\omega_0 = \frac{2\pi}{2\pi} = 1\).
De functie \(f(t) = t\) is een \textbf{oneven} functie, want \(f(-t) = -t = -f(t)\).
Dit betekent dat alle cosinus-coëfficiënten nul zijn:
\[
a_0 = 0 \quad \text{en} \quad a_n = 0
\]
We hoeven enkel de sinus-coëfficiënten \(b_n\) te berekenen:
\[
b_n = \frac{2}{T} \int_{-T/2}^{T/2} f(t) \sin(n\omega_0 t) dt = \frac{1}{\pi} \int_{-\pi}^{\pi} t \sin(nt) dt
\]
Omdat de integrand \(t \sin(nt)\) even is (oneven \(\times\) oneven = even), kunnen we schrijven:
\[
b_n = \frac{2}{\pi} \int_{0}^{\pi} t \sin(nt) dt
\]

We gebruiken partiële integratie: \(\int u dv = uv - \int v du\).
Stel \(u = t \Rightarrow du = dt\).
Stel \(dv = \sin(nt) dt \Rightarrow v = -\frac{1}{n} \cos(nt)\).

\[
\int_0^{\pi} t \sin(nt) dt = \left[ -\frac{t}{n} \cos(nt) \right]_0^{\pi} - \int_0^{\pi} \left( -\frac{1}{n} \cos(nt) \right) dt
\]
\[
= \left( -\frac{\pi}{n} \cos(n\pi) - 0 \right) + \frac{1}{n} \left[ \frac{1}{n} \sin(nt) \right]_0^{\pi}
\]
We weten dat \(\cos(n\pi) = (-1)^n\) en \(\sin(n\pi) = 0\).
\[
= -\frac{\pi}{n}(-1)^n + \frac{1}{n^2}(0 - 0) = \frac{\pi}{n}(-1)^{n+1}
\]

Invullen in de formule voor \(b_n\):
\[
b_n = \frac{2}{\pi} \cdot \frac{\pi}{n}(-1)^{n+1} = \frac{2}{n}(-1)^{n+1}
\]

De Fourierreeks is dus:
\[
f(t) = \sum_{n=1}^{\infty} \frac{2}{n}(-1)^{n+1} \sin(nt)
\]
\[
f(t) = 2 \left( \sin(t) - \frac{1}{2}\sin(2t) + \frac{1}{3}\sin(3t) - \dots \right)
\]

\boxed{\text{Resultaat:}} \(f(t) = 2 \sum_{n=1}^{\infty} \frac{(-1)^{n+1}}{n} \sin(nt)\).
\end{oefenblok}

\subsection{Oefening 7.4: Driehoeksgolf (Even Functie)}

\begin{oefenblok}[Probleem]
Bepaal de Fourierreeks van de periodieke driehoeksgolf $x(t)$ met periode $T=2$, gedefinieerd als:
$x(t) = 1 - |t|$ voor $-1 \le t \le 1$.
(Dit is een even functie, symmetrisch rond de y-as).
\end{oefenblok}

\begin{oefenblok}[Oplossing]
\textbf{Symmetrie:} De functie is even, dus $b_n = 0$ (geen sinustermen). We zoeken enkel $a_0$ en $a_n$.
Grondfrequentie $\omega_0 = 2\pi/T = \pi$.

\textbf{Gemiddelde ($a_0$):}
Oppervlakte van de driehoek (basis 2, hoogte 1) is $\frac{1}{2} \cdot 2 \cdot 1 = 1$.
Gemiddelde over periode $T=2$: $a_0 = \frac{1}{T} \int_{-1}^1 x(t) dt = \frac{1}{2} \cdot 1 = 0.5$.

\textbf{Coëfficiënten ($a_n$):}
$a_n = \frac{2}{T} \int_{-1}^1 (1-|t|) \cos(n\pi t) dt$.
Wegens even symmetrie: $a_n = \frac{4}{T} \int_0^1 (1-t) \cos(n\pi t) dt = 2 \int_0^1 (1-t) \cos(n\pi t) dt$.

Integreer in delen: $\int u dv = uv - \int v du$.
Kies $u = 1-t \Rightarrow du = -dt$.
Kies $dv = \cos(n\pi t) dt \Rightarrow v = \frac{1}{n\pi} \sin(n\pi t)$.

Term 1: $\left[ (1-t) \frac{\sin(n\pi t)}{n\pi} \right]_0^1 = (0) - (1 \cdot 0) = 0$.
Term 2: $- \int_0^1 \frac{1}{n\pi} \sin(n\pi t) (-dt) = \frac{1}{n\pi} \int_0^1 \sin(n\pi t) dt$.
$= \frac{1}{n\pi} \left[ \frac{-\cos(n\pi t)}{n\pi} \right]_0^1 = \frac{1}{(n\pi)^2} ( -\cos(n\pi) + \cos(0) )$.

We weten dat $\cos(n\pi) = (-1)^n$ en $\cos(0) = 1$.
Dus: $a_n = 2 \cdot \frac{1 - (-1)^n}{(n\pi)^2}$.

\textbf{Evaluatie:}
Voor even $n$: $1 - 1 = 0 \Rightarrow a_n = 0$.
Voor oneven $n$: $1 - (-1) = 2 \Rightarrow a_n = \frac{4}{(n\pi)^2}$.

\textbf{Reeks:}
$x(t) = 0.5 + \sum_{n=1,3,5\dots}^{\infty} \frac{4}{(n\pi)^2} \cos(n\pi t)$.

\textbf{Opmerking:} De coëfficiënten nemen af met $1/n^2$, wat typisch is voor signalen zonder sprongen maar met knikken (continue functies), in tegenstelling tot de blokgolf ($1/n$).
\end{oefenblok}

\subsection{Oefening 7.5: Fourierreeks van een Blokpuls (Asymmetrisch)}

\begin{oefenblok}[Probleem]
Bepaal de Fourierreeks van de functie:
\[
f(x) = \begin{cases} 
1 & -\pi < x < 0 \\
0 & 0 < x < \pi 
\end{cases}
\]
met periode \(2\pi\).
\end{oefenblok}

\begin{oefenblok}[Oplossing]
\textbf{Stap 1: Bereken \(a_0\)}
\[
a_0 = \frac{1}{\pi} \int_{-\pi}^{\pi} f(x) dx = \frac{1}{\pi} \left( \int_{-\pi}^{0} 1 dx + \int_{0}^{\pi} 0 dx \right)
\]
\[
a_0 = \frac{1}{\pi} [x]_{-\pi}^0 = \frac{1}{\pi} (0 - (-\pi)) = 1
\]
De constante term in de reeks is \(a_0/2 = 1/2\).

\textbf{Stap 2: Bereken \(a_n\)}
\[
a_n = \frac{1}{\pi} \int_{-\pi}^{0} 1 \cdot \cos(nx) dx = \frac{1}{\pi} \left[ \frac{\sin(nx)}{n} \right]_{-\pi}^0 = 0
\]
(want \(\sin(0)=0\) en \(\sin(-n\pi)=0\)).

\textbf{Stap 3: Bereken \(b_n\)}
\[
b_n = \frac{1}{\pi} \int_{-\pi}^{0} 1 \cdot \sin(nx) dx = \frac{1}{\pi} \left[ -\frac{\cos(nx)}{n} \right]_{-\pi}^0
\]
\[
b_n = -\frac{1}{n\pi} (\cos(0) - \cos(-n\pi)) = -\frac{1}{n\pi} (1 - (-1)^n)
\]

Als \(n\) even is, is \((-1)^n = 1\), dus \(b_n = 0\).
Als \(n\) oneven is, is \((-1)^n = -1\), dus \(b_n = -\frac{1}{n\pi} (1 - (-1)) = -\frac{2}{n\pi}\).

De reeks is:
\[
f(x) = \frac{1}{2} - \frac{2}{\pi} \left( \sin(x) + \frac{1}{3}\sin(3x) + \frac{1}{5}\sin(5x) + \dots \right)
\]

\boxed{\text{Resultaat:}} \(f(x) = \frac{1}{2} - \frac{2}{\pi} \sum_{k=0}^{\infty} \frac{\sin((2k+1)x)}{2k+1}\).
\end{oefenblok}

\subsection{Oefening 7.6: Halfgolf-Gelijkrichter}

\emph{Bron: Elektronica toepassingen.}

\begin{oefenblok}[Probleem]
Een halfgolf-gelijkrichter produceert het signaal:
\[
f(t) = \begin{cases}
\sin(t), & 0 < t < \pi \\
0, & \pi < t < 2\pi
\end{cases}
\text{ met periode } T = 2\pi
\]

Bepaal de DC-component $a_0$ en de eerste twee harmonischen.
\end{oefenblok}

\begin{oefenblok}[Oplossing]
De grondfrequentie is $\omega_0 = \frac{2\pi}{T} = 1$ rad/s.

\vspace{0.3cm}
\textbf{Stap 1: Bereken de DC-component $a_0$}

\[
a_0 = \frac{1}{T} \int_0^T f(t) dt = \frac{1}{2\pi} \int_0^{\pi} \sin(t) dt
\]
\[
= \frac{1}{2\pi} \left[-\cos(t)\right]_0^{\pi} = \frac{1}{2\pi}(-\cos(\pi) + \cos(0))
\]
\[
= \frac{1}{2\pi}(1 + 1) = \frac{1}{\pi} \approx 0.318
\]

\boxed{a_0 = \frac{1}{\pi}}

\vspace{0.3cm}
\textbf{Stap 2: Bereken $a_1$ en $b_1$}

Voor $n = 1$:
\[
a_1 = \frac{2}{T} \int_0^{\pi} \sin(t) \cos(t) dt = \frac{1}{\pi} \int_0^{\pi} \sin(t) \cos(t) dt
\]

Gebruik $\sin(t)\cos(t) = \frac{1}{2}\sin(2t)$:
\[
a_1 = \frac{1}{2\pi} \int_0^{\pi} \sin(2t) dt = \frac{1}{2\pi} \left[-\frac{\cos(2t)}{2}\right]_0^{\pi} = 0
\]

Voor $b_1$:
\[
b_1 = \frac{1}{\pi} \int_0^{\pi} \sin(t) \sin(t) dt = \frac{1}{\pi} \int_0^{\pi} \sin^2(t) dt
\]

Gebruik $\sin^2(t) = \frac{1 - \cos(2t)}{2}$:
\[
b_1 = \frac{1}{2\pi} \int_0^{\pi} (1 - \cos(2t)) dt = \frac{1}{2\pi} \left[t - \frac{\sin(2t)}{2}\right]_0^{\pi}
\]
\[
= \frac{1}{2\pi}(\pi - 0) = \frac{1}{2}
\]

\boxed{b_1 = \frac{1}{2}}

\vspace{0.3cm}
\textbf{Stap 3: Bereken $a_2$ en $b_2$}

Voor $n = 2$:
\[
a_2 = \frac{1}{\pi} \int_0^{\pi} \sin(t) \cos(2t) dt
\]

Na uitwerking (productregel): $a_2 = -\frac{2}{3\pi}$

$b_2 = 0$ (want de integraal van $\sin(t)\sin(2t)$ over een volle periode is nul).

\vspace{0.3cm}
\textbf{Fourierreeks (eerste termen):}

\[
f(t) = \frac{1}{\pi} + \frac{1}{2}\sin(t) - \frac{2}{3\pi}\cos(2t) - \frac{2}{15\pi}\cos(4t) - \cdots
\]

\textbf{Interpretatie:} De DC-component $\frac{1}{\pi}$ is de gemiddelde waarde. De eerste harmonische ($\sin(t)$) domineert met amplitude 0.5. Hogere harmonischen ($\cos(2t), \cos(4t), \ldots$) zijn zwakker en zorgen voor de "rimpel" in de gelijkgerichte golf.
\end{oefenblok}

\subsection{Oefening 7.7: Symmetrie-eigenschappen en Convergentie}

\emph{Bron: Geavanceerde eigenschappen.}

\begin{oefenblok}[Probleem]
Beschouw de periodieke blokgolf met periode $T = 2$:
\[
f(t) = \begin{cases}
1, & 0 < t < 1 \\
-1, & 1 < t < 2
\end{cases}
\]

\begin{enumerate}[label=(\alph*)]
\item Bepaal welke symmetrie-eigenschap de functie heeft.
\item Leg uit welke Fouriercoëfficiënten nul zijn.
\item Evalueer de reeks op $t = 1$ (discontinuïteit). Wat voorspelt het fenomeen van Gibbs?
\end{enumerate}
\end{oefenblok}

\begin{oefenblok}[Oplossing]
\textbf{(a) Symmetrie-analyse}

Verschuif de functie zodat ze symmetrisch is rond $t = 0$:
\[
g(t) = \begin{cases}
1, & -0.5 < t < 0.5 \\
-1, & 0.5 < t < 1.5
\end{cases}
\]

Dit verschuiven verandert de Fouriercoëfficiënten niet in grootte, alleen in fase.

Controleer: $g(-t) = -g(t)$ $\Rightarrow$ de functie is \belangrijk{oneven}.

\boxed{\text{De functie is oneven (antisymmetrisch).}}

\vspace{0.3cm}
\textbf{(b) Consequenties voor coëfficiënten}

Voor een oneven functie geldt:
\begin{itemize}
\item $a_0 = 0$ (geen DC-component)
\item $a_n = 0$ voor alle $n$ (geen cosinus-termen)
\item Alleen $b_n \neq 0$ (enkel sinus-termen)
\end{itemize}

Bovendien, omdat de functie ook \textit{half-periode symmetrie} heeft (anti-symmetrie rond $t = T/2$), zijn alle \textit{even harmonischen} nul.

\boxed{\text{Alleen oneven harmonischen zijn aanwezig: } b_1, b_3, b_5, \ldots}

De Fourierreeks is:
\[
f(t) = \frac{4}{\pi}\left(\sin(\pi t) + \frac{1}{3}\sin(3\pi t) + \frac{1}{5}\sin(5\pi t) + \cdots\right)
\]

\vspace{0.3cm}
\textbf{(c) Gibbs fenomeen bij discontinuïteit}

Op $t = 1$ springt de functie van $+1$ naar $-1$.

De Fourierreeks convergeert naar het gemiddelde van de linker- en rechterlimieten:
\[
f_{\text{reeks}}(1) = \frac{f(1^-) + f(1^+)}{2} = \frac{1 + (-1)}{2} = 0
\]

\textbf{Gibbs fenomeen:} Vlakbij discontinuïteiten vertoont de eindige Fouriersom een "overshoot" van ongeveer 9\% boven de stapgrootte, ongeacht hoeveel termen je gebruikt. Deze rimpels verkleinen in breedte maar niet in hoogte.

\boxed{\text{De reeks convergeert naar 0 op } t=1, \text{ met Gibbs overshoot van } \sim 1.18 \text{ nabij de sprong.}}

\textbf{Praktische betekenis:} Bij het reconstrueren van blokgolven (bijv. digitale signalen) met een beperkt aantal harmonischen krijg je altijd deze "ringing" bij scherpe overgangen.
\end{oefenblok}

\subsection{Oefening 7.8: Dubbelzijdige Gelijkrichting}

\begin{oefenblok}[Probleem]
Een dubbelzijdige gelijkrichter (full-wave rectifier) zet een sinus om in zijn absolute waarde:
\[
f(t) = |\sin(t)|
\]
De periode is nu \(T = \pi\) (want de negatieve helft wordt positief geklapt).
Bepaal de Fourierreeks.
\end{oefenblok}

\begin{oefenblok}[Oplossing]
De grondfrequentie is \(\omega_0 = \frac{2\pi}{T} = \frac{2\pi}{\pi} = 2\).

De functie is even (\(f(-t) = |-\sin(t)| = |\sin(t)| = f(t)\)).
Dus \(b_n = 0\). We zoeken enkel \(a_n\).

\textbf{Stap 1: DC-component \(a_0\)}
\[
a_0 = \frac{1}{T} \int_0^T f(t) dt = \frac{1}{\pi} \int_0^{\pi} \sin(t) dt
\]
\[
= \frac{1}{\pi} [-\cos(t)]_0^{\pi} = \frac{1}{\pi} (1 - (-1)) = \frac{2}{\pi}
\]

\textbf{Stap 2: Coëfficiënten \(a_n\)}
\[
a_n = \frac{2}{T} \int_0^T f(t) \cos(n\omega_0 t) dt = \frac{2}{\pi} \int_0^{\pi} \sin(t) \cos(2nt) dt
\]
Gebruik de productformule: \(2\sin(A)\cos(B) = \sin(A+B) + \sin(A-B)\).
\[
a_n = \frac{1}{\pi} \int_0^{\pi} (\sin((1+2n)t) + \sin((1-2n)t)) dt
\]
\[
= \frac{1}{\pi} \left[ \frac{-\cos((1+2n)t)}{1+2n} - \frac{\cos((1-2n)t)}{1-2n} \right]_0^{\pi}
\]

Omdat \(1+2n\) en \(1-2n\) oneven zijn, is \(\cos((1\pm 2n)\pi) = -1\) en \(\cos(0) = 1\).
De term tussen haakjes wordt \(\frac{-(-1) - (-1)}{1+2n} + \dots = \frac{2}{1+2n} + \frac{2}{1-2n}\).

\[
a_n = \frac{1}{\pi} \left( \frac{2}{1+2n} + \frac{2}{1-2n} \right) = \frac{2}{\pi} \frac{(1-2n) + (1+2n)}{1 - 4n^2} = \frac{4}{\pi(1-4n^2)}
\]

\boxed{\text{Resultaat:}}
\[
f(t) = \frac{2}{\pi} - \frac{4}{\pi} \sum_{n=1}^{\infty} \frac{\cos(2nt)}{4n^2-1}
\]
\[
f(t) = \frac{2}{\pi} - \frac{4}{\pi} \left( \frac{\cos(2t)}{3} + \frac{\cos(4t)}{15} + \frac{\cos(6t)}{35} + \dots \right)
\]
\end{oefenblok}

\subsection{Oefening 7.9: Sinusgolf met Enkelvoudige Gelijkrichting}

\begin{oefenblok}[Probleem]
Een ideale diode in een circuit zorgt voor 'halve-golf gelijkrichting'. Het signaal is \(x(t) = \sin(t)\) voor \(0 \le t < \pi\) en \(x(t) = 0\) voor \(\pi \le t < 2\pi\). De periode is \(T=2\pi\).
Bepaal de Fouriercoëfficiënt \(a_0\) (de gemiddelde waarde of DC-component).

\begin{center}
\begin{tikzpicture}[scale=0.8]
    \begin{axis}[
        width=10cm, height=4cm,
        axis lines=middle,
        xlabel={$t$}, ylabel={$x(t)$},
        domain=-1:7, samples=200,
        ymin=-1.5, ymax=1.5,
        xtick={0, 3.14, 6.28},
        xticklabels={0, $\pi$, $2\pi$},
        grid=both
    ]
    \addplot[thick, blue] {sin(deg(x)) * (sin(deg(x))>0)};
    \end{axis}
\end{tikzpicture}
\end{center}
\end{oefenblok}

\begin{oefenblok}[Oplossing]
De grondfrequentie is \(\omega_0 = 2\pi / T = 1\).
\[
a_0 = \frac{1}{T} \int_0^T x(t) dt = \frac{1}{2\pi} \left( \int_0^{\pi} \sin(t) dt + \int_{\pi}^{2\pi} 0 dt \right)
\]
\[
a_0 = \frac{1}{2\pi} \left[ -\cos(t) \right]_0^{\pi} = \frac{1}{2\pi} (-\cos(\pi) - (-\cos(0)))
\]
\[
a_0 = \frac{1}{2\pi} (-(-1) - (-1)) = \frac{1}{2\pi} (1 + 1) = \frac{2}{2\pi} = \frac{1}{\pi}
\]
De gemiddelde waarde van een enkelzijdig gelijkgerichte sinus is dus \(1/\pi \approx 0.318\) van de piekamplitude.
\end{oefenblok}

\subsection{Oefening 7.10: Directe Identificatie van Coëfficiënten}

\begin{oefenblok}[Probleem]
Gegeven is het signaal:
\[
f(t) = 5 + 2\cos(3t) - 4\sin(6t)
\]
Bepaal zonder integralen te berekenen:
\begin{enumerate}[label=(\alph*)]
    \item De grondfrequentie $\omega_0$ en de periode $T$.
    \item De Fouriercoëfficiënten $a_0, a_n$ en $b_n$.
\end{enumerate}
\end{oefenblok}

\begin{oefenblok}[Oplossing]
\textbf{(a) Frequentie en Periode}
De termen zijn constant (DC), $\cos(3t)$ en $\sin(6t)$.
De frequenties zijn $0$, $3$ en $6$ rad/s.
De grootste gemene deler (GGD) van de frequenties is de grondfrequentie:
\[
\omega_0 = 3 \text{ rad/s}
\]
De periode is:
\[
T = \frac{2\pi}{\omega_0} = \frac{2\pi}{3} \text{ s}
\]

\textbf{(b) Fouriercoëfficiënten}
De Fourierreeks is gedefinieerd als:
\[
f(t) = a_0 + \sum_{n=1}^{\infty} (a_n \cos(n\omega_0 t) + b_n \sin(n\omega_0 t))
\]
Met $\omega_0 = 3$, worden de termen:
\begin{itemize}
    \item $n=0$ (DC): $a_0 = 5$.
    \item $n=1$ ($1 \cdot 3t$): $2\cos(3t) \Rightarrow a_1 = 2, b_1 = 0$.
    \item $n=2$ ($2 \cdot 3t = 6t$): $-4\sin(6t) \Rightarrow a_2 = 0, b_2 = -4$.
    \item Alle andere $n > 2$: $a_n = 0, b_n = 0$.
\end{itemize}

\boxed{a_0 = 5, \quad a_1 = 2, \quad b_2 = -4, \quad \text{alle andere nul.}}
\end{oefenblok}

\subsection{Oefening 7.11: Symmetrie Herkenning}

\begin{oefenblok}[Probleem]
Bekijk de volgende twee signalen (beide met periode $T=2\pi$):
\begin{enumerate}[label=(\alph*)]
    \item $x(t) = t^2$ voor $-\pi < t < \pi$ (periodiek herhaald).
    \item $y(t) = t$ voor $-\pi < t < \pi$ (periodiek herhaald, zaagtand).
\end{enumerate}
Bepaal voor elk signaal op basis van symmetrie welke Fouriercoëfficiënten ($a_n$ of $b_n$) nul zijn.
\end{oefenblok}

\begin{oefenblok}[Oplossing]
\textbf{(a) Signaal $x(t) = t^2$}
We controleren de symmetrie:
\[
x(-t) = (-t)^2 = t^2 = x(t)
\]
De functie is \belangrijk{even} (spiegelsymmetrie om de y-as).
Voor even functies geldt:
\begin{itemize}
    \item De sinus-termen vallen weg (want sinus is oneven).
    \item $b_n = 0$ voor alle $n$.
\end{itemize}
De reeks bevat alleen cosinus-termen en eventueel een DC-component ($a_n$).

\textbf{(b) Signaal $y(t) = t$}
We controleren de symmetrie:
\[
y(-t) = (-t) = -t = -y(t)
\]
De functie is \belangrijk{oneven} (puntsymmetrie om de oorsprong).
Voor oneven functies geldt:
\begin{itemize}
    \item De cosinus-termen en de DC-component vallen weg (want cosinus en constante zijn even).
    \item $a_0 = 0$ en $a_n = 0$ voor alle $n$.
\end{itemize}
De reeks bevat alleen sinus-termen ($b_n$).
\end{oefenblok}

\chapter{Lineaire Tijdsinvariante Continue (LTC) systemen}

\section{Systeembeschrijving: De Heilige Drie-eenheid}

LTC-systemen vormen de ruggengraat van de regeltechniek en signaalverwerking. In dit hoofdstuk brengen we alles samen: differentiaalvergelijkingen, Laplace en convolutie. Om deze systemen echt te begrijpen, moeten we kijken naar drie fundamentele concepten: Eigenfuncties, Domeinen en Polen.

\subsection{1. Eigenfuncties: De Link met Lineaire Algebra}
In de lineaire algebra is een \textbf{eigenvector} $v$ van een matrix $A$ een vector die niet van richting verandert, maar enkel wordt geschaald: $Av = \lambda v$.

Voor LTI-systemen geldt een analoog concept. De \textbf{eigenfuncties} zijn de complexe e-machten $e^{st}$.
Als we een signaal $x(t) = e^{st}$ in een LTI-systeem sturen, is de output altijd een geschaalde versie van diezelfde e-macht:
\[
\text{Input: } e^{st} \quad \longrightarrow \quad \text{Systeem} \quad \longrightarrow \quad \text{Output: } H(s) \cdot e^{st}
\]
Hierbij is de schaalfactor $H(s)$ de \textbf{eigenwaarde}, die we de \textbf{Transferfunctie} noemen.
Dit is waarom we Laplace en Fourier gebruiken: ze ontbinden ingewikkelde signalen in deze "basis-bouwblokken" ($e^{st}$), die het systeem simpelweg één voor één schaalt.

\subsection{2. De Drie Domeinen}
We kunnen een systeem vanuit drie perspectieven bekijken, die elk een deel van de waarheid onthullen:

\begin{itemize}
    \item \textbf{Tijdsdomein ($t$): De Realiteit.}
    Hier leven de differentiaalvergelijkingen ($y'' + ay' + by = x$).
    De systeemkarakteristiek is de \textbf{Impulsrespons $h(t)$}.
    \emph{Operatie:} Convolutie $y(t) = x(t) * h(t)$.
    \emph{Nadeel:} Rekentechnisch zwaar (integralen).

    \item \textbf{Laplace-domein ($s = \sigma + j\omega$): Het Totaalplaatje.}
    Hier wordt calculus (differentiatie) omgezet in algebra (vermenigvuldiging).
    De systeemkarakteristiek is de \textbf{Transferfunctie $H(s)$}.
    \emph{Operatie:} Vermenigvuldiging $Y(s) = H(s) \cdot X(s)$.
    \emph{Kracht:} Analyseert zowel stabiliteit ($\sigma$) als frequentiegedrag ($\omega$).

    \item \textbf{Frequentiedomein ($s = j\omega$): De Permanente Toestand.}
    Dit is een "snede" door het Laplace-domein op de imaginaire as ($\sigma=0$).
    Het beschrijft hoe het systeem reageert op sinusgolven die al oneindig lang aan staan (geen inschakelverschijnselen).
    \emph{Link:} $H(j\omega)$ is de Frequentierespons (Bode-plot).
\end{itemize}

\subsection{3. Polen en Nulpunten: Het DNA van het Systeem}
De transferfunctie is meestal een breuk van polynomen:
\[
H(s) = \frac{N(s)}{D(s)} = K \frac{(s-z_1)(s-z_2)\dots}{(s-p_1)(s-p_2)\dots}
\]

\textbf{De Polen ($p_i$):}
Dit zijn de waarden van $s$ waarvoor de noemer $D(s)$ nul wordt ($H(s) \to \infty$).
\begin{itemize}
    \item \textbf{Wiskundig:} Het zijn de wortels van de karakteristieke vergelijking uit Hoofdstuk 2 ($\lambda$).
    \item \textbf{Fysisch:} Het zijn de \emph{natuurlijke frequenties} of \emph{resonantiemodi} van het systeem. Als je het systeem hier exciteert, krijg je een enorme respons (resonantie).
    \item \textbf{Stabiliteit:} De reële component ($\sigma$) van de pool bepaalt de demping.
    \begin{itemize}
        \item $\text{Re}(p) < 0$ (Linker halfvlak): Demping (Stabiel).
        \item $\text{Re}(p) > 0$ (Rechter halfvlak): Opblazen (Instabiel).
        \item $\text{Re}(p) = 0$ (Imaginaire as): Oscillatie (Marginaal stabiel).
    \end{itemize}
\end{itemize}

\textbf{De Nulpunten ($z_i$):}
Dit zijn de waarden waarvoor de teller $N(s)$ nul wordt ($H(s) = 0$).
\begin{itemize}
    \item \textbf{Fysisch:} Dit zijn frequenties die door het systeem worden \emph{geblokkeerd} of volledig uitgedoofd.
\end{itemize}

\section{Oefeningen: Systeemanalyse}

\subsection{Oefening 8.1: Analyse van Impulsrespons}

\begin{oefenblok}[Probleem]
Gegeven een systeem met \(h(t)=2e^{-5t}u(t)\).
\begin{enumerate}
    \item Bepaal de transferfunctie en stabiliteit.
    \item Bepaal de staprespons.
\end{enumerate}
\end{oefenblok}

\begin{oefenblok}[Oplossing]
\textbf{1. Transferfunctie:}
\[
H(s) = \mathcal{L}\{2e^{-5t}\} = \frac{2}{s+5}
\]
De pool ligt op \(s=-5\). Omdat \(-5 < 0\), is het systeem \textbf{stabiel}.

\textbf{2. Staprespons:}
De input is een stap \(u(t)\), dus \(X(s) = 1/s\).
\[
Y(s) = H(s)X(s) = \frac{2}{s(s+5)}
\]
Breuksplitsing:
\[
\frac{2}{s(s+5)} = \frac{A}{s} + \frac{B}{s+5}
\]
\(A = 2/5 = 0.4\), \(B = -2/5 = -0.4\).
\[
Y(s) = \frac{0.4}{s} - \frac{0.4}{s+5}
\]
Terugtransformeren:
\[
y(t) = 0.4(1 - e^{-5t})u(t)
\]
Dit beschrijft een exponentiële stijging naar een eindwaarde van 0.4. Dit is het typische gedrag van een eerste-orde laagdoorlaatfilter.
\end{oefenblok}

\subsection{Oefening 8.2: Numerieke Analyse van een Massa-Veer-Demper}

\begin{oefenblok}[Probleem]
We beschouwen een specifiek mechanisch systeem met de volgende parameters:
\begin{itemize}
    \item Massa \(m=5\) kg
    \item Veerconstante \(k=50\) N/m
    \item Demping \(c=10\) Ns/m
\end{itemize}
Het systeem wordt beschreven door de differentiaalvergelijking:
\[
5y'' + 10y' + 50y = f(t)
\]
Na deling door \(m\) krijgen we de genormaliseerde vorm:
\[
y'' + 2y' + 10y = 0.2f(t)
\]
De inputkracht is gegeven als \(f(t)=2.6\sin(2t)\). De beginvoorwaarden zijn \(y(0)=0.02\) m en \(y'(0)=0\) m/s.
De rechterlid term wordt dan \(0.2 \cdot 2.6\sin(2t) = 0.52\sin(2t)\).
\end{oefenblok}

\begin{oefenblok}[Oplossing]
\textbf{Homogene oplossing:}
Karakteristieke vergelijking: \(\lambda^2 + 2\lambda + 10 = 0\).
Discriminant \(D = 4 - 40 = -36\).
Wortels: \(\lambda = \frac{-2 \pm 6j}{2} = -1 \pm 3j\).
\[
y_h(t) = e^{-t}(C_1 \cos 3t + C_2 \sin 3t)
\]
Het systeem is ondergedempt met een eigenfrequentie van 3 rad/s.

\textbf{Particuliere oplossing:}
Stel \(y_p(t) = A\sin 2t + B\cos 2t\).
Differentieer tweemaal:
\[
y_p' = 2A\cos 2t - 2B\sin 2t
\]
\[
y_p'' = -4A\sin 2t - 4B\cos 2t
\]
Invullen in \(y'' + 2y' + 10y = 0.52\sin 2t\):
\[
(-4A\sin 2t - 4B\cos 2t) + 2(2A\cos 2t - 2B\sin 2t) + 10(A\sin 2t + B\cos 2t) = 0.52\sin 2t
\]
Groeperen per sinus en cosinus:
\begin{itemize}
    \item \(\sin 2t\): \(-4A - 4B + 10A = 6A - 4B = 0.52\)
    \item \(\cos 2t\): \(-4B + 4A + 10B = 4A + 6B = 0\)
\end{itemize}
Uit de tweede vgl volgt \(4A = -6B \Rightarrow A = -1.5B\).
Invullen in de eerste:
\[
6(-1.5B) - 4B = -9B - 4B = -13B = 0.52 \Rightarrow B = -0.04
\]
Dan is \(A = -1.5(-0.04) = 0.06\).
\[
y_p(t) = 0.06\sin 2t - 0.04\cos 2t
\]

\textbf{Algemene oplossing en constanten:}
\[
y(t) = e^{-t}(C_1 \cos 3t + C_2 \sin 3t) + 0.06\sin 2t - 0.04\cos 2t
\]
Beginvoorwaarde \(y(0) = 0.02\):
\[
C_1 - 0.04 = 0.02 \Rightarrow C_1 = 0.06
\]
Afgeleide \(y'(t)\):
\[
y'(t) = -e^{-t}(C_1 \cos 3t + C_2 \sin 3t) + e^{-t}(-3C_1 \sin 3t + 3C_2 \cos 3t) + 0.12\cos 2t + 0.08\sin 2t
\]
Beginvoorwaarde \(y'(0) = 0\):
\[
-C_1 + 3C_2 + 0.12 = 0 \Rightarrow -0.06 + 3C_2 + 0.12 = 0 \Rightarrow 3C_2 = -0.06 \Rightarrow C_2 = -0.02
\]

\textbf{Eindresultaat:}
\[
y(t) = e^{-t}(0.06\cos 3t - 0.02\sin 3t) + 0.06\sin 2t - 0.04\cos 2t
\]
\end{oefenblok}

\subsection{Oefening 8.3: Convolutie met een Stapfunctie}

\begin{oefenblok}[Probleem]
Bereken de convolutie \(y(t) = f(t) * g(t)\) waarbij \(f(t) = e^{-3t}u(t)\) en \(g(t) = u(t)\).
\end{oefenblok}

\begin{oefenblok}[Oplossing]
De convolutie met een stapfunctie komt neer op integratie.
\[
y(t) = \int_{-\infty}^{\infty} e^{-3\tau} u(\tau) u(t-\tau) d\tau
\]
De termen \(u(\tau)\) en \(u(t-\tau)\) beperken het integratie-interval tot \(0 < \tau < t\). (Als \(t < 0\) is de integraal 0).
\[
y(t) = \int_{0}^{t} e^{-3\tau} d\tau = \left[ \frac{-1}{3} e^{-3\tau} \right]_0^t
\]
\[
= \frac{-1}{3} (e^{-3t} - 1) = \frac{1}{3} (1 - e^{-3t})u(t)
\]
Dit bevestigt dat de staprespons van een systeem met transferfunctie \(1/(s+3)\) een exponentiële curve is.
\end{oefenblok}

\subsection{Oefening 8.4: Frequentierespons en Bode Diagram}

\emph{Bron: Frequentie-analyse.}

\begin{oefenblok}[Probleem]
Een systeem heeft overdracht $H(s) = \frac{20}{s+10}$.

\begin{enumerate}[label=(\alph*)]
\item Bepaal de frequentierespons $H(j\omega)$.
\item Bereken de magnitude in dB voor $\omega = 1, 10, 100$ rad/s.
\item Bepaal de 3dB knakfrequentie.
\end{enumerate}
\end{oefenblok}

\begin{oefenblok}[Oplossing]
\textbf{(a) Frequentierespons}

Vervang $s$ door $j\omega$:
\[
H(j\omega) = \frac{20}{j\omega + 10} = \frac{20}{10(1 + j\omega/10)} = \frac{2}{1 + j\omega/10}
\]

\vspace{0.3cm}
\textbf{(b) Magnitude in dB}

De magnitude is:
\[
|H(j\omega)| = \frac{2}{\sqrt{1 + (\omega/10)^2}}
\]

In dB:
\[
|H(j\omega)|_{\text{dB}} = 20\log_{10}|H(j\omega)| = 20\log_{10}(2) - 10\log_{10}(1 + (\omega/10)^2)
\]

Voor $\omega = 1$ rad/s:
\[
|H(j1)| = \frac{2}{\sqrt{1 + 0.01}} \approx 1.99 \quad \Rightarrow \quad |H|_{\text{dB}} \approx 6.0 \text{ dB}
\]

Voor $\omega = 10$ rad/s (knakfrequentie):
\[
|H(j10)| = \frac{2}{\sqrt{1 + 1}} = \frac{2}{\sqrt{2}} = \sqrt{2} \approx 1.41 \quad \Rightarrow \quad |H|_{\text{dB}} \approx 3.0 \text{ dB}
\]

Voor $\omega = 100$ rad/s:
\[
|H(j100)| = \frac{2}{\sqrt{1 + 100}} \approx 0.199 \quad \Rightarrow \quad |H|_{\text{dB}} \approx -14.0 \text{ dB}
\]

\vspace{0.3cm}
\textbf{(c) 3dB knakfrequentie}

De DC-versterking is $|H(0)| = 2$ (of 6 dB).

De 3dB knakfrequentie is waar $|H(j\omega)| = \frac{|H(0)|}{\sqrt{2}}$:
\[
\frac{2}{\sqrt{1 + (\omega_c/10)^2}} = \frac{2}{\sqrt{2}}
\]
\[
\sqrt{1 + (\omega_c/10)^2} = \sqrt{2} \quad \Rightarrow \quad 1 + (\omega_c/10)^2 = 2
\]
\[
(\omega_c/10)^2 = 1 \quad \Rightarrow \quad \omega_c = 10 \text{ rad/s}
\]

\boxed{\text{Knakfrequentie: } \omega_c = 10 \text{ rad/s of } f_c = \frac{10}{2\pi} \approx 1.59 \text{ Hz}}

\begin{center}
\begin{tikzpicture}
    \begin{semilogxaxis}[
        width=10cm, height=6cm,
        xlabel={Frequentie $\omega$ [rad/s]},
        ylabel={Magnitude [dB]},
        grid=both,
        xmin=0.1, xmax=1000,
        ymin=-40, ymax=10,
        title={Bode Magnitude Plot}
    ]
    \addplot[thick, blue, domain=0.1:1000, samples=100] {20*log10(2/sqrt(1+(x/10)^2))};
    \addplot[dashed, red, domain=0.1:10] {6};
    \addplot[dashed, red, domain=10:1000] {6 - 20*log10(x/10)};
    \end{semilogxaxis}
\end{tikzpicture}
\end{center}

\textbf{Conclusie:} Dit is een eerste-orde laagdoorlaatfilter. Boven de knakfrequentie rolt de magnitude af met -20 dB/decade.
\end{oefenblok}

\subsection{Oefening 8.5: Classificatie van Filters}

\begin{oefenblok}[Probleem]
Classificeer de volgende systemen als Laagdoorlaat (LP), Hoogdoorlaat (HP), Banddoorlaat (BP), Bandstop (BS) of All-pass (AP). Bepaal ook de orde van het systeem.
\begin{enumerate}
    \item \(H(s) = \frac{8s}{s^2 + 8s + 25}\)
    \item \(H(s) = \frac{s^2 + 16}{s^2 + 8s + 16}\)
    \item \(H(s) = \frac{s^2}{s^2 + 4s + 16}\)
    \item \(H(s) = \frac{s^2 - 8s + 16}{s^2 + 8s + 16}\)
\end{enumerate}
\end{oefenblok}

\begin{oefenblok}[Oplossing]
We analyseren het gedrag bij \(s \to 0\) (DC) en \(s \to \infty\) (hoge frequenties).

\textbf{1. Banddoorlaat (BP)}
\[ H(s) = \frac{8s}{s^2 + 8s + 25} \]
\begin{itemize}
    \item \(s \to 0 \Rightarrow H(0) = 0\) (blokkeert DC).
    \item \(s \to \infty \Rightarrow H(s) \approx 8s/s^2 \to 0\) (blokkeert hoge frequenties).
    \item Tussenin is er transmissie. Dit is een \textbf{2e orde Bandpass filter}.
\end{itemize}

\textbf{2. Bandstop (BS) / Notch}
\[ H(s) = \frac{s^2 + 16}{s^2 + 8s + 16} \]
\begin{itemize}
    \item \(s \to 0 \Rightarrow H(0) = 16/16 = 1\).
    \item \(s \to \infty \Rightarrow H(s) \approx s^2/s^2 = 1\).
    \item Bij \(s = j4\) (dus \(\omega = 4\)) wordt de teller \( (j4)^2 + 16 = -16 + 16 = 0\).
    \item Het filter blokkeert specifiek de frequentie \(\omega = 4\). Dit is een \textbf{2e orde Bandstop filter}.
\end{itemize}

\textbf{3. Hoogdoorlaat (HP)}
\[ H(s) = \frac{s^2}{s^2 + 4s + 16} \]
\begin{itemize}
    \item \(s \to 0 \Rightarrow H(0) = 0\) (blokkeert DC).
    \item \(s \to \infty \Rightarrow H(s) \approx s^2/s^2 = 1\) (laat hoge frequenties door).
    \item Dit is een \textbf{2e orde Highpass filter}.
\end{itemize}

\textbf{4. All-pass (AP)}
\[ H(s) = \frac{s^2 - 8s + 16}{s^2 + 8s + 16} = \frac{(s-4)^2}{(s+4)^2} \]
\begin{itemize}
    \item De polen liggen op \(-4\) (dubbel), de nulpunten op \(+4\) (dubbel).
    \item Voor \(s=j\omega\) is \(|H(j\omega)| = \frac{|j\omega - 4|^2}{|j\omega + 4|^2} = 1\) voor alle \(\omega\).
    \item De magnitude is constant 1, maar de fase verandert. Dit is een \textbf{2e orde All-pass filter}.
\end{itemize}
\end{oefenblok}

\subsection{Oefening 8.6: Cascade Systemen}

\emph{Bron: Systeemcombinaties.}

\begin{oefenblok}[Probleem]
Gegeven twee LTC-systemen in cascade:
\[
H_1(s) = \frac{5}{s+2}, \quad H_2(s) = \frac{3}{s+3}
\]

Bepaal:
\begin{enumerate}[label=(\alph*)]
\item De totale overdracht $H(s) = H_1(s) \cdot H_2(s)$
\item De impulsrespons $h(t)$ van het gecombineerde systeem
\item Is het systeem BIBO-stabiel?
\end{enumerate}
\end{oefenblok}

\begin{oefenblok}[Oplossing]
\textbf{(a) Totale overdracht}

Voor systemen in cascade geldt:
\[
H(s) = H_1(s) \cdot H_2(s) = \frac{5}{s+2} \cdot \frac{3}{s+3} = \frac{15}{(s+2)(s+3)}
\]

\boxed{H(s) = \frac{15}{(s+2)(s+3)}}

\vspace{0.3cm}
\textbf{(b) Impulsrespons}

Partieelbreuken:
\[
H(s) = \frac{15}{(s+2)(s+3)} = \frac{A}{s+2} + \frac{B}{s+3}
\]

Voor $s = -2$: $15 = A(1) \Rightarrow A = 15$

Voor $s = -3$: $15 = B(-1) \Rightarrow B = -15$

Dus:
\[
H(s) = \frac{15}{s+2} - \frac{15}{s+3}
\]

Inverse Laplace:
\[
\boxed{h(t) = 15(e^{-2t} - e^{-3t})u(t)}
\]

\vspace{0.3cm}
\textbf{(c) Stabiliteit}

De polen zijn $s = -2$ en $s = -3$, beide in het linker halfvlak.

\boxed{\text{Het systeem is BIBO-stabiel.}}

\textbf{Verificatie:} De impulsrespons $h(t) \to 0$ als $t \to \infty$, en de integraal $\int_0^{\infty} |h(t)| dt$ is eindig (kenmerk van stabiliteit).

\textbf{Fysische interpretatie:} Twee laagdoorlaatfilters in serie geven een sterkere demping bij hoge frequenties. De totale fase-verschuiving is de som van beide filters.
\end{oefenblok}

\subsection{Oefening 8.7: Tweede-Orde Resonantie}

\emph{Bron: Trillingsanalyse.}

\begin{oefenblok}[Probleem]
Een tweede-orde systeem heeft overdracht:
\[
H(s) = \frac{\omega_n^2}{s^2 + 2\zeta\omega_n s + \omega_n^2}
\]
met $\omega_n = 5$ rad/s en $\zeta = 0.2$ (ondergedempt).

Bepaal de resonantiefrequentie en de Q-factor.
\end{oefenblok}

\begin{oefenblok}[Oplossing]
\textbf{Resonantiefrequentie}

Voor een ondergedempt systeem ($\zeta < 1/\sqrt{2} \approx 0.707$) is de resonantiefrequentie:
\[
\omega_r = \omega_n\sqrt{1 - 2\zeta^2}
\]

Invullen:
\[
\omega_r = 5\sqrt{1 - 2(0.2)^2} = 5\sqrt{1 - 0.08} = 5\sqrt{0.92} \approx 4.80 \text{ rad/s}
\]

\boxed{\omega_r \approx 4.80 \text{ rad/s}}

\vspace{0.3cm}
\textbf{Q-factor}

De kwaliteitsfactor is:
\[
Q = \frac{1}{2\zeta} = \frac{1}{2(0.2)} = 2.5
\]

\boxed{Q = 2.5}

\vspace{0.3cm}
\textbf{Amplitude bij resonantie}

Bij $\omega = \omega_r$ bereikt $|H(j\omega)|$ zijn maximum:
\[
|H(j\omega_r)| = \frac{1}{2\zeta\sqrt{1-\zeta^2}} \approx \frac{1}{2(0.2)\sqrt{0.96}} \approx 2.55
\]

\textbf{Interpretatie:}
\begin{itemize}
\item Een lage $\zeta$ (0.2) betekent weinig demping, dus een scherpe piek bij resonantie.
\item De Q-factor van 2.5 geeft aan dat het systeem energie kan opslaan (maar niet extreem).
\item Bij mechanische systemen zou dit corresponderen met een licht gedempte massa-veer oscillator.
\item In filters: een bandpass filter met redelijk scherpe selectiviteit.
\end{itemize}
\end{oefenblok}

\chapter{Eigenwaarden en eigenvectoren}

\section{De Toestandsruimte (State-Space) en Eigenwaarden}

Bij complexe systemen, en in het bijzonder bij MIMO-systemen (Multiple Input Multiple Output), biedt de klassieke transferfunctie \(H(s)\) slechts een beperkt beeld (enkel input-output). De \textbf{toestandsruimtebeschrijving} kijkt ook naar het inwendige gedrag via interne variabelen, de zogenaamde \emph{toestanden} \(\mathbf{x}(t)\).

Het systeem wordt beschreven door twee matrixvergelijkingen:
\[
\begin{cases}
\dot{\mathbf{x}}(t) = A\mathbf{x}(t) + B\mathbf{u}(t) & \text{(Dynamica van het systeem)} \\
\mathbf{y}(t) = C\mathbf{x}(t) + D\mathbf{u}(t) & \text{(Output/Meting)}
\end{cases}
\]
Hierbij is \(A\) de \textbf{systeemmatrix}. Deze matrix bevat alle informatie over de interne dynamiek.

\subsection*{De rol van Eigenwaarden en Eigenvectoren}
De analyse van matrix \(A\) is cruciaal. Waarom? Omdat matrixvermenigvuldiging normaal gesproken vectoren roteert en schaalt, wat de analyse van gekoppelde differentiaalvergelijkingen moeilijk maakt.

Eigenvectoren zijn speciale richtingen in de ruimte waarvoor de matrixvermenigvuldiging zich gedraagt als een simpele scalaire vermenigvuldiging.
\[
A \mathbf{v} = \lambda \mathbf{v}
\]
Dit betekent dat als de toestand van het systeem zich langs een eigenvector bevindt, de verandering van de toestand ook in diezelfde richting ligt. De differentiaalvergelijkingen worden \concept{ontkoppeld}.

\begin{itemize}
    \item \textbf{Eigenwaarden (\(\lambda\)):} De eigenwaarden van \(A\) zijn identiek aan de \emph{polen} van de transferfunctie. Ze bepalen de \textbf{stabiliteit} (reëel deel \(<0\)) en de \textbf{snelheid/oscillatie} van de respons (imaginair deel).
    \item \textbf{Eigenvectoren (\(\mathbf{v}\)):} Een eigenvector \(\mathbf{v}\) definieert een specifieke \emph{richting} in de toestandsruimte. Dit noemen we een \textbf{natuurlijke modus} van het systeem.
\end{itemize}

Als we het systeem bekijken in de basis van zijn eigenvectoren (diagonalisatie), valt het complexe systeem uiteen in een set van onafhankelijke, simpele 1e-orde systemen.

\section{Oefeningen: Matrixanalyse en Eigenstructuur}

\subsection{Oefening 9.1: Verificatie van Eigenvectoren (3x3 Matrix)}

\begin{oefenblok}[Probleem]
Gegeven de matrix:
\[
A = \begin{pmatrix} 8 & 2 & 2 \\ -3 & 15 & 6 \\ -2 & 4 & 13 \end{pmatrix}
\]
Onderzoek of de vector \(\mathbf{v} = \begin{pmatrix} 0 \\ -1 \\ 1 \end{pmatrix}\) een eigenvector is.
\end{oefenblok}

\begin{oefenblok}[Oplossing]
We vermenigvuldigen \(A\) met \(\mathbf{v}\):
\[
A\mathbf{v} = \begin{pmatrix} 8 & 2 & 2 \\ -3 & 15 & 6 \\ -2 & 4 & 13 \end{pmatrix} \begin{pmatrix} 0 \\ -1 \\ 1 \end{pmatrix}
= \begin{pmatrix} 0 - 2 + 2 \\ 0 - 15 + 6 \\ 0 - 4 + 13 \end{pmatrix}
= \begin{pmatrix} 0 \\ -9 \\ 9 \end{pmatrix}
\]
We zien dat de resultaatvector geschreven kan worden als \(9 \cdot \begin{pmatrix} 0 \\ -1 \\ 1 \end{pmatrix}\).

\boxed{\text{Conclusie:}} Ja, \(\mathbf{v}\) is een eigenvector met eigenwaarde \(\lambda = 9\).
\end{oefenblok}

\subsection{Oefening 9.2: Eigenfuncties van Operatoren}

\begin{oefenblok}[Theorie]
Het concept van eigenwaarden beperkt zich niet tot matrices. Ook differentiaaloperatoren hebben eigenfuncties.

\textbf{Operator \(L = \frac{d}{dt}\):}
Zoek een functie \(x(t)\) zodat \(\frac{d}{dt}x(t) = \lambda x(t)\).
De oplossing is \(x(t) = e^{\lambda t}\). Dit is de reden waarom exponentiële functies zo belangrijk zijn in lineaire systemen.

\textbf{Operator \(L = \frac{d^2}{dt^2}\):}
Zoek functies waarvoor \(x''(t) = \lambda x(t)\).
Stel \(\lambda = -\omega^2\) (een negatief getal). Dan zijn \(x(t) = \sin(\omega t)\) en \(x(t) = \cos(\omega t)\) eigenfuncties, want:
\[
(\sin \omega t)'' = -\omega^2 \sin \omega t
\]
Dit verklaart waarom sinusoïden hun vorm behouden in lineaire systemen.
\end{oefenblok}

\section{Casus: MIMO Systemen (Gekoppelde Massa's)}

\subsection{Oefening 9.3: Gekoppelde Massa's}

\begin{oefenblok}[Probleemstelling]
Beschouw een systeem met twee massa's \(m\), verbonden door twee veren met stijfheid \(k\). Er is geen demping.
\begin{itemize}
    \item Inputs: Externe krachten \(f_1(t)\) op massa 1, \(f_2(t)\) op massa 2, en een positieverstoring \(x(t)\) aan de muur.
    \item Outputs: De posities \(y_1(t)\) en \(y_2(t)\).
\end{itemize}
\end{oefenblok}

\begin{oefenblok}[Oplossing]
\textbf{Stap 1: Differentiaalvergelijkingen (ODEs)}
We stellen de bewegingsvergelijkingen op volgens Newton (\(F=ma\)).
Voor massa 1 (krachten van veer links, veer rechts, en extern):
\[
my_1'' = -k(y_1 - x) + k(y_2 - y_1) + f_1
\]
Voor massa 2 (kracht van veer links en extern):
\[
my_2'' = -k(y_2 - y_1) + f_2
\]

\textbf{Stap 2: Matrixvorm (Toestandsruimte/MIMO)}
Herschik de termen om inputs en outputs te scheiden:
\[
my_1'' + 2ky_1 - ky_2 = kx + f_1
\]
\[
my_2'' - ky_1 + ky_2 = f_2
\]
In matrixvorm \(M\ddot{\mathbf{y}} + K\mathbf{y} = F\):
\[
\begin{pmatrix} m & 0 \\ 0 & m \end{pmatrix} \begin{pmatrix} y_1'' \\ y_2'' \end{pmatrix} + \begin{pmatrix} 2k & -k \\ -k & k \end{pmatrix} \begin{pmatrix} y_1 \\ y_2 \end{pmatrix} = \begin{pmatrix} kx + f_1 \\ f_2 \end{pmatrix}
\]

\textbf{Stap 3: Oplossing in s-domein (Transfermatrix)}
Neem de Laplace-transformatie (neem aan dat beginvoorwaarden nul zijn):
\[
ms^2 Y_1 + 2kY_1 - kY_2 = kX + F_1
\]
\[
ms^2 Y_2 - kY_1 + kY_2 = F_2
\]
Uit de tweede vergelijking isoleren we \(Y_1\):
\[
(ms^2 + k)Y_2 = kY_1 + F_2 \Rightarrow Y_1 = \frac{ms^2 + k}{k} Y_2 - \frac{1}{k} F_2
\]
Dit substitueren we in de eerste vergelijking om de transferfunctie voor \(Y_2\) te vinden. Na algebraïsche manipulatie verkrijgt men een noemer van de vierde orde:
\[
Y_2(s) = \frac{k^2 X(s) + kF_1(s) + (ms^2 + 2k)F_2(s)}{m^2 s^4 + 3mks^2 + k^2}
\]

De polen van dit systeem (de wortels van de noemer \(m^2 s^4 + 3mks^2 + k^2\)) bepalen de eigenfrequenties van de trilling. Aangezien er geen demping is (s-termen ontbreken in de noemer, enkel \(s^4\) en \(s^2\)), zullen de polen zuiver imaginair zijn, wat wijst op continue oscillatie.
\end{oefenblok}

\subsection{Oefening 9.4: Eigenwaarden en Stabiliteit}

\begin{oefenblok}[Probleem]
Gegeven de systeemmatrix:
\[
A = \begin{pmatrix} 4 & 2 \\ 1 & 3 \end{pmatrix}
\]
Bepaal de eigenwaarden en analyseer de stabiliteit.
\end{oefenblok}

\begin{oefenblok}[Oplossing]
\textbf{Karakteristieke vergelijking:}

\[\det(A - \lambda I) = \det \begin{pmatrix} 4-\lambda & 2 \\ 1 & 3-\lambda \end{pmatrix} = (4-\lambda)(3-\lambda) - 2\]

\[= \lambda^2 - 7\lambda + 10 = 0\]

\[\lambda_1 = 5, \quad \lambda_2 = 2\]

\textbf{Stabiliteitsanalyse:}

Beide eigenwaarden zijn positief reëel. Dit betekent dat de natuurlijke oplossingen van de vorm \(e^{5t}\) en \(e^{2t}\) zijn, welke exponentieel groeien.

\boxed{\text{Conclusie:}} Dit systeem is \textbf{onstabiel}. Elke kleine verstoring zal leiden tot exponentiële groei.
\end{oefenblok}

\subsection{Oefening 9.5: Van Mechanisch Systeem naar Toestandsruimte}

\begin{oefenblok}[Probleem]
Beschouw een massa-veer-dempersysteem beschreven door:
\[
m \ddot{y} + c \dot{y} + k y = u(t)
\]
met \(m=1\), \(c=2\), \(k=5\). De input \(u(t)\) is de kracht, de output \(y(t)\) is de positie.
Stel de toestandsruimtevergelijkingen op.
\end{oefenblok}

\begin{oefenblok}[Oplossing]
We kiezen de toestandsvariabelen als de positie en de snelheid (de energie-opslagvariabelen):
\[
x_1 = y \quad (\text{positie})
\]
\[
x_2 = \dot{y} \quad (\text{snelheid})
\]

Nu leiden we de afgeleiden van de toestanden af:
1. \(\dot{x}_1 = \dot{y} = x_2\)
2. \(\dot{x}_2 = \ddot{y}\). Uit de differentiaalvergelijking halen we \(\ddot{y}\):
\[
\ddot{y} = \frac{1}{m} (u - c\dot{y} - ky) = u - 2\dot{y} - 5y
\]
Dus: \(\dot{x}_2 = -5x_1 - 2x_2 + u\)

In matrixvorm \(\dot{\mathbf{x}} = A\mathbf{x} + B\mathbf{u}\):
\[
\begin{pmatrix} \dot{x}_1 \\ \dot{x}_2 \end{pmatrix} = 
\begin{pmatrix} 0 & 1 \\ -5 & -2 \end{pmatrix} 
\begin{pmatrix} x_1 \\ x_2 \end{pmatrix} + 
\begin{pmatrix} 0 \\ 1 \end{pmatrix} u
\]

De output is de positie \(y = x_1\):
\[
y = \begin{pmatrix} 1 & 0 \end{pmatrix} \begin{pmatrix} x_1 \\ x_2 \end{pmatrix} + [0]u
\]

Dus:
\[
A = \begin{pmatrix} 0 & 1 \\ -5 & -2 \end{pmatrix}, \quad B = \begin{pmatrix} 0 \\ 1 \end{pmatrix}, \quad C = \begin{pmatrix} 1 & 0 \end{pmatrix}, \quad D = 0
\]
\end{oefenblok}

\subsection{Oefening 9.6: Eigenvectoren en Modale Analyse}

\begin{oefenblok}[Probleem]
Gegeven het systeem uit Oefening 9.4 met \(A = \begin{pmatrix} 4 & 2 \\ 1 & 3 \end{pmatrix}\).
We vonden eigenwaarden \(\lambda_1 = 5\) en \(\lambda_2 = 2\).
1. Bepaal de bijbehorende eigenvectoren \(\mathbf{v}_1\) en \(\mathbf{v}_2\).
2. Geef de algemene oplossing voor de toestand \(\mathbf{x}(t)\) als er geen input is (\(\mathbf{u}=0\)).
\end{oefenblok}

\begin{oefenblok}[Oplossing]
\textbf{1. Eigenvectoren}

Voor \(\lambda_1 = 5\):
Los op \((A - 5I)\mathbf{v}_1 = \mathbf{0}\):
\[
\begin{pmatrix} 4-5 & 2 \\ 1 & 3-5 \end{pmatrix} \begin{pmatrix} v_{1a} \\ v_{1b} \end{pmatrix} = \begin{pmatrix} -1 & 2 \\ 1 & -2 \end{pmatrix} \begin{pmatrix} v_{1a} \\ v_{1b} \end{pmatrix} = \begin{pmatrix} 0 \\ 0 \end{pmatrix}
\]
Dit geeft \(-v_{1a} + 2v_{1b} = 0 \Rightarrow v_{1a} = 2v_{1b}\).
Kies \(v_{1b} = 1\), dan \(v_{1a} = 2\).
\[
\mathbf{v}_1 = \begin{pmatrix} 2 \\ 1 \end{pmatrix}
\]

Voor \(\lambda_2 = 2\):
Los op \((A - 2I)\mathbf{v}_2 = \mathbf{0}\):
\[
\begin{pmatrix} 4-2 & 2 \\ 1 & 3-2 \end{pmatrix} \begin{pmatrix} v_{2a} \\ v_{2b} \end{pmatrix} = \begin{pmatrix} 2 & 2 \\ 1 & 1 \end{pmatrix} \begin{pmatrix} v_{2a} \\ v_{2b} \end{pmatrix} = \begin{pmatrix} 0 \\ 0 \end{pmatrix}
\]
Dit geeft \(v_{2a} + v_{2b} = 0 \Rightarrow v_{2a} = -v_{2b}\).
Kies \(v_{2b} = 1\), dan \(v_{2a} = -1\).
\[
\mathbf{v}_2 = \begin{pmatrix} -1 \\ 1 \end{pmatrix}
\]

\textbf{2. Algemene Oplossing}
De oplossing is een lineaire combinatie van de modi:
\[
\mathbf{x}(t) = c_1 e^{\lambda_1 t} \mathbf{v}_1 + c_2 e^{\lambda_2 t} \mathbf{v}_2
\]
\[
\mathbf{x}(t) = c_1 e^{5t} \begin{pmatrix} 2 \\ 1 \end{pmatrix} + c_2 e^{2t} \begin{pmatrix} -1 \\ 1 \end{pmatrix}
\]

\textbf{Interpretatie:} De vector \(\mathbf{v}_1\) geeft de richting aan waarin de toestand het snelst "explodeert" (grootste positieve eigenwaarde).
\end{oefenblok}

\subsection{Oefening 9.7: MIMO Systeem (Gekoppelde Tanks)}

\begin{oefenblok}[Probleem]
Een systeem met twee inputs (\(u_1, u_2\)) en twee outputs (\(y_1, y_2\)) wordt gegeven door:
\[
\dot{\mathbf{x}} = \begin{pmatrix} -1 & 0 \\ 1 & -2 \end{pmatrix} \mathbf{x} + \begin{pmatrix} 1 & 0 \\ 0 & 1 \end{pmatrix} \mathbf{u}
\]
\[
\mathbf{y} = \begin{pmatrix} 1 & 0 \\ 0 & 1 \end{pmatrix} \mathbf{x}
\]
Bepaal de transfermatrix \(H(s)\).
\end{oefenblok}

\begin{oefenblok}[Oplossing]
De transfermatrix wordt gegeven door \(H(s) = C(sI - A)^{-1}B + D\).
Hier is \(C = I\), \(B = I\), \(D = 0\). Dus \(H(s) = (sI - A)^{-1}\).

Bereken \(sI - A\):
\[
sI - A = \begin{pmatrix} s & 0 \\ 0 & s \end{pmatrix} - \begin{pmatrix} -1 & 0 \\ 1 & -2 \end{pmatrix} = \begin{pmatrix} s+1 & 0 \\ -1 & s+2 \end{pmatrix}
\]

Bereken de inverse:
\[
(sI - A)^{-1} = \frac{1}{\det(sI-A)} \text{adj}(sI-A)
\]
Determinant: \((s+1)(s+2) - (0)(-1) = (s+1)(s+2)\).
Adjunct matrix (verwissel diagonaal, negatieven op niet-diagonaal):
\[
\text{adj}(sI-A) = \begin{pmatrix} s+2 & 0 \\ 1 & s+1 \end{pmatrix}
\]

Dus:
\[
H(s) = \frac{1}{(s+1)(s+2)} \begin{pmatrix} s+2 & 0 \\ 1 & s+1 \end{pmatrix} = \begin{pmatrix} \frac{1}{s+1} & 0 \\ \frac{1}{(s+1)(s+2)} & \frac{1}{s+2} \end{pmatrix}
\]

\textbf{Interpretatie:}
\begin{itemize}
    \item \(H_{11}(s) = \frac{1}{s+1}\): Input 1 beïnvloedt Output 1 direct (eerste orde).
    \item \(H_{12}(s) = 0\): Input 2 heeft \textbf{geen} invloed op Output 1.
    \item \(H_{21}(s)\): Input 1 beïnvloedt Output 2 via een cascade (tweede orde).
    \item \(H_{22}(s)\): Input 2 beïnvloedt Output 2 direct.
\end{itemize}
Dit is typisch voor tanks in serie: vloeistof stroomt van tank 1 naar tank 2, maar niet terug.
\end{oefenblok}

\subsection{Oefening 9.8: Gekoppelde Watertanks (MIMO - Bidirectioneel)}

\begin{oefenblok}[Probleem]
We beschouwen een systeem van twee verbonden watertanks. Water stroomt binnen in Tank 1 met debiet \(q_{in}\) (input \(u\)). Er is een stroom van Tank 1 naar Tank 2, en een uitstroom uit Tank 2. De lineaire differentiaalvergelijkingen (rondom een werkpunt) zijn gegeven door:
\[
\dot{h}_1 = -2h_1 + h_2 + u
\]
\[
\dot{h}_2 = h_1 - 3h_2
\]
Hierin zijn \(h_1\) en \(h_2\) de genormaliseerde waterhoogtes (toestanden).

\textbf{Vragen:}
\begin{enumerate}
    \item \textbf{Matrixvorm:} Schrijf dit systeem in de standaard toestandsruimtevorm \(\dot{\mathbf{x}} = A\mathbf{x} + B\mathbf{u}\), waarbij \(\mathbf{x} = \begin{pmatrix} h_1 \\ h_2 \end{pmatrix}\).
    \item \textbf{Eigenwaarden:} Bereken de eigenwaarden van de systeemmatrix \(A\).
    \item \textbf{Stabiliteit:} Is dit systeem stabiel? Motiveer je antwoord aan de hand van de gevonden eigenwaarden.
    \item \textbf{Fysische interpretatie:} Wat betekent het feit dat de eigenwaarden reëel zijn voor de oscillatie van het waterniveau?
\end{enumerate}
\end{oefenblok}

\begin{oefenblok}[Oplossing]
\textbf{1. Matrixvorm}
\[
\begin{pmatrix} \dot{h}_1 \\ \dot{h}_2 \end{pmatrix} = 
\begin{pmatrix} -2 & 1 \\ 1 & -3 \end{pmatrix} 
\begin{pmatrix} h_1 \\ h_2 \end{pmatrix} + 
\begin{pmatrix} 1 \\ 0 \end{pmatrix} u
\]

\textbf{2. Eigenwaarden}
Karakteristieke vergelijking: \(\det(A - \lambda I) = 0\).
\[
(-2-\lambda)(-3-\lambda) - (1)(1) = 0
\]
\[
(\lambda^2 + 5\lambda + 6) - 1 = \lambda^2 + 5\lambda + 5 = 0
\]
De wortels zijn:
\[
\lambda_{1,2} = \frac{-5 \pm \sqrt{25 - 20}}{2} = \frac{-5 \pm \sqrt{5}}{2}
\]
\[
\lambda_1 \approx -1.38, \quad \lambda_2 \approx -3.62
\]

\textbf{3. Stabiliteit}
Beide eigenwaarden zijn strikt negatief. Het systeem is dus \textbf{asymptotisch stabiel}.

\textbf{4. Fysische interpretatie}
Omdat de eigenwaarden reëel zijn (geen imaginair deel), zal het systeem naar evenwicht terugkeren zonder oscillaties (overgedempt gedrag). Dit is logisch voor een diffusie/stromingsproces; water klotst niet heen en weer zoals een veer, maar niveaus vereffenen zich exponentieel.
\end{oefenblok}

\subsection{Oefening 9.9: Massa-Veer Systeem en Demping}

\begin{oefenblok}[Probleem]
Beschouw een massa \(m=1\) die vastzit aan een veer met constante \(k=4\) en een demper met constante \(c\). De bewegingsvergelijking is:
\[
y'' + cy' + 4y = 0
\]
We definiëren de toestand \(\mathbf{x} = \begin{pmatrix} y \\ y' \end{pmatrix}\).

\textbf{Vragen:}
\begin{enumerate}
    \item Stel de systeemmatrix \(A\) op voor dit systeem.
    \item Bereken de eigenwaarden voor het geval \(c=0\) (geen demping). Wat voor beweging beschrijft dit?
    \item Bereken de eigenwaarden voor het geval \(c=5\) (zware demping).
    \item Bepaal de bijbehorende eigenvector voor \(\lambda_1\) in het geval \(c=5\).
\end{enumerate}
\end{oefenblok}

\begin{oefenblok}[Oplossing]
\textbf{1. Systeemmatrix}
Uit \(\dot{x}_1 = x_2\) en \(\dot{x}_2 = -4x_1 - cx_2\) volgt:
\[
A = \begin{pmatrix} 0 & 1 \\ -4 & -c \end{pmatrix}
\]

\textbf{2. Geval \(c=0\)}
\[
\det \begin{pmatrix} -\lambda & 1 \\ -4 & -\lambda \end{pmatrix} = \lambda^2 + 4 = 0 \Rightarrow \lambda = \pm 2j
\]
Zuiver imaginaire eigenwaarden duiden op een \textbf{ongedempte harmonische trilling} (sinus/cosinus).

\textbf{3. Geval \(c=5\)}
\[
\lambda^2 + 5\lambda + 4 = 0 \Rightarrow (\lambda+1)(\lambda+4) = 0
\]
Eigenwaarden: \(\lambda_1 = -1, \quad \lambda_2 = -4\).
Dit is een \textbf{overgedempt} systeem (beide reëel en negatief).

\textbf{4. Eigenvector voor \(\lambda_1 = -1\)}
Los op \((A - \lambda_1 I)\mathbf{v} = \mathbf{0}\):
\[
\begin{pmatrix} 0 - (-1) & 1 \\ -4 & -5 - (-1) \end{pmatrix} \begin{pmatrix} v_a \\ v_b \end{pmatrix} = \begin{pmatrix} 1 & 1 \\ -4 & -4 \end{pmatrix} \begin{pmatrix} v_a \\ v_b \end{pmatrix} = \begin{pmatrix} 0 \\ 0 \end{pmatrix}
\]
Dit geeft \(v_a + v_b = 0\), dus \(v_a = -v_b\).
Een mogelijke eigenvector is:
\[
\mathbf{v}_1 = \begin{pmatrix} 1 \\ -1 \end{pmatrix}
\]
\textbf{Interpretatie:} In deze modus bewegen positie en snelheid in tegengestelde richting (terugkeren naar evenwicht).
\end{oefenblok}

\subsection{Oefening 9.10: Eigenvectoren van een 3x3 Matrix}

\begin{oefenblok}[Probleem]
Gegeven de systeemmatrix:
\[
A = \begin{pmatrix} 1 & 2 & 2 \\ 0 & 2 & 1 \\ 0 & 0 & 3 \end{pmatrix}
\]
Dit is een bovendriehoeksmatrix, wat de analyse vereenvoudigt.

\textbf{Vragen:}
\begin{enumerate}
    \item Wat zijn de eigenwaarden van deze matrix? (Hint: Kijk naar de diagonaal).
    \item Zoek de eigenvector \(\mathbf{v}_3\) die hoort bij de grootste eigenwaarde.
    \item Als dit systeem \(\dot{\mathbf{x}} = A\mathbf{x}\) beschrijft, welke term in de oplossing zal op lange termijn domineren?
\end{enumerate}
\end{oefenblok}

\begin{oefenblok}[Oplossing]
\textbf{1. Eigenwaarden}
Bij een driehoeksmatrix zijn de eigenwaarden de diagonaalelementen:
\[
\lambda_1 = 1, \quad \lambda_2 = 2, \quad \lambda_3 = 3
\]

\textbf{2. Eigenvector voor \(\lambda_3 = 3\)}
Los op \((A - 3I)\mathbf{v} = \mathbf{0}\):
\[
\begin{pmatrix} 1-3 & 2 & 2 \\ 0 & 2-3 & 1 \\ 0 & 0 & 3-3 \end{pmatrix} \begin{pmatrix} x \\ y \\ z \end{pmatrix} = \begin{pmatrix} -2 & 2 & 2 \\ 0 & -1 & 1 \\ 0 & 0 & 0 \end{pmatrix} \begin{pmatrix} x \\ y \\ z \end{pmatrix} = \begin{pmatrix} 0 \\ 0 \\ 0 \end{pmatrix}
\]
Uit rij 2: \(-y + z = 0 \Rightarrow y = z\).
Uit rij 1: \(-2x + 2y + 2z = 0 \Rightarrow -x + y + z = 0\).
Substitueer \(y=z\): \(-x + 2z = 0 \Rightarrow x = 2z\).
Stel \(z=1\), dan is de eigenvector:
\[
\mathbf{v}_3 = \begin{pmatrix} 2 \\ 1 \\ 1 \end{pmatrix}
\]

\textbf{3. Dominantie}
De algemene oplossing is \(c_1 e^{1t} \mathbf{v}_1 + c_2 e^{2t} \mathbf{v}_2 + c_3 e^{3t} \mathbf{v}_3\).
Op lange termijn (\(t \to \infty\)) groeit \(e^{3t}\) het snelst. De systeemtoestand zal dus evolueren in de richting van eigenvector \(\mathbf{v}_3\). Aangezien de eigenwaarden positief zijn, is dit systeem \textbf{instabiel}.
\end{oefenblok}

\subsection{Oefening 9.11: Van Transferfunctie naar State-Space}

\begin{oefenblok}[Probleem]
Gegeven een mechanisch systeem beschreven door de differentiaalvergelijking:
\[
\dddot{y} + 6\ddot{y} + 11\dot{y} + 6y = u(t)
\]
\textbf{Vragen:}
\begin{enumerate}
    \item Definieer de toestandsvariabelen \(x_1, x_2, x_3\) zodanig dat \(x_1 = y\).
    \item Stel de "Companion Matrix" \(A\) op voor dit systeem.
    \item De karakteristieke vergelijking van de matrix is gelijk aan de noemer van de transferfunctie. Controleer dit door \(\det(sI - A)\) te berekenen (of de karakteristieke polynoom direct af te lezen).
\end{enumerate}
\end{oefenblok}

\begin{oefenblok}[Oplossing]
\textbf{1. Toestandsvariabelen}
Kies \(x_1 = y\), \(x_2 = \dot{y}\), \(x_3 = \ddot{y}\).

Dan geldt:
\begin{itemize}
    \item \(\dot{x}_1 = x_2\)
    \item \(\dot{x}_2 = x_3\)
    \item \(\dot{x}_3 = \dddot{y} = -6y - 11\dot{y} - 6\ddot{y} + u = -6x_1 - 11x_2 - 6x_3 + u\)
\end{itemize}

\textbf{2. Matrix A}
\[
A = \begin{pmatrix} 0 & 1 & 0 \\ 0 & 0 & 1 \\ -6 & -11 & -6 \end{pmatrix}
\]

\textbf{3. Karakteristieke Polynoom}
De karakteristieke polynoom is \(\lambda^3 + 6\lambda^2 + 11\lambda + 6 = 0\).
Dit komt exact overeen met de coëfficiënten van de differentiaalvergelijking (en dus de noemer van de transferfunctie).
\end{oefenblok}

\subsection{Oefening 9.12: Tijdsdomein Oplossing van een Symmetrisch MIMO Systeem}

\begin{oefenblok}[Probleem]
Beschouw een symmetrisch gekoppeld systeem (bijvoorbeeld twee identieke tanks met uitwisseling):
\[
\dot{\mathbf{x}} = \begin{pmatrix} -2 & 1 \\ 1 & -2 \end{pmatrix} \mathbf{x}
\]
Gegeven de beginconditie \(\mathbf{x}(0) = \begin{pmatrix} 10 \\ 0 \end{pmatrix}\) (Tank 1 is vol, Tank 2 is leeg).
Los dit stelsel op met behulp van eigenwaarden en eigenvectoren om \(x_1(t)\) en \(x_2(t)\) te vinden.
\end{oefenblok}

\begin{oefenblok}[Oplossing]
\textbf{Stap 1: Eigenwaarden}
\[
\det(A - \lambda I) = \det \begin{pmatrix} -2-\lambda & 1 \\ 1 & -2-\lambda \end{pmatrix} = (-2-\lambda)^2 - 1 = 0
\]
\[
(\lambda + 2)^2 = 1 \Rightarrow \lambda + 2 = \pm 1
\]
\[
\lambda_1 = -1, \quad \lambda_2 = -3
\]

\textbf{Stap 2: Eigenvectoren}
Voor \(\lambda_1 = -1\):
\[
\begin{pmatrix} -1 & 1 \\ 1 & -1 \end{pmatrix} \mathbf{v}_1 = \mathbf{0} \Rightarrow v_a = v_b \Rightarrow \mathbf{v}_1 = \begin{pmatrix} 1 \\ 1 \end{pmatrix}
\]
Voor \(\lambda_2 = -3\):
\[
\begin{pmatrix} 1 & 1 \\ 1 & 1 \end{pmatrix} \mathbf{v}_2 = \mathbf{0} \Rightarrow v_a = -v_b \Rightarrow \mathbf{v}_2 = \begin{pmatrix} 1 \\ -1 \end{pmatrix}
\]

\textbf{Stap 3: Algemene Oplossing}
\[
\mathbf{x}(t) = c_1 e^{-t} \begin{pmatrix} 1 \\ 1 \end{pmatrix} + c_2 e^{-3t} \begin{pmatrix} 1 \\ -1 \end{pmatrix}
\]

\textbf{Stap 4: Specifieke Oplossing (Beginvoorwaarden)}
Op \(t=0\):
\[
\begin{pmatrix} 10 \\ 0 \end{pmatrix} = c_1 \begin{pmatrix} 1 \\ 1 \end{pmatrix} + c_2 \begin{pmatrix} 1 \\ -1 \end{pmatrix} = \begin{pmatrix} c_1 + c_2 \\ c_1 - c_2 \end{pmatrix}
\]
Dit geeft het stelsel:
\[
\begin{cases} c_1 + c_2 = 10 \\ c_1 - c_2 = 0 \end{cases}
\]
Uit de tweede vgl volgt \(c_1 = c_2\). Invullen in de eerste: \(2c_1 = 10 \Rightarrow c_1 = 5\). Dus \(c_2 = 5\).

\textbf{Eindresultaat:}
\[
\mathbf{x}(t) = 5 e^{-t} \begin{pmatrix} 1 \\ 1 \end{pmatrix} + 5 e^{-3t} \begin{pmatrix} 1 \\ -1 \end{pmatrix}
\]
Of per component:
\[
x_1(t) = 5e^{-t} + 5e^{-3t}
\]
\[
x_2(t) = 5e^{-t} - 5e^{-3t}
\]

\textbf{Interpretatie:}
\begin{itemize}
    \item \(x_1\) begint op 10 en daalt naar 0.
    \item \(x_2\) begint op 0, stijgt eerst (door instroom uit tank 1) en daalt dan weer naar 0.
    \item De term \(e^{-t}\) is de "trage" modus (evenwichtsvereffening), \(e^{-3t}\) is de "snelle" modus (initieel verschil).
\end{itemize}

\begin{center}
\begin{tikzpicture}[scale=0.8]
    \begin{axis}[
        width=10cm, height=6cm,
        axis lines=middle,
        xlabel={$t$}, ylabel={Niveau},
        domain=0:5, samples=100,
        ymin=0, ymax=10,
        grid=both,
        legend pos=north east
    ]
    \addplot[thick, blue] {5*exp(-x) + 5*exp(-3*x)};
    \addlegendentry{$x_1(t)$}
    \addplot[thick, red] {5*exp(-x) - 5*exp(-3*x)};
    \addlegendentry{$x_2(t)$}
    \end{axis}
\end{tikzpicture}
\end{center}
\end{oefenblok}

\appendix


\end{document}
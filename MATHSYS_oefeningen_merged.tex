% Merged comprehensive MATHSYS exercises with all solutions
\documentclass[a4paper,11pt]{article}

\usepackage[dutch]{babel}
\usepackage[utf8]{inputenc}
\usepackage{amsmath}
\usepackage{amssymb}
\usepackage{geometry}
\usepackage{enumitem}
\usepackage{graphicx}
\usepackage{tikz}
\usepackage[hidelinks]{hyperref}

\geometry{margin=2.5cm}

\title{Wiskunde voor Systemen\\Oefeningen met Oplossingen}
\author{KU Leuven -- ESAT\\Campus Groep T\\[0.5em]\small Gebaseerd op Prof. Toon van Waterschoot}
\date{\today}

\begin{document}

\maketitle
\tableofcontents
\newpage

% FORMULARIUM SECTION
\section{Formularium}
\label{sec:formularium}

Dit formularium bevat alle essenti\u00eble formules nodig voor de oefeningen.

\subsection{Laplace Transform (LT)}
\label{form:laplace}

\subsubsection*{Definitie en eigenschappen}
\label{form:laplace-def}
\label{form:laplace-prop}

\begingroup
\setlength{\tabcolsep}{6pt}
\renewcommand{\arraystretch}{1.35}
\[
\begin{array}{@{}lcl@{}}
\text{Definitie:} & f(t)\,u(t) & \longleftrightarrow\; F(s)=\displaystyle\int_0^{\infty} f(t)e^{-st}\,dt\\
\text{translatie in } s: & f(t)e^{-at}\,u(t) & \longleftrightarrow\; F(s+a)\\
\text{translatie in } t: & f(t-a)\,u(t-a) & \longleftrightarrow\; e^{-as}F(s)\\
\text{afgeleide in } t: & \dfrac{d}{dt}\big(f(t)u(t)\big) & \longleftrightarrow\; sF(s)-f(0^+)\\
& \dfrac{d^2}{dt^2}\big(f(t)u(t)\big) & \longleftrightarrow\; s^2F(s)-s f(0^+)-f'(0^+)\\
\text{convolutie:} & \big(f*g\big)(t)\,u(t) & \longleftrightarrow\; F(s)\,G(s)\\
\end{array}
\]
\endgroup

\noindent\textbf{Initial value theorem:} $\lim_{t\to 0^+} f(t)=\lim_{s\to\infty} sF(s)$

\noindent\textbf{Final value theorem:} $\lim_{t\to\infty} f(t)=\lim_{s\to 0} sF(s)$

\subsubsection*{Laplace transform pairs}
\label{form:laplace-pairs}

\begingroup
\setlength{\tabcolsep}{10pt}
\renewcommand{\arraystretch}{1.35}
\[
\begin{array}{@{}lcl@{\qquad}lcl@{}}
e^{-at}u(t) & \longleftrightarrow & \dfrac{1}{s+a} & t^n u(t) & \longleftrightarrow & \dfrac{n!}{s^{n+1}}\\
\cos(a t)u(t) & \longleftrightarrow & \dfrac{s}{s^2+a^2} & \sin(a t)u(t) & \longleftrightarrow & \dfrac{a}{s^2+a^2}\\
\delta(t) & \longleftrightarrow & 1 & u(t) & \longleftrightarrow & \dfrac{1}{s}\\
\end{array}
\]
\endgroup

\subsection{Fourier Transform (FTC)}
\label{form:ft}

\subsubsection*{Definitie en basisformules}
\label{form:ft-def}

\[
X(\omega)=\int_{-\infty}^{\infty} x(t)e^{-j\omega t}\,dt, \qquad
x(t)=\frac{1}{2\pi}\int_{-\infty}^{\infty} X(\omega)e^{j\omega t}\,d\omega
\]

\subsubsection*{Useful Fourier pairs}
\label{form:ft-pairs}

\[
\begin{array}{@{}lcl@{}}
\text{Block:} & u(t+L/2)-u(t-L/2) & \longleftrightarrow\; L\,\mathrm{sinc}(\omega L/2)\\
\text{Impuls:} & \delta(t) & \longleftrightarrow\; 1\\
\text{Cosine:} & \cos(\omega_0 t) & \longleftrightarrow\; \pi[\delta(\omega+\omega_0)+\delta(\omega-\omega_0)]\\
\end{array}
\]

\subsection{Fourier Series (FS)}
\label{form:fs}

Voor periode $T$ met $\omega_0=\frac{2\pi}{T}$:

\[
f(t)=\frac{a_0}{2}+\sum_{k=1}^{\infty}\left[a_k\cos(k\omega_0 t)+b_k\sin(k\omega_0 t)\right]
\]

\[
c_k=\frac{1}{T}\int_{0}^{T} f(t)\,e^{-jk\omega_0 t}\,dt
\]

\newpage

% CHAPTER 1
\section{Hoofdstuk 1: Signalen en Systemen}

\subsection{Oefening 1.1: Lineaire Systemen}

Gegeven: $\mathcal{T}\{x(t)\} = 3x(t) + 2$

\textbf{Vraag:} Is dit systeem lineair?

\subsection{Oefening 1.2: RC-Circuit}

$R = 1000\,\Omega$, $C = 10\,\mu$F, $v_{\text{in}}(t) = 5u(t)$ V

\textbf{Vraag:}
\begin{enumerate}[label=(\alph*)]
\item Differentiaalvergelijking voor $v_{\text{uit}}(t)$
\item Tijdsconstante $\tau$
\item $v_{\text{uit}}(t)$ na 10 ms met $v_{\text{uit}}(0) = 0$ V
\end{enumerate}

\section{Hoofdstuk 2: Basissignalen en Bewerkingen}

\subsection{Oefening 2.1: Exponenti\u00eble Functies}

Gegeven: $x_1(t) = e^{0.2t}$ en $x_2(t) = e^{-0.5t}$

\textbf{Vraag:}
\begin{enumerate}[label=(\alph*)]
\item Welk signaal vertoont groei/verval?
\item Waarden op $t = 5$ s
\end{enumerate}

\subsection{Oefening 2.2: Sinus en Cosinus}

Gegeven: $x(t) = 3\sin(4\pi t + \frac{\pi}{6})$

\textbf{Vraag:}
\begin{enumerate}[label=(\alph*)]
\item Amplitude, frequentie, fasehoek
\item Schrijf als cosinusfunctie
\end{enumerate}

\section{Hoofdstuk 3: Laplacetransformatie}

\subsection{Oefening 3.1: Eenvoudige Laplacetransformaties}

\textbf{Vraag:} Bepaal Laplace-transformatie van:
\begin{enumerate}[label=(\alph*)]
\item $f(t) = 5u(t)$
\item $f(t) = e^{-3t}u(t)$
\item $f(t) = t \cdot u(t)$
\item $f(t) = \cos(5t) \cdot u(t)$
\end{enumerate}

\subsubsection*{Oplossing 3.1}

\begin{enumerate}[label=(\alph*)]

\item $\mathcal{L}\{5u(t)\}=5\cdot\frac{1}{s}=\frac{5}{s}$ 

\item $\mathcal{L}\{e^{-3t}u(t)\}=\frac{1}{s+3}$

\item $\mathcal{L}\{t\,u(t)\}=\frac{1}{s^2}$

\item $\mathcal{L}\{\cos(5t)u(t)\}=\frac{s}{s^2+25}$

\end{enumerate}

\subsection{Oefening 3.2: Inverse Laplacetransformatie}

\textbf{Gegeven:} $F(s) = \frac{3}{s+2} + \frac{5}{s^2 + 4}$

\textbf{Vraag:} Vind $f(t)$

\subsubsection*{Oplossing 3.2}

\[
f(t) = 3e^{-2t}u(t) + \frac{5}{2}\sin(2t)u(t)
\]

\subsection{Oefening 3.3: Eerste-Orde Systeem}

\textbf{Gegeven:} $\frac{dy}{dt} + 4y = 8u(t)$, $y(0) = 2$

\textbf{Vraag:} Bepaal $y(t)$

\subsubsection*{Oplossing 3.3}

Laplacetransformatie:
\[
sY(s) - 2 + 4Y(s) = \frac{8}{s}
\]

\[
Y(s)(s+4) = \frac{8 + 2s}{s} = \frac{8 + 2s}{s}
\]

Via partieelbreuken: $Y(s) = \frac{2}{s}$

Dus: $y(t) = 2u(t)$

\subsection{Oefening 3.4: Tweede-Orde Systeem}

\textbf{Gegeven:} $\frac{d^2y}{dt^2} + 4\frac{dy}{dt} + 3y = 0$, $y(0) = 1$, $y'(0) = 0$

\textbf{Vraag:}
\begin{enumerate}[label=(\alph*)]
\item Karakteristieke vergelijking
\item Wortels
\item Oplossing $y(t)$
\end{enumerate}

\subsubsection*{Oplossing 3.4}

\begin{enumerate}[label=(\alph*)]

\item Karakteristieke vergelijking: $\lambda^2 + 4\lambda + 3 = 0$

\item Wortels: $\lambda = -1, -3$

\item Met beginvoorwaarden $y(0) = 1$ en $y'(0) = 0$:

Stelsel:
\begin{align*}
c_1 + c_2 &= 1\\
-c_1 - 3c_2 &= 0
\end{align*}

Oplossing: $c_1 = 3/2$, $c_2 = -1/2$

\[
y(t) = \frac{3}{2}e^{-t} - \frac{1}{2}e^{-3t}
\]

\end{enumerate}

\subsection{Oefening 3.5: Laplace met Verschuiving}

\textbf{Gegeven:} $F(s) = \frac{2}{s^2 + 4}$

\textbf{Vraag:}
\begin{enumerate}[label=(\alph*)]
\item Bepaal $f(t) = \mathcal{L}^{-1}\{F(s)\}$
\item Bepaal $g(t) = \mathcal{L}^{-1}\{e^{-2s}F(s)\}$
\end{enumerate}

\subsubsection*{Oplossing 3.5}

\begin{enumerate}[label=(\alph*)]

\item $f(t) = \sin(2t)u(t)$

\item Met tijdsverschuivingsstelling:
\[
g(t) = \sin(2(t-2))u(t-2) = \sin(2t-4)u(t-2)
\]

\end{enumerate}

\subsection{Oefening 3.6: Partieelbreuken}

\textbf{Gegeven:} $F(s) = \frac{10}{(s+1)(s+2)(s+3)}$

\textbf{Vraag:} Bepaal $f(t)$

\subsubsection*{Oplossing 3.6}

Partieelbreukontwikkeling:
\[
\frac{10}{(s+1)(s+2)(s+3)} = \frac{5}{s+1} - \frac{10}{s+2} + \frac{5}{s+3}
\]

Inverse Laplace:
\[
f(t) = 5e^{-t}u(t) - 10e^{-2t}u(t) + 5e^{-3t}u(t)
\]

\subsection{Oefening 3.7: Begin- en Eindwaardestelling}

\textbf{Gegeven:} $F(s) = \frac{3s + 5}{s^2 + 4s + 3}$

\textbf{Vraag:}
\begin{enumerate}[label=(\alph*)]
\item $f(0^+)$ met beginwaardestelling
\item $f(\infty)$ met eindwaardestelling
\item Controleer door $f(t)$ te berekenen
\end{enumerate}

\subsubsection*{Oplossing 3.7}

\begin{enumerate}[label=(\alph*)]

\item Beginwaardestelling: $f(0^+) = \lim_{s\to\infty} s \cdot \frac{3s+5}{s^2+4s+3} = 3$

\item Eindwaardestelling: $f(\infty) = \lim_{s\to 0} s \cdot \frac{3s+5}{s^2+4s+3} = 0$

\item Via partieelbreuken: $F(s) = \frac{1}{s+1} + \frac{2}{s+3}$

Dus: $f(t) = (e^{-t} + 2e^{-3t})u(t)$

Verificatie: $f(0^+) = 1 + 2 = 3$ ✓ en $\lim_{t\to\infty} f(t) = 0$ ✓

\end{enumerate}

\subsection{Oefening 3.8: Convolutie via Laplace}

\textbf{Gegeven:} $f(t) = u(t) - u(t-1)$, $g(t) = e^{-2t}u(t)$

\textbf{Vraag:} Bepaal $y(t) = (f*g)(t)$

\subsubsection*{Oplossing 3.8}

Laplace-transformaties:
\[
F(s) = \frac{1-e^{-s}}{s}, \quad G(s) = \frac{1}{s+2}
\]

Product:
\[
Y(s) = \frac{1-e^{-s}}{s(s+2)}
\]

Via partieelbreuken: $\frac{1}{s(s+2)} = \frac{1}{2}(\frac{1}{s} - \frac{1}{s+2})$

Inverse Laplace:
\[
y(t)=\begin{cases}
\frac{1}{2}(1-e^{-2t}) & 0 \le t < 1\\
\frac{1}{2}(e^{-2(t-1)}-e^{-2t}) & t \ge 1
\end{cases}
\]

\subsection{Oefening 3.9: Laplace Basis}

\textbf{Vraag:}
\begin{enumerate}[label=(\alph*)]
\item $\mathcal{L}\{u(t)\}$
\item $\mathcal{L}\{e^{-2t}u(t)\}$
\item $\mathcal{L}\{t\,u(t)\}$
\item $\mathcal{L}^{-1}\{\frac{1}{s+3}+\frac{2}{s^2}\}$
\end{enumerate}

\subsubsection*{Oplossing 3.9}

\begin{enumerate}[label=(\alph*)]

\item $\mathcal{L}\{u(t)\} = \frac{1}{s}$

\item $\mathcal{L}\{e^{-2t}u(t)\} = \frac{1}{s+2}$

\item $\mathcal{L}\{t\,u(t)\} = \frac{1}{s^2}$

\item $f(t) = e^{-3t}u(t) + 2t\,u(t)$

\end{enumerate}

\newpage

\section{Hoofdstuk 6: LTC-Systemen}

\subsection{Oefening 6.1: Impulsrespons}

\textbf{Gegeven:} $h(t) = 2e^{-5t}u(t)$, $x(t) = u(t)$

\textbf{Vraag:}
\begin{enumerate}[label=(\alph*)]
\item Bepaal $y(t) = h(t) * x(t)$ via convolutie
\item Verifieer met Laplace
\end{enumerate}

\subsubsection*{Oplossing 6.1}

\begin{enumerate}[label=(\alph*)]

\item Via convolutie:
\[
y(t) = \int_0^t 2e^{-5\tau}d\tau = \frac{2}{5}(1-e^{-5t})u(t)
\]

\item Via Laplace: $H(s) = \frac{2}{s+5}$, $X(s) = \frac{1}{s}$

$Y(s) = \frac{2}{s(s+5)} = \frac{2/5}{s} - \frac{2/5}{s+5}$

Inverse: $y(t) = \frac{2}{5}(1-e^{-5t})u(t)$ ✓

\end{enumerate}

\subsection{Oefening 6.3: Frequentierespons}

\textbf{Gegeven:} $H(s) = \frac{10}{s+5}$

\textbf{Vraag:}
\begin{enumerate}[label=(\alph*)]
\item Frequentierespons $H(j\omega)$
\item Amplitude- en faserespons
\item 3dB bandbreedte
\end{enumerate}

\subsubsection*{Oplossing 6.3}

\begin{enumerate}[label=(\alph*)]

\item $H(j\omega) = \frac{10}{5+j\omega}$

\item $|H(j\omega)| = \frac{10}{\sqrt{25+\omega^2}}$, $\angle H(j\omega) = -\arctan(\omega/5)$

\item 3dB bandbreedte: $\omega_{3dB} = 5$ rad/s

\end{enumerate}

\subsection{Oefening 6.4: Cascade Systemen}

\textbf{Gegeven:} $H_1(s) = \frac{5}{s+2}$, $H_2(s) = \frac{3}{s+3}$

\textbf{Vraag:} Bepaal totale overdracht en impulsrespons

\subsubsection*{Oplossing 6.4}

\begin{enumerate}[label=(\alph*)]

\item $H(s) = \frac{15}{(s+2)(s+3)}$

\item Via partieelbreuken:
\[
h(t) = 15(e^{-2t} - e^{-3t})u(t)
\]

\end{enumerate}

\subsection{Oefening 6.5: Stabiliteit en Polen}

\textbf{Vraag:} Bepaal BIBO-stabiliteit

\begin{enumerate}[label=(\alph*)]
\item $H_1(s) = \frac{1}{s-2}$
\item $H_2(s) = \frac{1}{s^2+3s+2}$
\item $H_3(s) = \frac{1}{s^2+4}$
\end{enumerate}

\subsubsection*{Oplossing 6.5}

\begin{enumerate}[label=(\alph*)]

\item Pool op $s=2$ (rechterhalfvlak) $\Rightarrow$ \textbf{ONSTABIEL}

\item Polen op $s=-1,-2$ (linkerhalfvlak) $\Rightarrow$ \textbf{STABIEL}

\item Polen op $s=\pm 2j$ (imaginaire as) $\Rightarrow$ \textbf{MARGINAAL STABIEL}

\end{enumerate}

\section{Hoofdstuk 8: Examengerichte Oefeningen}

\subsection{Oefening 8.1: FTC-eigenschappen}

\textbf{Gegeven:} Bloksignaal $x(t) = \begin{cases}1 & -1/2<t<1/2\\0 & \text{elders}\end{cases}$

\textbf{Vraag:} Bepaal $Y(j\omega)$ voor $y(t) = x(t)\cos(2\pi t)$

\subsection{Oefening 8.12: DV via Laplace}

\textbf{Gegeven:} $H(s) = \frac{4}{s^2+3s+2}$, $x(t) = 2u(t)$, $y(0) = 0$, $y'(0) = 1$

\textbf{Vraag:} Los DV op en verifieer met Laplace

\subsubsection*{Oplossing 8.12}

Differentiaalvergelijking uit $H(s)$:
\[
y''(t) + 3y'(t) + 2y(t) = 8u(t)
\]

Karakteristieke vergelijking: $\lambda^2 + 3\lambda + 2 = 0 \Rightarrow \lambda = -1, -2$

Algemene oplossing: $y_h = c_1 e^{-t} + c_2 e^{-2t} + 4$

Met beginvoorwaarden: $c_1 = -9$, $c_2 = 5$

\[
y(t) = -9e^{-t} + 5e^{-2t} + 4
\]

Verificatie via Laplace geeft dezelfde uitkomst ✓

\subsection{Oefening 8.13: FTC + Parseval}

\textbf{Gegeven:} $f(t) = 3e^{-2t}u(t)$

\textbf{Vraag:} FTC bepalen, energie berekenen, Parseval verifiëren

\subsubsection*{Oplossing 8.13}

\begin{enumerate}[label=(\alph*)]

\item FTC via Laplace-link: $F(j\omega) = \frac{3}{2+j\omega}$

\item Amplitudespectrum: $|F(j\omega)| = \frac{3}{\sqrt{4+\omega^2}}$

\item Energie in tijdsdomein: $E = \int_0^{\infty} 9e^{-4t} dt = \frac{9}{4}$

\item Parseval: $E = \frac{1}{2\pi}\int_{-\infty}^{\infty} \frac{9}{4+\omega^2} d\omega = \frac{9}{4}$ ✓

\end{enumerate}

\end{document}

% ProductieTechnologie-Samenvatting-RubenRyckaert.tex
% Simple layout template with helpers for chapters, a formularium, and an easy figure helper.

\documentclass[a4paper,12pt,twoside]{report}
\usepackage[utf8]{inputenc}
\usepackage[dutch]{babel}
\usepackage[T1]{fontenc}
% Use default (Computer Modern) but ensure scalable fonts via Latin Modern, keep microtype
\usepackage{lmodern}
\usepackage[activate={true,nocompatibility},final,tracking=true,kerning=true,spacing=true,stretch=10,shrink=10]{microtype}
% Ensure microtype uses the correct spacing model with nonfrenchspacing
\microtypecontext{spacing=nonfrench}
% Slightly increase line spacing for readability
\usepackage{setspace}
\setstretch{1.08}
\usepackage{amsmath,amssymb}
\usepackage{graphicx}
\usepackage{tikz}
\usepackage{fancyhdr}
\usepackage{tcolorbox}
\usepackage{float}
% Page geometry: reduce margins and make them consistent
\usepackage[left=2.0cm,right=2.0cm,top=2.5cm,bottom=2.5cm]{geometry}
% Configure hyperref to avoid duplicate destination names (see hyperref option `hypertexnames`)
\usepackage[hidelinks,hypertexnames=false]{hyperref}
% Use `bookmark` to make outline writing more robust and avoid "file has changed" rerun warnings
\usepackage{bookmark}
\usepackage{caption}
\captionsetup{skip=3pt,aboveskip=3pt,belowskip=3pt} % reduce caption separation
\usepackage{enumitem}% Shared project macros (\frm, formularium, \fig)
\usepackage{./productie-macros}

% Hyphenation & tolerance adjustments to reduce underfull boxes
% Increase tolerance slightly and give TeX more emergency stretch so fewer
% underfull \hbox warnings occur in compact technical text
\tolerance=1000
\hyphenpenalty=300
\exhyphenpenalty=300
\emergencystretch=2em
% If you see a Babel warning about Dutch hyphenation patterns, install them
% and rebuild your format (tlmgr or system package manager) to enable real
% Dutch hyphenation and reduce badness further
\AtBeginDocument{\PackageWarning{productie-macros}{If you see 'No hyphenation patterns were preloaded for the language "Dutch"', please install Dutch hyphenation patterns and rebuild your TeX format (e.g., using tlmgr).}}

% Allow ragged bottom so LaTeX doesn't stretch pages to fill vertical space
\raggedbottom
% For customizing chapter styles and safely patching \chapter
\usepackage{etoolbox}
\usepackage{titlesec}
% Prevent figures from floating past chapter boundaries and ensure floats don't appear before they are defined
\usepackage{placeins}
\usepackage{flafter}
% Insert a FloatBarrier before every \chapter to keep floats in their chapter
\preto\chapter{\FloatBarrier}

% Make chapters keep number and title on the same line and avoid forcing a new page
\makeatletter
\patchcmd{\chapter}{\if@openright\cleardoublepage\else\clearpage\fi}{}{}{}
\makeatother
% Slightly smaller chapter titles (keep them prominent but less tall)
\titleformat{\chapter}[hang]{\normalfont\LARGE\bfseries}{\thechapter}{1em}{}
\titlespacing*{\chapter}{0pt}{0.5ex}{0.5ex} % tighten above/below spacing to reduce blank space between chapters and content
% Ensure section and subsection align to same left margin and use sensible spacing
% Slightly reduce section/subsection size so titles are less dominant
\titleformat{\section}[hang]{\normalfont\large\bfseries\raggedright}{\thesection}{1em}{}
\titlespacing*{\section}{0pt}{1.5ex}{0.8ex}
\titleformat{\subsection}[hang]{\normalfont\normalsize\bfseries\raggedright}{\thesubsection}{1em}{}
\titlespacing*{\subsection}{0pt}{1.0ex}{0.6ex}

% Paragraph spacing and list defaults for consistent left alignment
\setlength{\parindent}{0pt} % if you prefer paragraph indent restore this to 1em
\setlength{\parskip}{0.35ex plus 0.1ex} % slightly larger parskip for better readability
% Compact list indentation (smaller distance from page margin)
% Aggressively tighten vertical spacing in lists
\setlist[itemize,1]{leftmargin=1em,itemsep=0pt,parsep=0pt,partopsep=0pt,topsep=0pt}
\setlist[itemize,2]{leftmargin=1.4em,itemsep=0pt,parsep=0pt,partopsep=0pt,topsep=0pt}
\setlist[enumerate,1]{leftmargin=1em,itemsep=0pt,parsep=0pt,partopsep=0pt,topsep=0pt}
\setlist[enumerate,2]{leftmargin=1.4em,itemsep=0pt,parsep=0pt,partopsep=0pt,topsep=0pt}
% ------------------ Header / Footer ------------------
% Avoid fancyhdr warning about small headheight
\setlength{\headheight}{14pt}
\pagestyle{fancy}
\fancyhf{}
\fancyhead[LE,RO]{\small\bfseries\thepage}
\fancyhead[LO,RE]{\small\nouppercase{\leftmark}}
\renewcommand{\headrulewidth}{0.3pt}
% Reduce vertical spacing around floats to avoid large gaps
\setlength{\floatsep}{6pt plus 1pt minus 1pt}     % between floats
\setlength{\textfloatsep}{8pt plus 1pt minus 1pt} % between floats and text
\setlength{\intextsep}{6pt plus 1pt minus 1pt}    % for in-text floats (top/bottom of figure)

% ------------------ Formularium helpers ------------------
% Provided by `productie-macros.sty` (\frm, formularium, \fig helpers).  
% You can override box colors using \setfrmcolors{<back>}{<frame>} after loading the package.

% ------------------ Figure helper ------------------
% Provided by `productie-macros.sty` via the \fig helper.

% ------------------ Document metadata ------------------
\title{Productietechnologie — Samenvatting}
\author{Ruben Ryckaert}
\date{\today}

\begin{document}
\maketitle
\microtypesetup{protrusion=false}
% Also disable protrusion for LoF and LoT so those lists use consistent spacing
\pretocmd{\listoffigures}{\microtypesetup{protrusion=false}}{}{}
\apptocmd{\listoffigures}{\microtypesetup{protrusion=true}}{}{}
\pretocmd{\listoftables}{\microtypesetup{protrusion=false}}{}{}
\apptocmd{\listoftables}{\microtypesetup{protrusion=true}}{}{}
\tableofcontents
\microtypesetup{protrusion=true}

\begin{formularium}
% Formularium entries can be added here if needed
\end{formularium}


\chapter{inleiding}
\textbf{Wat is ProductieTechnologie?}
\newline
ProductieTechnologie gaat over het produceren van 
goederen. Hier komt veel bij te pas.
Niet alleen verschillende technieken en machines,
maar ook kosten, snelheid, kwaliteit, ...
\newline

Deze samenvatting hoopt een overzicht te geven
van de belangrijkste begrippen en technieken


\textbf{Hieronder verschillende productietechnieken,}
\begin{itemize}
  \item Gieten
    \begin{itemize}
      \item Zandgieten
      \item Spuitgieten
    \end{itemize}
  \item Frezen
  \item Lassen
    \begin{itemize}
      \item CO2-lassen
      \item MIG/MAG, TIG, ...
    \end{itemize}
  \item Vonkerosie
  \item Waterstraalsnijden
  \item Chemisch bewerken
  \item 3D-printen
  \item Draaien
  \item Snijden
  \item Ponsen
  \item Stralen
  \begin{figure}[ht]
  \centering\includegraphics[width=0.5\textwidth]{straalbewerkingen.png}
  \caption{Overzicht van bewerkingen met stralen, gebruikmakend van verschillende energiedragers}
  \label{fig:stralen_overzicht}
\end{figure}
\end{itemize}

\subsection{Keuzes bij productie}

Bij produceren moet je afhankelijk van al deze technieken
keuzes maken over welke technieken het beste is. Hoeveel producten
moet ik produceren en wat kost dat? Het is allemaal afhankelijk 
van de eisen die aan het product worden gesteld.

\begin{itemize}
  \item Kosten
  \item snelheid
  \item kwaliteit
  \item milieu
  \item veiligheid
  \item functionaliteit
  \item materiaal
  \item tolerantie
  \item oppervlaktekwaliteit
  \item aantal
  \item onderhoud
\end{itemize}

al deze factoren zijn belangrijk bij het kiezen van een productietechniek.


\section{Passing}

Passing is een maat voor hoe goed twee oppervlakken op
 elkaar aansluiten.

 \begin{itemize}
    \item Losse Passing: Er is nog speling tussen de twee oppervlakken.
    \item Nauw Passing: De twee oppervlakken sluiten goed op elkaar aan, er is
      bijna geen speling meer.
    \item PersPassing: De twee oppervlakken worden in elkaar gedrukt.
 \end{itemize}

 \section{Tolerantie}
    Toleranties worden geklassificeerd via diagrammen 

    \begin{itemize}
        \item inwendige
        \item uitwendige
        \item passing 
    \end{itemize}

    \begin{figure}[ht]
      \centering
      \includegraphics[width=0.32\textwidth]{image1.png}\hfill
      \includegraphics[width=0.32\textwidth]{image2.png}\hfill
      \includegraphics[width=0.32\textwidth]{image3.png}
      \caption{Tolerantie diagrammen voor inwendige, uitwendige en passing}
      \label{fig:tolerantie_diagrammen}
    \end{figure}

    Je moet kiezen welke tolerantie nodig is voor een product.
    Precieze toleranties zijn duurder om te produceren.

    \begin{figure}[ht]
      \centering
      \includegraphics[width=0.5\textwidth]{image5.png}
      \caption{Kosten vs Tolerantie}
      \label{fig:kosten_vs_tolerantie}
    \end{figure}

    \section{Oppervlaktekwaliteit}
    Oppervlaktekwaliteit moet ook gekozen worden bij het produceren van een product.
    \begin{itemize}
      \item Een ruw oppervlak is goedkoper om te produceren.
      \item Een glad oppervlak is duurder om te produceren.
      \item Soms is een glad oppervlak nodig voor de functionaliteit van het product.
      \item Textuur kan ook functioneel zijn (antislip, esthetisch, \ldots). bv: een
        handvat, keyboard, tafels, pennen, \ldots
    \end{itemize}

    Oppervlaktekwaliteit wordt uitgedrukt in ruwheid.
    Ra, Rz, Rmax

\begin{figure}[ht]
  \centering
  \includegraphics[width=0.4\textwidth]{image6.png}\hfill
  \includegraphics[width=0.4\textwidth]{image7.png}
  \caption{Voorbeelden van oppervlaktekwaliteiten}
  \label{fig:oppervlaktekwaliteiten}
\end{figure}

  verschillende productietechnieken hebben verschillende oppervlaktekwaliteiten.



  \chapter{Materialen}

  Dit hoofdstuk gaat over de efecten van verschillende materialen
  op productietechnieken.
  Ook het effect van de productietechniek op het materiaal zelf.
  \newline

  \begin{figure}[ht]
  \centering
  \includegraphics[width=0.5\textwidth]{image9.png}
  \caption{Effect van thermische process op materialen}
  \label{fig:image9.png}
\end{figure}

  Er zijn verschillende soorten materialen die je kunt kiezen.
  Allemaal hebben ze verschillende materiaaleigenschappen.
  \begin{itemize}
    \item Metalen
    \item Kunststoffen
    \item Keramiek
    \item Composieten
  \end{itemize}

  \subsection{Vervorming}

  Je hebt elastische en plastische vervormingen in een materiaal die gebeureren
  tijden het bewerken van een materiaal.
  \begin{itemize}
    \item Elastische vervorming: Het materiaal keert terug naar zijn originele vorm
      nadat de kracht is weggenomen.
    \item Plastische vervorming: Het materiaal blijft vervormd nadat de kracht is
      weggenomen.
  \end{itemize}
  Elastische vervorming gegeven door hooke's law:
  \frm{Hooke's Law}{\sigma = E \cdot \varepsilon}{waarbij $\sigma$ de spanning is, $E$ de elasticiteitsmodulus en $\varepsilon$ de rek.}


  \chapter{Verspanen:Algemeen}
  Verspannen is het verwijderen van materiaal van een werkstuk.
  Dit kan door Boren, Frezen, Draaien, Slijpen, \ldots
  \newline
  Je begint met een ruw werkstuk en je verwijdert materiaal tot je 
  de gewenste vorm en afmetingen hebt.
  \newline

  \textbf{Voordelen}
  \begin{itemize}
    \item Hoge precisie
    \item Goede tolerantie
    \item Goede oppervlaktekwaliteit
    \item Flexibiliteit in ontwerp
  \end{itemize}

  \textbf{Nadelen}
  \begin{itemize}
    \item Materiaalverlies
    \item Hogere kosten bij grote aantallen
    \item Langere productietijd
    \item energieintensies
    \item Vervuilend (spanen, koelvloeistof)
  \end{itemize}

  Bij verspannen kunnen verschillende tools gebruikt worden.
  Deze tools hebben verschillende snijvlakken en geometrieën
  die geschikt zijn voor verschillende materialen en bewerkingen.
  \newline
  \begin{itemize}
    \item bijtel
    \item frees
    \item boor
    \item slijpschijf
    \item 
  \end{itemize}

  \section{bijtelbewerkingen}

  Bij bijtelbewerkingen wordt materiaal verwijderd door een scherpe
  bijtel over het werkstuk te bewegen.
  \begin{figure}[H]
  \centering
  \includegraphics[width=0.8\textwidth]{image10.png}
  \caption{Verwijdering van materiaal door een bijtel -> creert spanen}
  \label{fig:image10.png}
\end{figure}

  Bij het verspannen met een bijtel ontstaan er spanen.
  Spanen zijn kleine stukjes materiaal die worden verwijderd
  van het werkstuk.
  De grote van de spanen wordt bepaald door de snedediepte, de voeding, de spaanhoek, de wrijvingscoëfficiënt \ldots
  \newline
  Zometeen meer in detail hierover

  Spanen is een plastische vervorming van de spanen
  maar het oppervlak van het werkstuk ondergaat ook een elastische vervorming.
  Dit kan leiden tot oppervlaktefouten zoals ruwheid, hardheid, \ldots
  \newline




\begin{figure}[ht]
  \centering
  \includegraphics[width=0.8\textwidth]{image11.png}
  \caption{}
  \label{fig:image11}
\end{figure}

\frm{Afschuifhoek}{\phi = 45^{\circ} + \dfrac{\gamma}{2} - \dfrac{\mu}{2}}{waarbij $\gamma$ de spaanhoek is en $\mu$ de wrijvingshoek tussen spaan en snijvlak.}

De afschuifhoek bepaalt de richting waarin de spaan wordt afgesneden.
Een grotere afschuifhoek leidt tot een betere spaanvorming en minder kracht

\frm{afschuifvlak A}{A = \frac{b*h}{\sin(\phi)}}{waarbij $b$ de breedte, $h$ de hoogte en $\phi$ de afschuifhoek.}

\frm{afschuifkracht}{F = A*\tau}{waarbij $A$ het afschuifvlak is en $\tau$ de schuifspanning van het materiaal.}

Grotere spaanhoek -> kleiner afschuifvlak -> minder kracht nodig om spaan te vormen.

\begin{itemize}
  \item Spaanhoek -> groter -> minder kracht nodig. Tussen -10° en 30°
  \item Wighoek -> groter -> meer kracht nodig. Zo groot mogelijk
    \begin{figure}[ht]
    \centering
    \includegraphics[width=0.4\textwidth]{image12.png}
    \caption{Wighoek}
    \label{fig:image12}
  \end{figure}
  \item Vrijloophoek -> groter -> minder wrijving tussen werkstuk en bijtel. Tussen 6° en 10°
    \begin{figure}[ht]
    \centering
    \includegraphics[width=0.4\textwidth]{image14.png}
    \caption{Vrijloophoek}
    \label{fig:image14}
  \end{figure}
  \item Snedediepte -> groter -> meer kracht nodig
  \item Voeding -> groter -> meer kracht nodig
\end{itemize}

De spaanhoek is enorm belangrijk. Een grote spaanhoek snijdt makkelijk
materialen zoals aluminium, koper, kunststof.
Voor hardere materialen zoals staal is een kleinere spaanhoek nodig
omdat het materiaal anders te hard is om te snijden en je moet veel te veel kracht zetten.
Negatieve spaanhoeken worden gebruikt voor zeer harde materialen.

\begin{figure}[ht]
  \centering
  \includegraphics[width=0.5\textwidth]{image15.png}
  \caption{Verschillende spaanhoeken voor verschillende materialen}
  \label{fig:image15.png}
\end{figure}

\subsection{Soorten spaanvorming}


\section{Beweging, snelheden en voedingen, temperaturen, slijtage}

Er zijn verschillende factoren die de kracht op je werkstuk bepalen
\begin{itemize}
  \item Snijsnelheid (v)
  \item Voeding (f)
  \item Snedediepte (a)
  \item Snedebreedte (b)
  \item Snededikte (h)
\end{itemize}

\begin{figure}[ht]
  \centering
  \includegraphics[width=0.4\textwidth]{image16.png}
  \caption{Snededoorsnede bij bijtelbewerking}
  \label{fig:image16.png}
\end{figure}

\frm{Snededoorsnede}{A_d = a \cdot b}{waarbij $a$ de snedediepte is en $b$ de snedebreedte.}


\begin{figure}[ht]
  \centering
  \includegraphics[width=0.4\textwidth]{image17.png}
  \caption{}
  \label{fig:image17.png}
\end{figure}

\subsection{Krachten}



\frm{Snijsnelheid}{v = \pi \cdot d \cdot n}{waarbij $d$ de diameter is en $n$ het toerental in omwentelingen per minuut.}



\section{factoren bij bijtelbewerking}
- koelvloeistof
- Build up edge
- Warmte 
- Slijtage

\section{Snijmaterialen}


  \chapter{Verspanen:Boren}



  \chapter{Verspanen:Frezen}

  \chapter{Verspanen:Hybridetechnieken}

  \chapter{Verspanen:Slijpen}
  \chapter{Fysische en Chemische  afnemende bewerkingen}

  \chapter{Scheiden}
  \chapter{Automatiseren \& machinekeuze}

  \chapter{Productie werkvoorbereiding}


  \chapter{productiegericht ontwerpen}





\end{document}